\section{参数高效微调技术}

传统的全参数微调需要更新模型的所有参数,对于大模型来说成本极高。参数高效微调(Parameter-Efficient Fine-Tuning, PEFT)技术通过只更新少量参数就能达到接近全参数微调的效果,大幅降低了微调成本。

\subsection{LoRA (Low-Rank Adaptation)}

\textbf{概念解释}:LoRA 是 Microsoft 在 2021 年提出的参数高效微调方法。其核心思想是:对于预训练模型的权重矩阵 $\mathbf{W}$,不直接更新它,而是学习一个低秩分解的增量 $\Delta\mathbf{W}$,使得 $\mathbf{W} + \Delta\mathbf{W}$ 能够适应新任务。

\textbf{数学公式}:

对于原始权重矩阵 $\mathbf{W} \in \mathbb{R}^{d \times k}$,LoRA 将其更新分解为:
\begin{equation}
\mathbf{W}' = \mathbf{W} + \Delta\mathbf{W} = \mathbf{W} + \mathbf{B}\mathbf{A}
\end{equation}

其中:
\begin{itemize}
    \item $\mathbf{A} \in \mathbb{R}^{r \times k}$:低秩矩阵,随机初始化
    \item $\mathbf{B} \in \mathbb{R}^{d \times r}$:低秩矩阵,初始化为零
    \item $r \ll \min(d, k)$:秩(rank),通常 $r \in \{4, 8, 16, 32, 64\}$
\end{itemize}

在前向传播时:
\begin{equation}
\mathbf{h} = \mathbf{W}'\mathbf{x} = (\mathbf{W} + \mathbf{B}\mathbf{A})\mathbf{x} = \mathbf{W}\mathbf{x} + \mathbf{B}(\mathbf{A}\mathbf{x})
\end{equation}

\textbf{参数量对比}:
\begin{itemize}
    \item 全参数微调:$d \times k$ 个参数
    \item LoRA:$r \times (d + k)$ 个参数
    \item 参数减少比例:$\frac{r(d + k)}{dk} = r\left(\frac{1}{k} + \frac{1}{d}\right)$
\end{itemize}

当 $r = 8$,$d = 4096$,$k = 4096$ 时,参数量从 $16,777,216$ 减少到 $65,536$,减少了约 $256$ 倍。

\textbf{算法原理}:
\begin{algorithm}
\caption{LoRA 微调算法}
\begin{algorithmic}[1]
\REQUIRE 预训练模型权重 $\mathbf{W}$,训练数据 $\mathcal{D}$,秩 $r$,学习率 $\eta$
\ENSURE LoRA 权重 $\mathbf{A}$ 和 $\mathbf{B}$
\STATE 初始化 $\mathbf{A}$ 为随机小值,$\mathbf{B}$ 为零矩阵
\STATE 冻结原始权重 $\mathbf{W}$
\REPEAT
    \FOR{每个批次 $(\mathbf{x}, \mathbf{y}) \in \mathcal{D}$}
        \STATE 前向传播:$\mathbf{h} = \mathbf{W}\mathbf{x} + \mathbf{B}(\mathbf{A}\mathbf{x})$
        \STATE 计算损失:$\mathcal{L} = \text{loss}(\mathbf{h}, \mathbf{y})$
        \STATE 反向传播,只更新 $\mathbf{A}$ 和 $\mathbf{B}$
        \STATE $\mathbf{A} \leftarrow \mathbf{A} - \eta \nabla_{\mathbf{A}}\mathcal{L}$
        \STATE $\mathbf{B} \leftarrow \mathbf{B} - \eta \nabla_{\mathbf{B}}\mathcal{L}$
    \ENDFOR
\UNTIL{收敛}
\end{algorithmic}
\end{algorithm}

\textbf{代码实现}:

\begin{lstlisting}[caption=LoRA 从零实现]
import torch
import torch.nn as nn
import torch.nn.functional as F

class LoRALayer(nn.Module):
    """LoRA 层实现"""
    
    def __init__(self, in_features, out_features, rank=8, alpha=16, dropout=0.0):
        """
        参数:
            in_features: 输入特征维度
            out_features: 输出特征维度
            rank: LoRA 的秩
            alpha: 缩放因子,通常等于 rank
            dropout: Dropout 概率
        """
        super().__init__()
        self.rank = rank
        self.alpha = alpha
        self.scaling = alpha / rank
        
        # LoRA 矩阵 A 和 B
        self.lora_A = nn.Parameter(torch.randn(rank, in_features) * 0.02)
        self.lora_B = nn.Parameter(torch.zeros(out_features, rank))
        
        # Dropout
        self.dropout = nn.Dropout(dropout) if dropout > 0 else nn.Identity()
        
        # 原始权重(冻结)
        self.weight = None  # 将在外部设置
    
    def forward(self, x, weight):
        """
        前向传播
        
        参数:
            x: 输入张量 (batch_size, ..., in_features)
            weight: 原始权重矩阵 (out_features, in_features)
        """
        # 原始输出
        original_output = F.linear(x, weight)
        
        # LoRA 输出: B @ (A @ x)
        x_dropout = self.dropout(x)
        lora_output = F.linear(
            F.linear(x_dropout, self.lora_A),
            self.lora_B
        )
        
        # 缩放并相加
        return original_output + self.scaling * lora_output

class LoRALinear(nn.Module):
    """包装的 LoRA 线性层"""
    
    def __init__(self, linear_layer, rank=8, alpha=16, dropout=0.0):
        super().__init__()
        self.original = linear_layer
        self.lora = LoRALayer(
            linear_layer.in_features,
            linear_layer.out_features,
            rank,
            alpha,
            dropout
        )
        # 冻结原始权重
        for param in self.original.parameters():
            param.requires_grad = False
    
    def forward(self, x):
        return self.lora(x, self.original.weight)

# 使用示例
# 假设我们有一个预训练的线性层
pretrained_linear = nn.Linear(768, 3072)

# 包装为 LoRA 层
lora_linear = LoRALinear(pretrained_linear, rank=8, alpha=16)

# 前向传播
x = torch.randn(32, 768)  # batch_size=32
output = lora_linear(x)
print(f"输入形状: {x.shape}")
print(f"输出形状: {output.shape}")

# 检查可训练参数
total_params = sum(p.numel() for p in lora_linear.parameters() if p.requires_grad)
print(f"可训练参数数量: {total_params}")  # 8 * (768 + 3072) = 30720
print(f"原始参数数量: {pretrained_linear.weight.numel()}")  # 768 * 3072 = 2359296
print(f"参数减少比例: {pretrained_linear.weight.numel() / total_params:.2f}x")
\end{lstlisting}

\textbf{使用 PEFT 库实现}:

\begin{lstlisting}[caption=使用 Hugging Face PEFT 库]
from transformers import AutoModelForCausalLM, AutoTokenizer
from peft import LoraConfig, get_peft_model, TaskType

# 加载预训练模型
model_name = "meta-llama/Llama-2-7b-hf"  # 示例模型
model = AutoModelForCausalLM.from_pretrained(
    model_name,
    torch_dtype=torch.float16,
    device_map="auto"
)
tokenizer = AutoTokenizer.from_pretrained(model_name)

# 配置 LoRA
lora_config = LoraConfig(
    task_type=TaskType.CAUSAL_LM,
    r=8,  # rank
    lora_alpha=16,  # alpha
    lora_dropout=0.1,
    target_modules=["q_proj", "v_proj", "k_proj", "o_proj"],  # 目标模块
    bias="none"
)

# 应用 LoRA
model = get_peft_model(model, lora_config)

# 打印可训练参数
model.print_trainable_parameters()
# trainable params: 4,194,304 || all params: 6,738,415,616 || trainable\%: 0.06

# 训练(只更新 LoRA 参数)
# ... 训练代码 ...
\end{lstlisting}

\textbf{适用场景}:
\begin{itemize}
    \item 资源受限环境下的模型微调
    \item 需要快速适应多个任务的场景
    \item 模型服务化部署前的快速迭代
\end{itemize}

\textbf{优势与局限}:

\textbf{优势}:
\begin{itemize}
    \item 参数量大幅减少(通常减少 100-1000 倍)
    \item 训练速度快,内存占用低
    \item 可以保存多个 LoRA 适配器,快速切换任务
    \item 效果接近全参数微调
\end{itemize}

\textbf{局限}:
\begin{itemize}
    \item 在某些复杂任务上可能不如全参数微调
    \item 需要选择合适的 rank 和目标模块
    \item 多个 LoRA 适配器组合时可能产生冲突
\end{itemize}

\subsection{QLoRA (Quantized LoRA)}

\textbf{概念解释}:QLoRA 是 LoRA 的量化版本,通过 4-bit 量化进一步降低内存占用,使得在消费级 GPU 上微调大模型成为可能。

\textbf{量化原理}:

QLoRA 使用 4-bit NormalFloat (NF4) 量化,将 32-bit 浮点数映射到 4-bit 整数:

\begin{equation}
Q(x) = \text{round}\left(\frac{x - \text{offset}}{\text{scale}}\right) \times \text{scale} + \text{offset}
\end{equation}

其中:
\begin{itemize}
    \item $\text{offset}$:量化偏移量
    \item $\text{scale}$:量化缩放因子
    \item $Q(x)$:量化后的值
\end{itemize}

\textbf{与 LoRA 的区别}:
\begin{itemize}
    \item LoRA:原始权重保持 FP16/BF16,只量化激活值
    \item QLoRA:原始权重量化为 4-bit,激活值量化为 8-bit,LoRA 权重保持 16-bit
    \item 内存节省:QLoRA 可以节省约 75\% 的内存
\end{itemize}

\textbf{数学表达式}:

对于量化权重 $\mathbf{W}_Q$ 和 LoRA 增量:
\begin{equation}
\mathbf{W}' = \mathbf{W}_Q + \frac{\alpha}{r}\mathbf{B}\mathbf{A}
\end{equation}

其中 $\mathbf{W}_Q$ 是量化后的权重,在推理时动态反量化。

\textbf{代码实现}:

\begin{lstlisting}[caption=使用 bitsandbytes 和 PEFT 实现 QLoRA]
from transformers import AutoModelForCausalLM, AutoTokenizer, BitsAndBytesConfig
from peft import LoraConfig, get_peft_model, prepare_model_for_kbit_training
import torch

# 4-bit 量化配置
bnb_config = BitsAndBytesConfig(
    load_in_4bit=True,
    bnb_4bit_quant_type="nf4",
    bnb_4bit_compute_dtype=torch.float16,
    bnb_4bit_use_double_quant=True,  # 嵌套量化
)

# 加载模型(自动量化)
model_name = "meta-llama/Llama-2-7b-hf"
model = AutoModelForCausalLM.from_pretrained(
    model_name,
    quantization_config=bnb_config,
    device_map="auto"
)
tokenizer = AutoTokenizer.from_pretrained(model_name)

# 准备模型进行 k-bit 训练
model = prepare_model_for_kbit_training(model)

# LoRA 配置
lora_config = LoraConfig(
    r=8,
    lora_alpha=16,
    target_modules=["q_proj", "v_proj"],
    lora_dropout=0.1,
    bias="none",
    task_type="CAUSAL_LM"
)

# 应用 LoRA
model = get_peft_model(model, lora_config)

# 打印内存使用
model.print_trainable_parameters()
# trainable params: 4,194,304 || all params: 6,738,415,616 || trainable\%: 0.06

# 内存占用对比:
# FP16 LoRA: ~14GB
# QLoRA: ~6GB (节省约 57\%)
\end{lstlisting}

\textbf{内存优化效果}:

对于 7B 参数的模型:
\begin{itemize}
    \item 全参数微调(FP16):约 28GB
    \item LoRA(FP16):约 14GB
    \item QLoRA(4-bit):约 6GB
\end{itemize}

\textbf{优势与局限}:

\textbf{优势}:
\begin{itemize}
    \item 内存占用极低,可在消费级 GPU 上运行
    \item 训练速度与 LoRA 相当
    \item 效果损失很小(通常 < 1\%)
\end{itemize}

\textbf{局限}:
\begin{itemize}
    \item 量化可能引入轻微的性能损失
    \item 需要支持 4-bit 量化的硬件
    \item 某些操作可能不支持量化
\end{itemize}

\subsection{PEFT 框架}

\textbf{概念解释}:PEFT(Parameter-Efficient Fine-Tuning)是 Hugging Face 提供的统一框架,支持多种参数高效微调方法。

\textbf{支持的方法}:
\begin{itemize}
    \item LoRA
    \item Prefix Tuning
    \item P-Tuning v2
    \item Prompt Tuning
    \item AdaLoRA
    \item 自定义方法
\end{itemize}

\textbf{统一接口}:

\begin{lstlisting}[caption=PEFT 框架使用示例]
from peft import (
    LoraConfig,
    PrefixTuningConfig,
    PromptTuningConfig,
    get_peft_model
)

# LoRA 配置
lora_config = LoraConfig(
    r=8,
    lora_alpha=16,
    target_modules=["q_proj", "v_proj"]
)

# Prefix Tuning 配置
prefix_config = PrefixTuningConfig(
    task_type="CAUSAL_LM",
    num_virtual_tokens=20
)

# Prompt Tuning 配置
prompt_config = PromptTuningConfig(
    task_type="CAUSAL_LM",
    num_virtual_tokens=20
)

# 统一接口应用
model = get_peft_model(model, lora_config)  # 或 prefix_config, prompt_config
\end{lstlisting}

\subsection{LoRA 变体方法}

\textbf{AdaLoRA}:自适应 LoRA,动态调整 rank。

\textbf{数学公式}:
\begin{equation}
\mathbf{W}' = \mathbf{W} + \sum_{i=1}^{r} s_i \mathbf{b}_i \mathbf{a}_i^T
\end{equation}

其中 $s_i$ 是重要性分数,用于剪枝不重要的 LoRA 模块。

\textbf{DoRA (Weight-Decomposed Low-Rank Adaptation)}:将权重分解为幅度和方向。

\textbf{数学公式}:
\begin{equation}
\mathbf{W}' = \frac{m}{||\mathbf{W} + \Delta\mathbf{W}||_c} (\mathbf{W} + \Delta\mathbf{W})
\end{equation}

其中 $m$ 是可学习的幅度参数,$||\cdot||_c$ 是列范数。

\subsection{参数效率对比}

\begin{table}[h]
\centering
\caption{不同 PEFT 方法的参数效率对比}
\begin{tabular}{lcccc}
\toprule
方法 & 参数量比例 & 内存占用 & 训练速度 & 效果 \\
\midrule
全参数微调 & 100\% & 高 & 慢 & 最佳 \\
\midrule
LoRA & 0.1-1\% & 中 & 快 & 接近最佳 \\
\midrule
QLoRA & 0.1-1\% & 低 & 快 & 接近最佳 \\
\midrule
Prefix Tuning & 0.1-0.5\% & 低 & 快 & 良好 \\
\midrule
P-Tuning v2 & 0.1-1\% & 中 & 中 & 良好 \\
\midrule
AdaLoRA & 0.1-1\% & 中 & 中 & 接近最佳 \\
\bottomrule
\end{tabular}
\end{table}

\section{监督微调技术}

\subsection{SFT (Supervised Fine-Tuning)}

\textbf{概念解释}:SFT 是在有标签数据上对预训练模型进行微调,使模型适应特定任务。

\textbf{数据格式}:

\begin{lstlisting}[caption=SFT 数据格式示例]
# JSON 格式
{
    "instruction": "将以下文本翻译成英文",
    "input": "你好,世界",
    "output": "Hello, world"
}

# 对话格式
{
    "messages": [
        {"role": "user", "content": "什么是机器学习?"},
        {"role": "assistant", "content": "机器学习是..."}
    ]
}
\end{lstlisting}

\textbf{训练流程}:

\begin{lstlisting}[caption=SFT 训练示例]
from transformers import (
    AutoModelForCausalLM,
    AutoTokenizer,
    TrainingArguments,
    Trainer,
    DataCollatorForLanguageModeling
)
from datasets import load_dataset
import torch

# 加载模型和分词器
model_name = "meta-llama/Llama-2-7b-hf"
model = AutoModelForCausalLM.from_pretrained(model_name)
tokenizer = AutoTokenizer.from_pretrained(model_name)
tokenizer.pad_token = tokenizer.eos_token

# 准备数据
def format_prompt(example):
    prompt = f"### Instruction:\n{example['instruction']}\n\n### Input:\n{example['input']}\n\n### Response:\n{example['output']}"
    return {"text": prompt}

dataset = load_dataset("json", data_files="train.json")
dataset = dataset.map(format_prompt)

def tokenize_function(examples):
    return tokenizer(
        examples["text"],
        truncation=True,
        max_length=512,
        padding="max_length"
    )

tokenized_dataset = dataset.map(tokenize_function, batched=True)

# 训练参数
training_args = TrainingArguments(
    output_dir="./results",
    num_train_epochs=3,
    per_device_train_batch_size=4,
    gradient_accumulation_steps=4,
    learning_rate=2e-5,
    fp16=True,
    logging_steps=100,
    save_steps=500,
)

# 数据整理器
data_collator = DataCollatorForLanguageModeling(
    tokenizer=tokenizer,
    mlm=False  # 因果语言建模
)

# 训练器
trainer = Trainer(
    model=model,
    args=training_args,
    train_dataset=tokenized_dataset["train"],
    data_collator=data_collator,
)

# 开始训练
trainer.train()
\end{lstlisting}

\subsection{指令微调(Instruction Tuning)}

\textbf{概念解释}:指令微调通过大量指令-输出对训练模型,使模型能够理解和遵循指令。

\textbf{数据构建}:

\begin{lstlisting}[caption=指令数据构建示例]
# 指令数据格式
instruction_data = [
    {
        "instruction": "解释以下概念",
        "input": "量子计算",
        "output": "量子计算是利用量子力学原理..."
    },
    {
        "instruction": "总结以下文本",
        "input": "长文本...",
        "output": "总结内容..."
    }
]

# 使用 Alpaca 格式
def format_alpaca(example):
    return {
        "text": f"""Below is an instruction that describes a task. Write a response that appropriately completes the request.

### Instruction:
{example['instruction']}

### Input:
{example['input']}

### Response:
{example['output']}"""
    }
\end{lstlisting}

\textbf{效果评估}:

\begin{itemize}
    \item 指令遵循准确率
    \item 输出质量评分
    \item 任务完成率
\end{itemize}

\subsection{对话微调(Chat Fine-Tuning)}

\textbf{概念解释}:对话微调专门针对多轮对话场景,训练模型进行自然对话。

\textbf{多轮对话数据处理}:

\begin{lstlisting}[caption=对话数据处理示例]
def format_chat(messages):
    """格式化多轮对话"""
    formatted = ""
    for msg in messages:
        role = msg["role"]
        content = msg["content"]
        if role == "user":
            formatted += f"User: {content}\n"
        elif role == "assistant":
            formatted += f"Assistant: {content}\n"
    return formatted

# 对话数据示例
chat_data = {
    "messages": [
        {"role": "user", "content": "你好"},
        {"role": "assistant", "content": "你好!有什么可以帮助你的吗?"},
        {"role": "user", "content": "介绍一下Python"},
        {"role": "assistant", "content": "Python是一种高级编程语言..."}
    ]
}
\end{lstlisting}

\section{推理加速技术}

\subsection{vLLM (Very Large Language Model)}

\textbf{概念解释}:vLLM 是专门为大模型推理优化的服务框架,通过 PagedAttention 等技术大幅提升吞吐量。

\textbf{PagedAttention 原理}:

传统 Attention 的 KV Cache 是连续的,导致内存碎片化。PagedAttention 将 KV Cache 分页管理:

\begin{equation}
\text{Attention}(\mathbf{Q}, \mathbf{K}, \mathbf{V}) = \text{softmax}\left(\frac{\mathbf{Q}\mathbf{K}^T}{\sqrt{d_k}}\right)\mathbf{V}
\end{equation}

PagedAttention 将 $\mathbf{K}$ 和 $\mathbf{V}$ 存储在非连续的页面中,按需分配。

\textbf{KV Cache 优化}:

\begin{lstlisting}[caption=vLLM 使用示例]
from vllm import LLM, SamplingParams

# 初始化模型
llm = LLM(
    model="meta-llama/Llama-2-7b-hf",
    tensor_parallel_size=1,
    gpu_memory_utilization=0.9
)

# 采样参数
sampling_params = SamplingParams(
    temperature=0.7,
    top_p=0.9,
    max_tokens=512
)

# 批量推理
prompts = [
    "什么是机器学习?",
    "解释一下深度学习",
    "Python 的特点是什么?"
]

outputs = llm.generate(prompts, sampling_params)

for output in outputs:
    print(f"Prompt: {output.prompt}")
    print(f"Generated: {output.outputs[0].text}\n")
\end{lstlisting}

\textbf{性能优势}:
\begin{itemize}
    \item 吞吐量提升 2-4 倍
    \item 内存利用率提高 50-80\%
    \item 支持连续批处理
\end{itemize}

\section{SGLang、vLLM、Transformer、PyTorch 对比}

\subsection{框架定位与用途}

\textbf{PyTorch}:
\begin{itemize}
    \item \textbf{定位}:深度学习框架,提供底层张量计算和自动微分
    \item \textbf{用途}:模型训练、研究、原型开发
    \item \textbf{特点}:灵活、动态图、丰富的生态系统
\end{itemize}

\textbf{Transformer}:
\begin{itemize}
    \item \textbf{定位}:模型架构,定义网络结构
    \item \textbf{用途}:构建编码器-解码器、自注意力等结构
    \item \textbf{特点}:标准化的注意力机制实现
\end{itemize}

\textbf{vLLM}:
\begin{itemize}
    \item \textbf{定位}:推理服务框架,优化推理性能
    \item \textbf{用途}:生产环境部署、高吞吐量推理服务
    \item \textbf{特点}:PagedAttention、连续批处理、高吞吐量
\end{itemize}

\textbf{SGLang}:
\begin{itemize}
    \item \textbf{定位}:结构化生成框架,优化复杂提示推理
    \item \textbf{用途}:函数调用、JSON 生成、批量推理
    \item \textbf{特点}:RadixAttention、结构化生成、高性能
\end{itemize}

\subsection{性能对比}

\begin{table}[h]
\centering
\begin{tabular}{lcccc}
\toprule
\textbf{特性} & \textbf{PyTorch} & \textbf{Transformer} & \textbf{vLLM} & \textbf{SGLang} \\
\midrule
训练速度 & 中等 & 中等 & 不支持训练 & 不支持训练 \\
\midrule
推理吞吐量 & 低 & 低 & 高 & 很高 \\
\midrule
内存效率 & 中等 & 中等 & 高 & 很高 \\
\midrule
结构化生成 & 需手动实现 & 需手动实现 & 需手动实现 & 原生支持 \\
\midrule
前缀共享 & 需手动实现 & 需手动实现 & 部分支持 & 自动优化 \\
\midrule
易用性 & 高 & 中等 & 高 & 高 \\
\bottomrule
\end{tabular}
\caption{框架特性对比}
\end{table}

\subsection{优劣对比}

\textbf{PyTorch}:

\textbf{优势}:
\begin{itemize}
    \item 灵活性强,支持动态图
    \item 丰富的预训练模型和工具
    \item 活跃的社区和文档
    \item 适合研究和实验
\end{itemize}

\textbf{劣势}:
\begin{itemize}
    \item 推理性能一般
    \item 内存占用较大
    \item 生产部署需要额外优化
\end{itemize}

\textbf{Transformer}:

\textbf{优势}:
\begin{itemize}
    \item 标准化实现,易于理解
    \item 广泛使用,兼容性好
    \item 支持多种注意力机制
\end{itemize}

\textbf{劣势}:
\begin{itemize}
    \item 推理性能未优化
    \item 内存效率一般
    \item 需要手动优化才能用于生产
\end{itemize}

\textbf{vLLM}:

\textbf{优势}:
\begin{itemize}
    \item 高吞吐量推理(2-10 倍提升)
    \item PagedAttention 优化内存
    \item 连续批处理,提高 GPU 利用率
    \item 易于部署和使用
\end{itemize}

\textbf{劣势}:
\begin{itemize}
    \item 不支持训练
    \item 对结构化生成支持有限
    \item 复杂提示场景性能一般
\end{itemize}

\textbf{SGLang}:

\textbf{优势}:
\begin{itemize}
    \item 极高的推理吞吐量(在某些场景下比 vLLM 快 2-5 倍)
    \item RadixAttention 自动前缀共享
    \item 原生支持结构化生成
    \item 复杂提示场景性能优秀
\end{itemize}

\textbf{劣势}:
\begin{itemize}
    \item 不支持训练
    \item 相对较新,生态不如 vLLM 成熟
    \item 主要针对特定场景优化
\end{itemize}

\subsection{使用场景建议}

\textbf{选择 PyTorch}:
\begin{itemize}
    \item 需要训练模型
    \item 研究和实验阶段
    \item 需要高度定制化
\end{itemize}

\textbf{选择 Transformer}:
\begin{itemize}
    \item 需要标准化的注意力实现
    \item 构建自定义模型架构
    \item 学习和理解 Transformer 原理
\end{itemize}

\textbf{选择 vLLM}:
\begin{itemize}
    \item 生产环境部署
    \item 需要高吞吐量推理
    \item 通用文本生成任务
\end{itemize}

\textbf{选择 SGLang}:
\begin{itemize}
    \item 函数调用和 JSON 生成
    \item 批量处理相似请求
    \item 复杂提示场景
    \item 需要极致推理性能
\end{itemize}

\section{训练和推理代码示例}

\subsection{PyTorch 训练和推理}

\textbf{训练示例}:

\begin{lstlisting}[language=Python, caption=PyTorch 训练示例]
import torch
import torch.nn as nn
from transformers import GPT2LMHeadModel, GPT2Tokenizer, Trainer, TrainingArguments

# 1. 加载模型和分词器
model = GPT2LMHeadModel.from_pretrained('gpt2')
tokenizer = GPT2Tokenizer.from_pretrained('gpt2')
tokenizer.pad_token = tokenizer.eos_token

# 2. 准备数据
def prepare_dataset(texts):
    encodings = tokenizer(texts, truncation=True, padding=True, 
                         max_length=512, return_tensors='pt')
    return encodings

train_texts = ["Your training data here..."]
train_dataset = prepare_dataset(train_texts)

# 3. 配置训练参数
training_args = TrainingArguments(
    output_dir='./results',
    num_train_epochs=3,
    per_device_train_batch_size=4,
    gradient_accumulation_steps=2,
    learning_rate=5e-5,
    logging_dir='./logs',
)

# 4. 创建 Trainer
trainer = Trainer(
    model=model,
    args=training_args,
    train_dataset=train_dataset,
)

# 5. 开始训练
trainer.train()

# 6. 保存模型
model.save_pretrained('./saved_model')
tokenizer.save_pretrained('./saved_model')
\end{lstlisting}

\textbf{推理示例}:

\begin{lstlisting}[language=Python, caption=PyTorch 推理示例]
import torch
from transformers import GPT2LMHeadModel, GPT2Tokenizer

# 1. 加载模型
model = GPT2LMHeadModel.from_pretrained('./saved_model')
tokenizer = GPT2Tokenizer.from_pretrained('./saved_model')
model.eval()

# 2. 准备输入
prompt = "The future of AI is"
inputs = tokenizer(prompt, return_tensors='pt')

# 3. 生成
with torch.no_grad():
    outputs = model.generate(
        inputs.input_ids,
        max_length=100,
        num_return_sequences=1,
        temperature=0.7,
        do_sample=True,
    )

# 4. 解码输出
generated_text = tokenizer.decode(outputs[0], skip_special_tokens=True)
print(generated_text)
\end{lstlisting}

\textbf{代码解释}:
\begin{itemize}
    \item \textbf{训练}:使用 Trainer API,配置训练参数,自动处理批次、优化器、学习率调度
    \item \textbf{推理}:使用 \texttt{generate()} 方法,支持多种生成策略(采样、贪婪等)
    \item \textbf{优势}:灵活、易于调试、支持自定义训练循环
    \item \textbf{劣势}:推理性能一般,需要手动优化批处理
\end{itemize}

\subsection{Transformer 架构训练和推理}

\textbf{自定义 Transformer 训练}:

\begin{lstlisting}[language=Python, caption=Transformer 架构训练示例]
import torch
import torch.nn as nn
from transformers import Transformer, TransformerConfig

# 1. 定义 Transformer 配置
config = TransformerConfig(
    vocab_size=50257,
    d_model=768,
    nhead=12,
    num_encoder_layers=12,
    num_decoder_layers=12,
    dim_feedforward=3072,
)

# 2. 创建模型
model = Transformer(config)

# 3. 定义损失函数和优化器
criterion = nn.CrossEntropyLoss()
optimizer = torch.optim.Adam(model.parameters(), lr=1e-4)

# 4. 训练循环
model.train()
for epoch in range(num_epochs):
    for batch in dataloader:
        src, tgt = batch
        
        # 前向传播
        output = model(src, tgt)
        loss = criterion(output.view(-1, vocab_size), tgt.view(-1))
        
        # 反向传播
        optimizer.zero_grad()
        loss.backward()
        optimizer.step()
\end{lstlisting}

\textbf{Transformer 推理}:

\begin{lstlisting}[language=Python, caption=Transformer 架构推理示例]
import torch

# 1. 设置为评估模式
model.eval()

# 2. 编码输入
src = tokenizer.encode("Hello, world!")
src = torch.tensor([src])

# 3. 生成(自回归)
tgt = torch.zeros((1, 1), dtype=torch.long)
for i in range(max_length):
    output = model(src, tgt)
    next_token = output[:, -1, :].argmax(dim=-1)
    tgt = torch.cat([tgt, next_token.unsqueeze(1)], dim=1)
    
    if next_token == eos_token_id:
        break

# 4. 解码
generated = tokenizer.decode(tgt[0].tolist())
\end{lstlisting}

\textbf{代码解释}:
\begin{itemize}
    \item \textbf{训练}:手动实现训练循环,完全控制训练过程
    \item \textbf{推理}:自回归生成,逐步预测下一个 token
    \item \textbf{优势}:理解底层机制,可以自定义注意力等组件
    \item \textbf{劣势}:需要手动实现很多功能,代码复杂
\end{itemize}

\subsection{vLLM 推理服务}

\textbf{服务部署}:

\begin{lstlisting}[language=Python, caption=vLLM 推理服务示例]
from vllm import LLM, SamplingParams

# 1. 加载模型
llm = LLM(
    model="meta-llama/Llama-2-7b-hf",
    tensor_parallel_size=1,  # 单 GPU
    gpu_memory_utilization=0.9,  # GPU 内存使用率
)

# 2. 配置采样参数
sampling_params = SamplingParams(
    temperature=0.7,
    top_p=0.9,
    max_tokens=100,
)

# 3. 批量推理
prompts = [
    "The future of AI is",
    "Machine learning is",
    "Deep learning enables",
]

outputs = llm.generate(prompts, sampling_params)

# 4. 处理输出
for output in outputs:
    prompt = output.prompt
    generated_text = output.outputs[0].text
    print(f"Prompt: {prompt}")
    print(f"Generated: {generated_text}\n")
\end{lstlisting}

\textbf{API 服务}:

\begin{lstlisting}[language=Python, caption=vLLM API 服务示例]
from vllm.engine.arg_utils import AsyncEngineArgs
from vllm.engine.async_llm_engine import AsyncLLMEngine
from vllm.sampling_params import SamplingParams
import asyncio

# 1. 初始化异步引擎
engine_args = AsyncEngineArgs(
    model="meta-llama/Llama-2-7b-hf",
    tensor_parallel_size=1,
)
engine = AsyncLLMEngine.from_engine_args(engine_args)

# 2. 异步生成函数
async def generate_async(prompt: str):
    sampling_params = SamplingParams(temperature=0.7, max_tokens=100)
    request_id = "0"
    
    async for request_output in engine.generate(prompt, sampling_params, request_id):
        if request_output.finished:
            return request_output.outputs[0].text

# 3. 使用示例
async def main():
    result = await generate_async("The future of AI is")
    print(result)

asyncio.run(main())
\end{lstlisting}

\textbf{代码解释}:
\begin{itemize}
    \item \textbf{批量推理}:自动批处理,PagedAttention 优化内存
    \item \textbf{API 服务}:异步引擎,支持高并发
    \item \textbf{优势}:高吞吐量、内存高效、易于部署
    \item \textbf{劣势}:不支持训练,需要预训练模型
\end{itemize}

\subsection{SGLang 推理服务}

\textbf{基础推理}:

\begin{lstlisting}[language=Python, caption=SGLang 基础推理示例]
import sglang as sgl

# 1. 加载模型
runtime = sgl.Runtime(model_path="meta-llama/Llama-2-7b-hf")

# 2. 简单生成
prompt = "The future of AI is"
response = runtime.generate(prompt, max_new_tokens=100, temperature=0.7)
print(response.text)
\end{lstlisting}

\textbf{结构化生成(JSON)}:

\begin{lstlisting}[language=Python, caption=SGLang JSON 生成示例]
import sglang as sgl
import json

# 1. 定义 JSON Schema
schema = {
    "type": "object",
    "properties": {
        "name": {"type": "string"},
        "age": {"type": "integer"},
        "city": {"type": "string"}
    }
}

# 2. 创建运行时
runtime = sgl.Runtime(model_path="meta-llama/Llama-2-7b-hf")

# 3. 结构化生成
prompt = "Generate a person's information:"
response = runtime.generate(
    prompt,
    schema=schema,
    max_new_tokens=200,
)

# 4. 解析 JSON
result = json.loads(response.text)
print(result)
\end{lstlisting}

\textbf{函数调用}:

\begin{lstlisting}[language=Python, caption=SGLang 函数调用示例]
import sglang as sgl

# 1. 定义函数
functions = [
    {
        "name": "get_weather",
        "description": "Get the weather for a location",
        "parameters": {
            "type": "object",
            "properties": {
                "location": {"type": "string"},
                "unit": {"type": "string", "enum": ["celsius", "fahrenheit"]}
            }
        }
    }
]

# 2. 创建运行时
runtime = sgl.Runtime(model_path="meta-llama/Llama-2-7b-hf")

# 3. 函数调用生成
prompt = "What's the weather in Beijing?"
response = runtime.generate(
    prompt,
    functions=functions,
    function_call="auto",
    max_new_tokens=200,
)

# 4. 提取函数调用
function_call = response.function_call
print(f"Function: {function_call['name']}")
print(f"Arguments: {function_call['arguments']}")
\end{lstlisting}

\textbf{批量推理(前缀共享)}:

\begin{lstlisting}[language=Python, caption=SGLang 批量推理示例]
import sglang as sgl

# 1. 创建运行时
runtime = sgl.Runtime(model_path="meta-llama/Llama-2-7b-hf")

# 2. 共享前缀的批量请求
# 这些请求共享相同的系统提示
system_prompt = "You are a helpful assistant."
user_queries = [
    "What is AI?",
    "Explain machine learning",
    "What is deep learning?",
]

# 3. 批量生成(自动前缀共享)
responses = runtime.generate_batch(
    [system_prompt + "\n\n" + query for query in user_queries],
    max_new_tokens=100,
    temperature=0.7,
)

# 4. 处理响应
for i, response in enumerate(responses):
    print(f"Query: {user_queries[i]}")
    print(f"Response: {response.text}\n")
\end{lstlisting}

\textbf{代码解释}:
\begin{itemize}
    \item \textbf{基础推理}:简洁的 API,类似 vLLM
    \item \textbf{结构化生成}:原生支持 JSON Schema,自动验证格式
    \item \textbf{函数调用}:原生支持函数调用,自动解析参数
    \item \textbf{批量推理}:RadixAttention 自动共享公共前缀,大幅提升性能
    \item \textbf{优势}:结构化生成能力强,复杂提示性能优秀
    \item \textbf{劣势}:相对较新,文档和示例较少
\end{itemize}

\subsection{性能对比代码示例}

\textbf{吞吐量测试}:

\begin{lstlisting}[language=Python, caption=性能对比测试示例]
import time
import torch
from vllm import LLM, SamplingParams
import sglang as sgl

# 测试数据
prompts = ["The future of AI is"] * 100

# 1. PyTorch 测试
def test_pytorch():
    model = GPT2LMHeadModel.from_pretrained('gpt2')
    model.eval()
    
    start = time.time()
    for prompt in prompts:
        inputs = tokenizer(prompt, return_tensors='pt')
        with torch.no_grad():
            outputs = model.generate(inputs.input_ids, max_length=50)
    pytorch_time = time.time() - start
    
    return pytorch_time

# 2. vLLM 测试
def test_vllm():
    llm = LLM(model="meta-llama/Llama-2-7b-hf")
    sampling_params = SamplingParams(max_tokens=50)
    
    start = time.time()
    llm.generate(prompts, sampling_params)
    vllm_time = time.time() - start
    
    return vllm_time

# 3. SGLang 测试
def test_sglang():
    runtime = sgl.Runtime(model_path="meta-llama/Llama-2-7b-hf")
    
    start = time.time()
    runtime.generate_batch(prompts, max_new_tokens=50)
    sglang_time = time.time() - start
    
    return sglang_time

# 运行测试
pytorch_time = test_pytorch()
vllm_time = test_vllm()
sglang_time = test_sglang()

print(f"PyTorch: {pytorch_time:.2f}s")
print(f"vLLM: {vllm_time:.2f}s ({pytorch_time/vllm_time:.1f}x faster)")
print(f"SGLang: {sglang_time:.2f}s ({pytorch_time/sglang_time:.1f}x faster)")
\end{lstlisting}

\subsection{Flash Attention}

\textbf{概念解释}:Flash Attention 通过分块计算和在线 softmax 优化注意力机制,减少内存占用。

\textbf{数学原理}:

标准 Attention:
\begin{equation}
\mathbf{O} = \text{softmax}\left(\frac{\mathbf{Q}\mathbf{K}^T}{\sqrt{d_k}}\right)\mathbf{V}
\end{equation}

Flash Attention 分块计算:
\begin{align}
\mathbf{O}_i &= \sum_{j=1}^{N} \frac{\exp(\mathbf{s}_{ij} - m_i)}{\sum_{k=1}^{N} \exp(\mathbf{s}_{ik} - m_i)} \mathbf{V}_j \\
m_i &= \max_j \mathbf{s}_{ij}, \quad \mathbf{s}_{ij} = \frac{\mathbf{Q}_i \mathbf{K}_j^T}{\sqrt{d_k}}
\end{align}

\textbf{内存优化}:
\begin{itemize}
    \item 标准 Attention:$O(N^2)$ 内存
    \item Flash Attention:$O(N)$ 内存
\end{itemize}

\begin{lstlisting}[caption=Flash Attention 使用示例]
import torch
from flash_attn import flash_attn_func

# 输入
q = torch.randn(32, 128, 8, 64, dtype=torch.float16, device="cuda")
k = torch.randn(32, 128, 8, 64, dtype=torch.float16, device="cuda")
v = torch.randn(32, 128, 8, 64, dtype=torch.float16, device="cuda")

# Flash Attention
output = flash_attn_func(q, k, v, dropout_p=0.0, softmax_scale=1.0/sqrt(64))

print(f"输出形状: {output.shape}")  # (32, 128, 8, 64)
\end{lstlisting}

\subsection{量化推理}

\textbf{概念解释}:量化推理通过降低模型精度减少内存占用和加速推理。

\textbf{INT8 量化}:

\begin{equation}
Q(x) = \text{round}\left(\frac{x}{\text{scale}}\right) \times \text{scale}
\end{equation}

\textbf{GPTQ 量化}:

GPTQ(GPT Quantization)是一种后训练量化方法:

\begin{equation}
\arg\min_{\hat{\mathbf{W}}} ||\mathbf{W}\mathbf{X} - \hat{\mathbf{W}}\mathbf{X}||_2^2
\end{equation}

\begin{lstlisting}[caption=GPTQ 量化示例]
from transformers import AutoModelForCausalLM, AutoTokenizer
from auto_gptq import AutoGPTQForCausalLM

# 加载模型
model_name = "meta-llama/Llama-2-7b-hf"

# GPTQ 量化
model = AutoGPTQForCausalLM.from_quantized(
    model_name,
    use_safetensors=True,
    device="cuda:0"
)

tokenizer = AutoTokenizer.from_pretrained(model_name)

# 推理
inputs = tokenizer("Hello, how are you?", return_tensors="pt").to("cuda")
outputs = model.generate(**inputs, max_length=50)
print(tokenizer.decode(outputs[0]))
\end{lstlisting}

\subsection{模型并行与张量并行}

\textbf{概念解释}:当模型太大无法放入单卡时,需要将模型分布到多个 GPU。

\textbf{张量并行}:将矩阵乘法分块:

\begin{equation}
\mathbf{Y} = \mathbf{X}\mathbf{W} = \mathbf{X}[\mathbf{W}_1 | \mathbf{W}_2] = [\mathbf{X}\mathbf{W}_1 | \mathbf{X}\mathbf{W}_2]
\end{equation}

\textbf{流水线并行}:将模型按层分割到不同 GPU。

\section{模型部署技术}

\subsection{模型压缩}

\textbf{知识蒸馏}:用大模型(教师)指导小模型(学生)学习。

\textbf{数学公式}:
\begin{equation}
\mathcal{L} = \alpha \mathcal{L}_{CE}(\mathbf{y}, \mathbf{p}_s) + (1-\alpha) \mathcal{L}_{KL}(\mathbf{p}_t, \mathbf{p}_s)
\end{equation}

其中 $\mathbf{p}_t$ 是教师模型的输出,$\mathbf{p}_s$ 是学生模型的输出。

\subsection{服务化部署}

\textbf{API 服务}:使用 FastAPI、Flask 等框架部署模型服务。

\begin{lstlisting}[caption=模型服务化部署示例]
from fastapi import FastAPI
from transformers import pipeline
import torch

app = FastAPI()

# 加载模型
model = pipeline(
    "text-generation",
    model="meta-llama/Llama-2-7b-hf",
    device=0 if torch.cuda.is_available() else -1
)

@app.post("/generate")
async def generate_text(prompt: str, max_length: int = 100):
    result = model(prompt, max_length=max_length)
    return {"generated_text": result[0]["generated_text"]}

# 运行: uvicorn app:app --host 0.0.0.0 --port 8000
\end{lstlisting}

\subsection{边缘部署}

\textbf{模型量化}:INT8/INT4 量化减少模型大小。

\textbf{模型剪枝}:移除不重要的权重。

\textbf{专用硬件}:使用 NPU、TPU 等专用芯片加速。

\section{推理优化技术}

\subsection{KV Cache}

\textbf{概念解释}:KV Cache 缓存之前计算的 Key 和 Value,避免重复计算。

\textbf{优化效果}:
\begin{itemize}
    \item 减少计算量:从 $O(n^2)$ 到 $O(n)$
    \item 提升速度:2-10 倍加速
\end{itemize}

\subsection{连续批处理(Continuous Batching)}

\textbf{概念解释}:动态管理批次,新请求可以立即加入,完成的请求可以立即释放。

\textbf{优势}:
\begin{itemize}
    \item 提高 GPU 利用率
    \item 降低延迟
    \item 支持动态负载
\end{itemize}

\subsection{动态批处理}

\textbf{概念解释}:根据请求长度动态调整批次大小。

\textbf{策略}:
\begin{itemize}
    \item 短请求优先
    \item 长度相似请求分组
    \item 最大批次大小限制
\end{itemize}

\section{检索增强生成(RAG)技术详解}

检索增强生成(Retrieval-Augmented Generation, RAG)是大语言模型应用中的核心技术,通过结合检索和生成,显著提升模型在知识密集型任务上的表现。

\subsection{RAG 核心原理}

\textbf{概念解释}:RAG 将信息检索与文本生成相结合,在生成答案前先从外部知识库检索相关信息,然后将检索到的信息作为上下文输入给生成模型。

\textbf{数学表示}:

RAG 的生成过程可以表示为:
\begin{equation}
P(y|x) = \sum_{z \in \text{Top-K}(x)} P(z|x) \cdot P(y|x, z)
\end{equation}

其中:
\begin{itemize}
    \item $x$ 是用户查询
    \item $z$ 是从知识库检索到的文档
    \item $\text{Top-K}(x)$ 表示检索到的 Top-K 相关文档
    \item $P(z|x)$ 是检索模型给出的文档相关性分数
    \item $P(y|x, z)$ 是生成模型基于查询和检索文档生成答案的概率
\end{itemize}

\textbf{RAG 的优势}:
\begin{itemize}
    \item \textbf{知识更新}:无需重新训练模型即可更新知识
    \item \textbf{可解释性}:可以追溯到生成答案的来源文档
    \item \textbf{减少幻觉}:基于检索到的真实文档生成,降低编造信息的风险
    \item \textbf{领域适应}:可以快速适应特定领域的知识库
\end{itemize}

\subsection{RAG 系统架构}

\textbf{两阶段检索架构}:

\begin{enumerate}
    \item \textbf{召回层(Recall Layer)}:
    \begin{itemize}
        \item 使用快速检索算法(如 BM25、KNN)从大规模知识库中召回候选文档
        \item 目标:高召回率,不遗漏相关文档
        \item 常用方法:BM25(关键词匹配)、向量检索(语义相似度)
    \end{itemize}
    
    \item \textbf{排序层(Ranking Layer)}:
    \begin{itemize}
        \item 使用精细排序模型(如 Cross-Encoder)对召回文档进行重排序
        \item 目标:高精确度,确保最相关的文档排在前面
        \item 常用方法:Cross-Encoder、LambdaMART
    \end{itemize}
\end{enumerate}

\textbf{混合检索策略}:

结合关键词检索和语义检索:
\begin{equation}
\text{Score}(q, d) = \alpha \cdot \text{BM25}(q, d) + (1-\alpha) \cdot \text{Sim}(E(q), E(d))
\end{equation}

其中:
\begin{itemize}
    \item $\text{BM25}(q, d)$ 是 BM25 关键词匹配分数
    \item $\text{Sim}(E(q), E(d))$ 是查询和文档嵌入向量的相似度
    \item $\alpha$ 是混合权重,通常设置为 0.3-0.7
\end{itemize}

\subsection{查询重写(Query Rewrite)}

\textbf{概念解释}:用户查询往往简短且不完整,查询重写模型将简短查询扩展为完整的语义查询,提高检索准确性。

\textbf{技术实现}:

使用 T5 等序列到序列模型进行查询重写:
\begin{equation}
q' = \text{T5}(q, \text{context})
\end{equation}

\textbf{应用场景}:
\begin{itemize}
    \item \textbf{查询扩展}:将"退改政策"扩展为"酒店退改政策、退改时间限制、退改费用"
    \item \textbf{同义词替换}:将"取消"替换为"退订"、"退款"等
    \item \textbf{意图理解}:理解用户真实意图,如"改签"实际需要查询"变更政策"
\end{itemize}

\textbf{效果提升}:查询重写可以提升检索相关性 30-40\%,显著改善 RAG 系统的整体表现。

\subsection{重排序(Rerank)技术}

\textbf{Cross-Encoder 重排序}:

Cross-Encoder 将查询和文档拼接后输入模型,进行深度交互建模:

\begin{equation}
\text{Score}(q, d) = \text{CrossEncoder}([\text{CLS}] q [\text{SEP}] d)
\end{equation}

\textbf{优势}:
\begin{itemize}
    \item \textbf{深度交互}:查询和文档在模型内部充分交互,理解更准确
    \item \textbf{高精确度}:相比 Bi-Encoder,精确度提升 20-40\%
    \item \textbf{解决误召回}:有效解决"相似政策误召回"问题
\end{itemize}

\textbf{LambdaMART 融合排序}:

LambdaMART 是学习排序(Learning to Rank)算法,用于优化多个排序信号的融合权重:

\begin{equation}
\text{FinalScore} = \sum_{i=1}^{n} w_i \cdot \text{Score}_i(q, d)
\end{equation}

其中 $\text{Score}_i$ 可以是:
\begin{itemize}
    \item BM25 分数
    \item 向量相似度分数
    \item Cross-Encoder 分数
    \item 其他特征(文档长度、时间戳等)
\end{itemize}

\section{对比学习与负样本挖掘}

\subsection{对比学习(Contrastive Learning)}

\textbf{概念解释}:对比学习通过拉近相似样本、推远不相似样本来学习表示,是训练高质量嵌入模型的核心技术。

\textbf{数学原理}:

对比学习的损失函数(InfoNCE):
\begin{equation}
\mathcal{L}_{\text{contrastive}} = -\log \frac{\exp(\text{sim}(q, d^+) / \tau)}{\exp(\text{sim}(q, d^+) / \tau) + \sum_{d^-} \exp(\text{sim}(q, d^-) / \tau)}
\end{equation}

其中:
\begin{itemize}
    \item $q$ 是查询向量
    \item $d^+$ 是正样本(相关文档)
    \item $d^-$ 是负样本(不相关文档)
    \item $\tau$ 是温度参数,控制分布的尖锐程度
    \item $\text{sim}(\cdot, \cdot)$ 是相似度函数(如余弦相似度)
\end{itemize}

\textbf{在嵌入模型微调中的应用}:

对于 RoBERTa 等预训练模型,使用对比学习进行领域适应:
\begin{enumerate}
    \item \textbf{正样本构建}:查询和其对应的相关文档
    \item \textbf{负样本构建}:查询和随机采样的不相关文档
    \item \textbf{训练目标}:最大化正样本相似度,最小化负样本相似度
\end{enumerate}

\textbf{效果}:对比学习可以提升业务意图识别准确率 20-30\%,提升长尾查询匹配率 30-40\%。

\subsection{困难负样本挖掘(Hard Negative Mining)}

\textbf{概念解释}:困难负样本是与查询相似但不相关的样本,挖掘这些样本可以显著提升模型的学习效果。

\textbf{困难负样本识别}:

\begin{equation}
d_{\text{hard}} = \arg\max_{d \in \mathcal{D}_{\text{neg}}} \text{sim}(E(q), E(d))
\end{equation}

即选择与查询最相似但不相关的文档作为困难负样本。

\textbf{挖掘策略}:

\begin{enumerate}
    \item \textbf{在线挖掘}:在训练过程中动态选择困难负样本
    \item \textbf{离线挖掘}:使用当前模型在数据集中挖掘困难负样本,然后用于下一轮训练
    \item \textbf{混合策略}:结合随机负样本和困难负样本
\end{enumerate}

\textbf{效果提升}:困难负样本挖掘可以提升召回率 20-30\%,使模型更好地区分相似但不相关的文档。

\section{知识蒸馏(Knowledge Distillation)}

\subsection{知识蒸馏原理}

\textbf{概念解释}:知识蒸馏是一种模型压缩技术,通过让小型学生模型学习大型教师模型的知识,在保持性能的同时大幅降低模型大小和推理成本。

\textbf{数学表示}:

知识蒸馏的损失函数:
\begin{equation}
\mathcal{L}_{\text{KD}} = \alpha \cdot \mathcal{L}_{\text{CE}}(y, \text{softmax}(z_s)) + (1-\alpha) \cdot \mathcal{L}_{\text{KL}}(\text{softmax}(z_t/T), \text{softmax}(z_s/T))
\end{equation}

其中:
\begin{itemize}
    \item $z_t$ 是教师模型的 logits
    \item $z_s$ 是学生模型的 logits
    \item $T$ 是温度参数(通常 $T=3-5$),温度越高,分布越平滑
    \item $\alpha$ 是平衡系数(通常 $\alpha=0.3-0.5$)
    \item $\mathcal{L}_{\text{CE}}$ 是交叉熵损失
    \item $\mathcal{L}_{\text{KL}}$ 是 KL 散度损失
\end{itemize}

\textbf{温度参数的作用}:

温度参数 $T$ 用于软化概率分布:
\begin{equation}
p_i = \frac{\exp(z_i / T)}{\sum_j \exp(z_j / T)}
\end{equation}

温度越高,分布越平滑,学生模型可以学习到教师模型的"软标签"中包含的丰富信息。

\subsection{在 RAG 系统中的应用}

\textbf{轻量级模型部署}:

\begin{itemize}
    \item \textbf{高频查询处理}:将 50\% 的高频简单查询切换到本地轻量级模型
    \item \textbf{成本降低}:减少 API 调用成本 80-85\%
    \item \textbf{延迟降低}:本地推理延迟更低,用户体验更好
\end{itemize}

\textbf{蒸馏策略}:

\begin{enumerate}
    \item \textbf{任务特定蒸馏}:针对特定任务(如意图分类)训练小型模型
    \item \textbf{多任务蒸馏}:将多个任务的知识蒸馏到单一小型模型
    \item \textbf{渐进式蒸馏}:逐步减小模型大小,保持性能
\end{enumerate}

\section{大语言模型中的强化学习}

\subsection{大模型 RL 概述}

\textbf{严谨解释}:

大语言模型中的强化学习(RL for LLM)是一种通过与环境交互来优化语言模型生成策略的方法。在数学上,这可以形式化为一个部分可观测马尔可夫决策过程(POMDP):

\begin{equation}
\mathcal{M} = (\mathcal{S}, \mathcal{A}, \mathcal{P}, \mathcal{R}, \gamma)
\end{equation}

其中:
\begin{itemize}
    \item $\mathcal{S}$ 是状态空间,表示当前对话历史和上下文
    \item $\mathcal{A}$ 是动作空间,表示所有可能的 token 序列
    \item $\mathcal{P}: \mathcal{S} \times \mathcal{A} \rightarrow \mathcal{S}$ 是状态转移概率,由语言模型决定
    \item $\mathcal{R}: \mathcal{S} \times \mathcal{A} \rightarrow \mathbb{R}$ 是奖励函数,评估生成质量
    \item $\gamma \in [0, 1]$ 是折扣因子
\end{itemize}

目标是最优策略 $\pi^*$,最大化期望累积奖励:

\begin{equation}
\pi^* = \arg\max_{\pi} \mathbb{E}_{\tau \sim \pi} \left[ \sum_{t=0}^{T} \gamma^t r_t \right]
\end{equation}

\textbf{通俗解释}:

想象你在训练一个AI助手,它需要学会说"好话"(符合人类偏好)。传统方法是给它看很多例子让它模仿,但强化学习就像给它一个"评分系统":
\begin{itemize}
    \item AI 生成一段话
    \item 人类(或奖励模型)给它打分
    \item AI 根据分数调整策略,下次生成更好的话
    \item 反复这个过程,AI 越来越会说"好话"
\end{itemize}

这就像训练宠物:做对了给奖励,做错了不给奖励,久而久之它就知道什么行为是好的。

\subsection{大模型 RL 的应用场景}

\textbf{1. 人类反馈强化学习(RLHF)}

\textbf{严谨解释}:RLHF 通过人类标注的偏好数据训练奖励模型,然后使用强化学习优化生成策略,使模型输出符合人类价值观和偏好。

\textbf{通俗解释}:让AI学会"投其所好"。通过人类打分,AI学会生成人类喜欢的内容,比如更安全、更有帮助、更符合事实的回答。

\textbf{应用场景}:
\begin{itemize}
    \item ChatGPT、Claude 等对话模型的训练
    \item 内容安全对齐,减少有害内容生成
    \item 风格对齐,使输出符合特定风格
\end{itemize}

\textbf{2. 检索策略优化}

\textbf{严谨解释}:在 RAG 系统中,使用强化学习优化检索策略,使检索到的文档能够生成更高质量的回复。

\textbf{通俗解释}:教AI"找资料"的技巧。通过强化学习,AI学会从知识库中检索最相关的文档,从而生成更好的答案。

\textbf{应用场景}:
\begin{itemize}
    \item 智能客服系统
    \item 知识问答系统
    \item 文档摘要生成
\end{itemize}

\textbf{3. 代码生成优化}

\textbf{严谨解释}:使用强化学习优化代码生成策略,使生成的代码更符合编程规范、更易读、更高效。

\textbf{通俗解释}:教AI写"好代码"。通过强化学习,AI学会生成不仅功能正确,而且风格优雅、效率高的代码。

\textbf{应用场景}:
\begin{itemize}
    \item GitHub Copilot 等代码助手
    \item 代码重构工具
    \item 代码审查助手
\end{itemize}

\subsection{大模型 RL 与传统 RL 的对比}

\begin{table}[h]
\centering
\begin{tabular}{lcc}
\toprule
\textbf{特性} & \textbf{传统 RL} & \textbf{大模型 RL} \\
\midrule
状态空间 & 离散/连续(如游戏状态) & 高维离散(token 序列) \\
\midrule
动作空间 & 有限动作集合 & 巨大动作空间(词汇表大小) \\
\midrule
奖励信号 & 环境直接给出 & 需要人类或奖励模型 \\
\midrule
样本效率 & 通常较低 & 极低(需要大量交互) \\
\midrule
探索策略 & $\epsilon$-贪婪、UCB 等 & 采样策略、温度参数 \\
\midrule
训练方式 & 在线学习 & 离线+在线混合 \\
\midrule
主要挑战 & 探索-利用平衡 & 奖励设计、稳定性 \\
\bottomrule
\end{tabular}
\caption{传统 RL 与大模型 RL 对比}
\end{table}

\textbf{关键区别详解}:

\textbf{1. 动作空间规模}

\textbf{严谨解释}:
\begin{itemize}
    \item \textbf{传统 RL}:动作空间通常是有限的,如 Atari 游戏有 4-18 个动作
    \item \textbf{大模型 RL}:动作空间是词汇表大小,通常 $|\mathcal{V}| = 30,000-100,000$
\end{itemize}

\textbf{通俗解释}:
\begin{itemize}
    \item \textbf{传统 RL}:就像在一个有10个按钮的控制面板上操作
    \item \textbf{大模型 RL}:就像在一个有5万个按钮的控制面板上操作,每个按钮代表一个词
\end{itemize}

\textbf{2. 奖励信号}

\textbf{严谨解释}:
\begin{itemize}
    \item \textbf{传统 RL}:环境直接提供奖励,如游戏得分、到达目标
    \item \textbf{大模型 RL}:需要人类标注或训练奖励模型 $r_\phi(x, y)$ 来评估生成质量
\end{itemize}

\textbf{通俗解释}:
\begin{itemize}
    \item \textbf{传统 RL}:游戏自动告诉你得分,很明确
    \item \textbf{大模型 RL}:需要人工判断"这段话好不好",主观性强
\end{itemize}

\textbf{3. 训练稳定性}

\textbf{严谨解释}:
\begin{itemize}
    \item \textbf{传统 RL}:策略更新通常较稳定,因为动作空间小
    \item \textbf{大模型 RL}:策略更新容易导致模型"崩溃"(生成无意义文本),需要 KL 散度约束
\end{itemize}

\textbf{通俗解释}:
\begin{itemize}
    \item \textbf{传统 RL}:调整策略后,机器人可能走偏一点,但还能走
    \item \textbf{大模型 RL}:调整策略后,AI可能突然"胡言乱语",需要小心约束
\end{itemize}

\textbf{4. 样本效率}

\textbf{严谨解释}:
\begin{itemize}
    \item \textbf{传统 RL}:可能需要 $10^6-10^8$ 个样本
    \item \textbf{大模型 RL}:可能需要 $10^9-10^{12}$ 个样本,但通过预训练可以大幅减少
\end{itemize}

\textbf{通俗解释}:
\begin{itemize}
    \item \textbf{传统 RL}:需要大量试错,但相对可控
    \item \textbf{大模型 RL}:需要海量数据,但预训练模型提供了很好的起点
\end{itemize}

\section{PPO、GRPO、DPO 详解}

\subsection{PPO(Proximal Policy Optimization)}

\textbf{严谨解释}:

PPO 是一种策略梯度算法,通过限制策略更新的幅度来保证训练稳定性。PPO 的目标函数为:

\begin{equation}
\mathcal{L}_{\text{PPO}}(\theta) = \mathbb{E}_{(s_t, a_t) \sim \pi_{\theta_{\text{old}}}} \left[ \min \left( r_t(\theta) \hat{A}_t, \text{clip}(r_t(\theta), 1-\epsilon, 1+\epsilon) \hat{A}_t \right) \right]
\end{equation}

其中:
\begin{itemize}
    \item $r_t(\theta) = \frac{\pi_\theta(a_t|s_t)}{\pi_{\theta_{\text{old}}}(a_t|s_t)}$ 是重要性采样比率
    \item $\hat{A}_t$ 是优势函数估计(通常使用 GAE)
    \item $\epsilon$ 是裁剪参数(通常 $\epsilon = 0.1-0.2$)
    \item $\text{clip}(x, a, b) = \max(a, \min(x, b))$ 是裁剪函数
\end{itemize}

\textbf{通俗解释}:

PPO 就像给AI一个"安全绳":
\begin{itemize}
    \item 传统策略梯度:AI可能"大步前进",容易"摔倒"(策略崩溃)
    \item PPO:限制AI每次只能"小步前进",即使走错也不会太远
    \item 裁剪机制:如果AI想"走太远",就把它"拉回来"
\end{itemize}

这就像学骑自行车时,有人在后面扶着,防止你摔倒。

\textbf{在 RLHF 中的应用}:

\begin{enumerate}
    \item \textbf{训练奖励模型}:使用人类偏好数据训练 $r_\phi(x, y)$
    \item \textbf{优化生成策略}:使用 PPO 优化 $\pi_\theta$,最大化奖励
    \item \textbf{KL 散度约束}:防止策略偏离参考模型太远
\end{enumerate}

\textbf{PPO 的优势}:
\begin{itemize}
    \item \textbf{稳定性}:通过裁剪保证训练稳定
    \item \textbf{效率}:可以多次使用同一批数据
    \item \textbf{简单}:实现相对简单,超参数少
\end{itemize}

\textbf{PPO 的劣势}:
\begin{itemize}
    \item \textbf{需要奖励模型}:必须先训练奖励模型
    \item \textbf{样本效率}:相比 DPO,样本效率较低
    \item \textbf{超参数敏感}:$\epsilon$ 的选择影响性能
\end{itemize}

\subsection{GRPO(Group Relative Policy Optimization)}

\textbf{严谨解释}:

GRPO 是一种无需奖励模型的策略优化方法,通过组内相对比较来优化策略。GRPO 的损失函数为:

\begin{equation}
\mathcal{L}_{\text{GRPO}}(\theta) = -\mathbb{E}_{(x, \{y_i\}_{i=1}^n) \sim \mathcal{D}} \left[ \log \frac{\exp(\beta \log \pi_\theta(y_{\text{best}}|x))}{\sum_{i=1}^n \exp(\beta \log \pi_\theta(y_i|x))} \right]
\end{equation}

其中:
\begin{itemize}
    \item $\{y_i\}_{i=1}^n$ 是一组候选回复
    \item $y_{\text{best}}$ 是组内最优回复(由人类标注)
    \item $\beta$ 是温度参数
\end{itemize}

\textbf{通俗解释}:

GRPO 就像"选美比赛":
\begin{itemize}
    \item 给AI看一组候选回复(比如5个)
    \item 人类选出最好的一个
    \item AI学习"为什么这个最好",调整策略生成类似的回复
\end{itemize}

这比"打分"更简单:不需要精确的分数,只需要知道"哪个更好"。

\textbf{GRPO 的优势}:
\begin{itemize}
    \item \textbf{无需奖励模型}:直接使用人类偏好
    \item \textbf{相对简单}:实现比 DPO 更简单
    \item \textbf{组内比较}:可以同时比较多个候选
\end{itemize}

\textbf{GRPO 的劣势}:
\begin{itemize}
    \item \textbf{需要多个候选}:每个样本需要生成多个候选回复
    \item \textbf{计算成本}:生成多个候选增加计算量
    \item \textbf{性能}:通常不如 DPO
\end{itemize}

\subsection{DPO(Direct Preference Optimization)}

\textbf{严谨解释}:

DPO 是一种无需奖励模型的偏好对齐方法,通过直接优化策略来符合人类偏好。DPO 的核心思想是将奖励模型隐式地嵌入到策略优化中。

\textbf{DPO 的损失函数}:

\begin{equation}
\mathcal{L}_{\text{DPO}}(\theta) = -\mathbb{E}_{(x, y_w, y_l) \sim \mathcal{D}} \left[ \log \sigma \left( \beta \log \frac{\pi_\theta(y_w|x)}{\pi_{\text{ref}}(y_w|x)} - \beta \log \frac{\pi_\theta(y_l|x)}{\pi_{\text{ref}}(y_l|x)} \right) \right]
\end{equation}

其中:
\begin{itemize}
    \item $(x, y_w, y_l)$ 是偏好数据,$y_w$ 是偏好回复,$y_l$ 是非偏好回复
    \item $\pi_\theta$ 是待优化的策略
    \item $\pi_{\text{ref}}$ 是参考策略(通常是 SFT 后的模型)
    \item $\beta$ 是温度参数,控制优化强度
    \item $\sigma$ 是 sigmoid 函数
\end{itemize}

\textbf{为什么 DPO 无需奖励模型?数学解释}:

\textbf{关键洞察}:DPO 通过数学变换,将奖励模型的优化问题转化为策略优化问题。

\textbf{步骤 1:奖励模型的最优形式}

在 RLHF 中,给定奖励模型 $r_\phi(x, y)$,最优策略为:

\begin{equation}
\pi^*(y|x) = \frac{1}{Z(x)} \pi_{\text{ref}}(y|x) \exp\left(\frac{1}{\beta} r_\phi(x, y)\right)
\end{equation}

其中 $Z(x) = \sum_y \pi_{\text{ref}}(y|x) \exp(\frac{1}{\beta} r_\phi(x, y))$ 是归一化常数。

\textbf{步骤 2:奖励函数的隐式表示}

从最优策略可以反推出奖励函数:

\begin{equation}
r_\phi(x, y) = \beta \log \frac{\pi^*(y|x)}{\pi_{\text{ref}}(y|x)} + \beta \log Z(x)
\end{equation}

注意:$Z(x)$ 只依赖于 $x$,在比较两个回复 $y_w$ 和 $y_l$ 时会抵消。

\textbf{步骤 3:偏好概率建模}

使用 Bradley-Terry 模型建模人类偏好:

\begin{equation}
P(y_w \succ y_l | x) = \frac{\exp(r_\phi(x, y_w))}{\exp(r_\phi(x, y_w)) + \exp(r_\phi(x, y_l))} = \sigma(r_\phi(x, y_w) - r_\phi(x, y_l))
\end{equation}

\textbf{步骤 4:替换奖励函数}

将步骤 2 的奖励函数代入步骤 3:

\begin{align}
P(y_w \succ y_l | x) &= \sigma\left( \beta \log \frac{\pi^*(y_w|x)}{\pi_{\text{ref}}(y_w|x)} - \beta \log \frac{\pi^*(y_l|x)}{\pi_{\text{ref}}(y_l|x)} \right) \\
&= \sigma\left( \beta \log \frac{\pi^*(y_w|x)}{\pi_{\text{ref}}(y_w|x)} - \beta \log \frac{\pi^*(y_l|x)}{\pi_{\text{ref}}(y_l|x)} \right)
\end{align}

\textbf{步骤 5:直接优化策略}

最大化对数似然:

\begin{equation}
\max_\theta \mathbb{E}_{(x, y_w, y_l)} \left[ \log \sigma \left( \beta \log \frac{\pi_\theta(y_w|x)}{\pi_{\text{ref}}(y_w|x)} - \beta \log \frac{\pi_\theta(y_l|x)}{\pi_{\text{ref}}(y_l|x)} \right) \right]
\end{equation}

这就是 DPO 的损失函数(取负号后最小化)。

\textbf{关键点}:
\begin{itemize}
    \item \textbf{奖励模型被隐式表示}:通过策略比率 $\frac{\pi_\theta(y|x)}{\pi_{\text{ref}}(y|x)}$ 隐式表示奖励
    \item \textbf{归一化常数抵消}:$Z(x)$ 在比较时抵消,不需要显式计算
    \item \textbf{直接优化策略}:不需要先训练奖励模型,直接优化策略即可
\end{itemize}

\textbf{通俗解释}:

传统 RLHF 需要两步:
\begin{enumerate}
    \item 训练奖励模型:教AI"什么是好,什么是坏"(打分系统)
    \item 优化策略:让AI生成高分内容
\end{enumerate}

DPO 只需要一步:
\begin{enumerate}
    \item 直接优化策略:让AI生成"人类更喜欢"的内容
\end{enumerate}

\textbf{为什么可以这样?}

想象你在学做菜:
\begin{itemize}
    \item \textbf{传统方法}:先学"评分标准"(什么菜好吃),再学"怎么做高分菜"
    \item \textbf{DPO方法}:直接学"怎么做人类更喜欢的菜",不需要明确的评分标准
\end{itemize}

DPO 的"魔法"在于:通过比较两个回复,AI自动学会了"什么是好",不需要明确的分数。

\textbf{DPO 的优势}:
\begin{itemize}
    \item \textbf{无需奖励模型}:减少训练步骤和计算成本
    \item \textbf{训练稳定}:相比 RLHF,训练更稳定
    \item \textbf{计算高效}:只需要一次前向传播
    \item \textbf{效果优秀}:在风格对齐任务上效果显著
\end{itemize}

\textbf{DPO 的劣势}:
\begin{itemize}
    \item \textbf{需要参考模型}:需要 SFT 后的参考模型
    \item \textbf{偏好数据质量}:对偏好数据质量要求高
    \item \textbf{难以处理复杂奖励}:对于需要多维度评估的任务,不如显式奖励模型灵活
\end{itemize}

\subsection{PPO、GRPO、DPO 对比}

\begin{table}[h]
\centering
\begin{tabular}{lccc}
\toprule
\textbf{特性} & \textbf{PPO} & \textbf{GRPO} & \textbf{DPO} \\
\midrule
需要奖励模型 & 是 & 否 & 否 \\
\midrule
训练步骤 & 2步(奖励模型+策略优化) & 1步 & 1步 \\
\midrule
样本效率 & 中等 & 较低 & 较高 \\
\midrule
训练稳定性 & 中等 & 较高 & 高 \\
\midrule
计算成本 & 高 & 中等 & 低 \\
\midrule
适用场景 & 复杂奖励、多维度评估 & 组内比较 & 风格对齐、偏好学习 \\
\bottomrule
\end{tabular}
\caption{PPO、GRPO、DPO 对比}
\end{table}

\section{强化学习在 RAG 中的应用}

\subsection{REINFORCE 算法优化检索策略}

\textbf{概念解释}:REINFORCE 是一种策略梯度算法,用于优化 RAG 系统中的检索策略,使生成的回复更符合业务标准。

\textbf{数学原理}:

REINFORCE 的策略梯度:
\begin{equation}
\nabla_\theta J(\theta) = \mathbb{E}_{\tau \sim \pi_\theta} \left[ \sum_{t=0}^{T} \nabla_\theta \log \pi_\theta(a_t|s_t) \cdot R(\tau) \right]
\end{equation}

其中:
\begin{itemize}
    \item $\pi_\theta(a_t|s_t)$ 是在状态 $s_t$ 下选择动作 $a_t$ 的策略
    \item $R(\tau)$ 是轨迹 $\tau$ 的累积奖励
    \item $\theta$ 是策略参数
\end{itemize}

\textbf{在 RAG 中的应用}:

\begin{enumerate}
    \item \textbf{状态}:当前查询和已检索的文档
    \item \textbf{动作}:选择检索哪些文档
    \item \textbf{奖励}:生成回复的质量(如 ROUGE-L 分数、人工评估分数)
    \item \textbf{目标}:最大化生成回复的质量
\end{enumerate}

\textbf{效果}:REINFORCE 优化可以提升 ROUGE-L 分数 25-30\%,使回复更符合客服黄金标准。


\section{噪声嵌入训练(NEFTune)}

\subsection{NEFTune 原理}

\textbf{概念解释}:NEFTune(Noise Embeddings for Fine-Tuning)通过在嵌入层添加噪声来提升模型在小样本上的泛化能力,解决过拟合导致的机械重复问题。

\textbf{数学表示}:

NEFTune 在嵌入层添加噪声:
\begin{equation}
\tilde{\mathbf{e}}_i = \mathbf{e}_i + \mathbf{n}_i, \quad \mathbf{n}_i \sim \mathcal{N}(0, \sigma^2 \mathbf{I})
\end{equation}

其中:
\begin{itemize}
    \item $\mathbf{e}_i$ 是原始嵌入向量
    \item $\tilde{\mathbf{e}}_i$ 是添加噪声后的嵌入向量
    \item $\sigma$ 是噪声强度(通常 $\sigma=0.1-0.2$)
\end{itemize}

\textbf{作用机制}:

\begin{itemize}
    \item \textbf{正则化效果}:噪声起到正则化作用,防止过拟合
    \item \textbf{提升泛化}:增强模型在未见数据上的表现
    \item \textbf{减少重复}:有效解决小样本微调中的机械重复问题
\end{itemize}

\textbf{应用效果}:

在 LoRA 微调中,NEFTune 可以:
\begin{itemize}
    \item 提升生成多样性 15-25\%
    \item 减少重复生成 30-40\%
    \item 提升下游任务性能 5-10\%
\end{itemize}

\section{检索算法详解}

\subsection{BM25 算法}

\textbf{概念解释}:BM25(Best Matching 25)是信息检索中最经典的关键词匹配算法,基于词频和逆文档频率计算文档相关性。

\textbf{数学公式}:

BM25 分数计算:
\begin{equation}
\text{BM25}(q, d) = \sum_{t \in q} \text{IDF}(t) \cdot \frac{f(t, d) \cdot (k_1 + 1)}{f(t, d) + k_1 \cdot (1 - b + b \cdot \frac{|d|}{\text{avgdl}})}
\end{equation}

其中:
\begin{itemize}
    \item $f(t, d)$ 是词项 $t$ 在文档 $d$ 中的词频
    \item $|d|$ 是文档长度
    \item $\text{avgdl}$ 是平均文档长度
    \item $\text{IDF}(t) = \log \frac{N - n(t) + 0.5}{n(t) + 0.5}$,$N$ 是总文档数,$n(t)$ 是包含词项 $t$ 的文档数
    \item $k_1$ 和 $b$ 是超参数(通常 $k_1=1.2$, $b=0.75$)
\end{itemize}

\textbf{优势}:

\begin{itemize}
    \item \textbf{快速}:计算效率高,适合大规模检索
    \item \textbf{有效}:在关键词匹配任务上表现优秀
    \item \textbf{可解释}:分数计算过程透明
\end{itemize}

\subsection{KNN 向量检索}

\textbf{概念解释}:KNN(K-Nearest Neighbors)向量检索通过计算查询向量和文档向量的相似度,找到最相似的 K 个文档。

\textbf{相似度计算}:

常用余弦相似度:
\begin{equation}
\text{sim}(q, d) = \frac{\mathbf{q} \cdot \mathbf{d}}{||\mathbf{q}|| \cdot ||\mathbf{d}||} = \cos(\theta)
\end{equation}

\textbf{加速方法}:

\begin{itemize}
    \item \textbf{HNSW}:分层导航小世界图,近似最近邻搜索
    \item \textbf{IVF}:倒排文件索引,快速检索
    \item \textbf{PQ}:乘积量化,压缩向量
\end{itemize}

\subsection{HNSW 算法}

\textbf{概念解释}:HNSW(Hierarchical Navigable Small World)是一种高效的近似最近邻搜索算法,通过构建多层图结构实现快速检索。

\textbf{算法特点}:

\begin{itemize}
    \item \textbf{多层结构}:构建多个层次的图,上层节点少,下层节点多
    \item \textbf{快速搜索}:从上层开始搜索,逐步向下层细化
    \item \textbf{高精度}:在保持高检索精度的同时,大幅提升检索速度
\end{itemize}

\textbf{复杂度}:

\begin{itemize}
    \item \textbf{搜索复杂度}:$O(\log N)$,其中 $N$ 是文档数量
    \item \textbf{空间复杂度}:$O(N \cdot M)$,其中 $M$ 是平均连接数
\end{itemize}

\subsection{SimHash 和 MinHash}

\textbf{SimHash}:

SimHash 用于快速计算文本相似度,生成固定长度的哈希值:

\begin{equation}
\text{SimHash}(d) = \text{sign}(\sum_{t \in d} w(t) \cdot \mathbf{h}(t))
\end{equation}

其中 $\mathbf{h}(t)$ 是词项 $t$ 的随机哈希向量。

\textbf{应用}:
\begin{itemize}
    \item \textbf{去重}:快速识别相似文档
    \item \textbf{聚类}:基于哈希值进行快速聚类
\end{itemize}

\textbf{MinHash}:

MinHash 用于估计集合的 Jaccard 相似度:

\begin{equation}
\text{Jaccard}(A, B) \approx \frac{1}{k} \sum_{i=1}^{k} \mathbf{1}[\min(A_i) = \min(B_i)]
\end{equation}

\section{聚类算法在 NLP 中的应用}

\subsection{K-Means 聚类}

\textbf{概念解释}:K-Means 是一种经典的聚类算法,将数据点分为 K 个簇,使得簇内距离最小、簇间距离最大。

\textbf{算法流程}:

\begin{enumerate}
    \item 随机初始化 K 个聚类中心
    \item 将每个数据点分配到最近的聚类中心
    \item 更新聚类中心为簇内点的均值
    \item 重复步骤 2-3 直到收敛
\end{enumerate}

\textbf{目标函数}:

\begin{equation}
J = \sum_{i=1}^{K} \sum_{\mathbf{x} \in C_i} ||\mathbf{x} - \boldsymbol{\mu}_i||^2
\end{equation}

其中 $C_i$ 是第 $i$ 个簇,$\boldsymbol{\mu}_i$ 是簇中心。

\textbf{在 NLP 中的应用}:

\begin{itemize}
    \item \textbf{文档聚类}:将相似文档分组
    \item \textbf{主题发现}:发现文档中的主题
    \item \textbf{数据预处理}:在 RAG 系统中对文档进行预处理
\end{itemize}

\subsection{DBSCAN 聚类}

\textbf{概念解释}:DBSCAN(Density-Based Spatial Clustering of Applications with Noise)是一种基于密度的聚类算法,可以发现任意形状的簇并识别噪声点。

\textbf{核心概念}:

\begin{itemize}
    \item \textbf{核心点}:邻域内至少有 $\text{minPts}$ 个点的点
    \item \textbf{边界点}:在核心点的邻域内但不是核心点的点
    \item \textbf{噪声点}:既不是核心点也不是边界点的点
\end{itemize}

\textbf{优势}:

\begin{itemize}
    \item \textbf{无需预设簇数}:自动发现簇的数量
    \item \textbf{发现任意形状}:可以发现非球形的簇
    \item \textbf{识别噪声}:自动识别异常点
\end{itemize}

\section{主成分分析(PCA)}

\subsection{PCA 原理}

\textbf{概念解释}:PCA(Principal Component Analysis)是一种降维技术,通过找到数据的主要变化方向,将高维数据投影到低维空间。

\textbf{数学原理}:

PCA 的目标是找到投影方向 $\mathbf{w}$,使得投影后的方差最大:

\begin{equation}
\max_{\mathbf{w}} \mathbf{w}^T \mathbf{C} \mathbf{w}, \quad \text{s.t.} \quad ||\mathbf{w}|| = 1
\end{equation}

其中 $\mathbf{C}$ 是协方差矩阵。

\textbf{求解方法}:

通过特征值分解:
\begin{equation}
\mathbf{C} = \mathbf{U} \boldsymbol{\Lambda} \mathbf{U}^T
\end{equation}

主成分就是协方差矩阵的特征向量,按特征值大小排序。

\textbf{在 NLP 中的应用}:

\begin{itemize}
    \item \textbf{特征降维}:降低词向量维度
    \item \textbf{可视化}:将高维数据可视化到 2D/3D
    \item \textbf{噪声去除}:去除数据中的噪声成分
\end{itemize}

\section{流行大语言模型介绍}

\subsection{Qwen 系列}

\textbf{Qwen2.5-1.5B}:

\begin{itemize}
    \item \textbf{发布机构}:阿里巴巴通义千问团队
    \item \textbf{参数量}:1.5B(15 亿参数)
    \item \textbf{特点}:
    \begin{itemize}
        \item 轻量级模型,适合本地部署
        \item 支持多语言(中文、英文等)
        \item 开源可商用
        \item 在对话、代码生成等任务上表现优秀
    \end{itemize}
    \item \textbf{应用场景}:
    \begin{itemize}
        \item 个人数字分身
        \item 边缘设备部署
        \item 低成本微调实验
    \end{itemize}
\end{itemize}

\textbf{Qwen2.5 系列其他模型}:

\begin{itemize}
    \item \textbf{Qwen2.5-7B}:中等规模,平衡性能和效率
    \item \textbf{Qwen2.5-14B}:大规模模型,性能更强
    \item \textbf{Qwen2.5-72B}:超大规模模型,接近 GPT-4 性能
\end{itemize}

\subsection{LLaMA 系列}

\textbf{LLaMA 3}:

\begin{itemize}
    \item \textbf{发布机构}:Meta(Facebook)
    \item \textbf{参数量}:8B, 70B, 405B
    \item \textbf{特点}:
    \begin{itemize}
        \item 开源可商用
        \item 训练数据质量高
        \item 指令遵循能力强
        \item 代码能力优秀
    \end{itemize}
\end{itemize}

\textbf{LLaMA 2}:

\begin{itemize}
    \item \textbf{特点}:
    \begin{itemize}
        \item 首个开源可商用的大语言模型
        \item 支持对话和代码生成
        \item 安全性对齐
    \end{itemize}
\end{itemize}

\subsection{GPT 系列}

\textbf{GPT-4}:

\begin{itemize}
    \item \textbf{发布机构}:OpenAI
    \item \textbf{特点}:
    \begin{itemize}
        \item 多模态能力(文本、图像)
        \item 强大的推理能力
        \item 代码生成能力优秀
        \item 闭源,通过 API 使用
    \end{itemize}
\end{itemize}

\textbf{GPT-3.5}:

\begin{itemize}
    \item \textbf{特点}:
    \begin{itemize}
        \item 175B 参数
        \item 强大的少样本学习能力
        \item 广泛的应用生态
    \end{itemize}
\end{itemize}

\subsection{Claude 系列}

\textbf{Claude 3}:

\begin{itemize}
    \item \textbf{发布机构}:Anthropic
    \item \textbf{模型}:Opus, Sonnet, Haiku
    \item \textbf{特点}:
    \begin{itemize}
        \item 安全性强
        \item 长上下文支持(200K tokens)
        \item 推理能力优秀
    \end{itemize}
\end{itemize}

\subsection{GLM 系列}

\textbf{GLM-4}:

\begin{itemize}
    \item \textbf{发布机构}:智谱 AI
    \item \textbf{特点}:
    \begin{itemize}
        \item 中文能力优秀
        \item 多模态支持
        \item 开源版本可用
    \end{itemize}
\end{itemize}

\subsection{Mistral 系列}

\textbf{Mistral 7B}:

\begin{itemize}
    \item \textbf{发布机构}:Mistral AI
    \item \textbf{特点}:
    \begin{itemize}
        \item 7B 参数,性能优秀
        \item 开源可商用
        \item 推理效率高
    \end{itemize}
\end{itemize}

\section{预训练模型架构详解}

\subsection{RoBERTa}

\textbf{概念解释}:RoBERTa(Robustly Optimized BERT Pretraining Approach)是 BERT 的优化版本,通过改进训练策略显著提升了性能。

\textbf{改进点}:

\begin{itemize}
    \item \textbf{动态掩码}:每次训练时动态生成掩码,而不是静态掩码
    \item \textbf{移除 NSP 任务}:去除了下一句预测任务
    \item \textbf{更大批次}:使用更大的批次大小(8K)
    \item \textbf{更长训练}:训练更多步数
    \item \textbf{更多数据}:使用更多训练数据
\end{itemize}

\textbf{应用场景}:

\begin{itemize}
    \item \textbf{文本分类}:情感分析、意图识别
    \item \textbf{文本匹配}:相似度计算、检索
    \item \textbf{命名实体识别}:NER 任务
    \item \textbf{嵌入模型}:作为嵌入模型的基础
\end{itemize}

\textbf{微调策略}:

使用对比学习和困难负样本挖掘进行领域适应:
\begin{itemize}
    \item 提升业务意图识别准确率 20\%
    \item 提升长尾查询匹配率 30\%
    \item 提升召回率 25\%
\end{itemize}

\subsection{T5}

\textbf{概念解释}:T5(Text-To-Text Transfer Transformer)将所有 NLP 任务统一为文本到文本的生成任务。

\textbf{架构特点}:

\begin{itemize}
    \item \textbf{编码器-解码器架构}:使用 Transformer 的编码器-解码器结构
    \item \textbf{统一框架}:所有任务都转换为"输入文本 → 输出文本"
    \item \textbf{任务前缀}:通过任务前缀区分不同任务(如 "translate:", "summarize:")
\end{itemize}

\textbf{应用场景}:

\begin{itemize}
    \item \textbf{文本生成}:摘要、翻译、改写
    \item \textbf{查询重写}:将简短查询扩展为完整查询
    \item \textbf{文本转换}:格式转换、风格转换
\end{itemize}

\textbf{微调效果}:

在查询重写任务上:
\begin{itemize}
    \item 提升重写查询与原始意图的相关性 35\%
    \item 有效提升召回准确率
\end{itemize}

\subsection{Cross-Encoder}

\textbf{概念解释}:Cross-Encoder 将查询和文档拼接后输入模型,进行深度交互建模,用于精细排序。

\textbf{架构}:

\begin{equation}
\text{Score}(q, d) = \text{Linear}(\text{CLS}(\text{Transformer}([\text{CLS}] q [\text{SEP}] d)))
\end{equation}

\textbf{优势}:

\begin{itemize}
    \item \textbf{深度交互}:查询和文档在模型内部充分交互
    \item \textbf{高精确度}:比 Bi-Encoder 精确度提升 20-40\%
    \item \textbf{解决误召回}:有效解决"相似政策误召回"问题
\end{itemize}

\textbf{应用}:

在 RAG 系统的排序层:
\begin{itemize}
    \item Top K 数据精确度提升 40\%
    \item P@1 精确度提升 37\%
\end{itemize}

\section{框架与工具}

\subsection{LangChain}

\textbf{概念解释}:LangChain 是一个用于构建基于大语言模型应用的框架,提供了 RAG、Agent 等高级功能的实现。

\textbf{核心组件}:

\begin{itemize}
    \item \textbf{LLMs}:大语言模型接口
    \item \textbf{Chains}:链式调用,组合多个组件
    \item \textbf{Agents}:智能代理,可以调用工具
    \item \textbf{Memory}:记忆管理
    \item \textbf{Vector Stores}:向量数据库集成
\end{itemize}

\textbf{RAG 实现}:

\begin{lstlisting}[language=Python, caption=LangChain RAG 示例]
from langchain.llms import OpenAI
from langchain.vectorstores import FAISS
from langchain.chains import RetrievalQA

# 创建向量存储
vectorstore = FAISS.from_documents(documents, embeddings)

# 创建 RAG 链
qa_chain = RetrievalQA.from_chain_type(
    llm=OpenAI(),
    chain_type="stuff",
    retriever=vectorstore.as_retriever()
)

# 查询
result = qa_chain.run("What is RAG?")
\end{lstlisting}

\textbf{应用场景}:

\begin{itemize}
    \item \textbf{RAG 应用}:快速构建检索增强生成系统
    \item \textbf{智能代理}:构建可以调用工具的 AI 代理
    \item \textbf{对话系统}:构建多轮对话系统
\end{itemize}

\subsection{DeepSpeed}

\textbf{概念解释}:DeepSpeed 是微软开发的深度学习优化库,提供分布式训练、内存优化、模型压缩等功能。

\textbf{核心功能}:

\begin{itemize}
    \item \textbf{ZeRO}:零冗余优化器,大幅降低内存占用
    \item \textbf{梯度压缩}:压缩梯度,减少通信开销
    \item \textbf{混合精度训练}:FP16/BF16 训练,加速训练
    \item \textbf{模型并行}:支持模型并行和数据并行
\end{itemize}

\textbf{ZeRO 优化}:

\begin{itemize}
    \item \textbf{ZeRO-1}:优化器状态分片
    \item \textbf{ZeRO-2}:优化器状态 + 梯度分片
    \item \textbf{ZeRO-3}:优化器状态 + 梯度 + 参数分片
\end{itemize}

\textbf{效果}:

\begin{itemize}
    \item 内存占用降低 4-8 倍
    \item 支持训练更大规模的模型
    \item 训练速度提升 2-4 倍
\end{itemize}

\section{上下文学习(In-Context Learning, ICL)}

\subsection{ICL 原理}

\textbf{概念解释}:In-Context Learning 是大语言模型的核心能力之一,模型通过观察少量示例(few-shot examples)就能理解任务并执行,无需参数更新。

\textbf{数学表示}:

给定任务示例 $(x_1, y_1), (x_2, y_2), \ldots, (x_k, y_k)$ 和查询 $x_{k+1}$,模型预测:

\begin{equation}
P(y_{k+1} | x_1, y_1, \ldots, x_k, y_k, x_{k+1}) = \prod_{i=1}^{|y_{k+1}|} P(y_{k+1}^{(i)} | x_1, y_1, \ldots, x_k, y_k, x_{k+1}, y_{k+1}^{(<i)})
\end{equation}

\textbf{ICL 的特点}:

\begin{itemize}
    \item \textbf{无需训练}:不需要梯度更新,直接推理
    \item \textbf{快速适应}:通过示例快速理解任务
    \item \textbf{灵活性强}:可以处理各种任务
\end{itemize}

\subsection{ICL 的应用场景}

\textbf{角色认知增强}:

在个人数字分身等应用中,使用 ICL 增强模型的角色认知:

\begin{lstlisting}[language=Python, caption=ICL 角色认知示例]
prompt = """
你是一个数字分身,具有以下特点:
- 语言风格:简洁、直接
- 常用表达:"好的"、"没问题"、"明白了"
- 回复习惯:先确认理解,再给出建议

示例对话:
用户:今天天气怎么样?
分身:好的,今天天气晴朗,温度25度,适合外出。

用户:帮我安排一下明天的行程
分身:明白了,我来帮你安排明天的行程...
"""
\end{lstlisting}

\textbf{动态适应}:

通过添加最近的对话记录作为 Context,实现动态适应:

\begin{equation}
\text{Context} = [\text{历史对话}_1, \text{历史对话}_2, \ldots, \text{最近对话}_k]
\end{equation}

这样模型可以:
\begin{itemize}
    \item 理解对话上下文
    \item 保持对话连贯性
    \item 适应最新的对话风格
\end{itemize}

\section{长上下文管理}

\subsection{基于轮次衰减的上下文管理}

\textbf{概念解释}:在长对话中,需要管理上下文长度,保留重要信息,压缩或删除不重要信息。

\textbf{重要性评分公式}:

\begin{equation}
S_i = e^{-\lambda \Delta t_i}
\end{equation}

其中:
\begin{itemize}
    \item $S_i$ 是第 $i$ 轮对话的重要性分数
    \item $\Delta t_i$ 是当前轮次与第 $i$ 轮的距离(轮次数)
    \item $\lambda$ 是衰减系数(通常 $\lambda = 0.2$)
\end{itemize}

\textbf{上下文策略}:

\begin{enumerate}
    \item \textbf{保留高重要性对话}:保留最近 $K$ 轮高重要性分数的对话
    \item \textbf{语义摘要}:对低重要性分数的旧对话进行语义摘要
    \item \textbf{混合输入}:构造"摘要 + 最近上下文"的混合输入
\end{enumerate}

\textbf{效果}:

\begin{itemize}
    \item 有效解决长对话中的遗忘问题
    \item 提升回复连贯性 25-30\%
    \item 减少上下文长度 40-60\%
\end{itemize}

\section{Agent、Function Calling、Tools 与 MCP}

\subsection{Agent(智能代理)}

\textbf{严谨解释}:

Agent 是一个能够感知环境、做出决策并执行动作的自主系统。在大语言模型应用中,Agent 通常指能够使用工具、与环境交互、完成复杂任务的智能系统。

\textbf{数学表示}:

Agent 可以形式化为一个元组:
\begin{equation}
\text{Agent} = (\mathcal{S}, \mathcal{A}, \mathcal{T}, \mathcal{R}, \pi)
\end{equation}

其中:
\begin{itemize}
    \item $\mathcal{S}$ 是状态空间(当前上下文、对话历史等)
    \item $\mathcal{A}$ 是动作空间(生成文本、调用工具等)
    \item $\mathcal{T}: \mathcal{S} \times \mathcal{A} \rightarrow \mathcal{S}$ 是状态转移函数
    \item $\mathcal{R}: \mathcal{S} \times \mathcal{A} \rightarrow \mathbb{R}$ 是奖励函数
    \item $\pi: \mathcal{S} \rightarrow \mathcal{A}$ 是策略函数(由 LLM 实现)
\end{itemize}

\textbf{通俗解释}:

Agent 就像一个"AI助手",它不仅能回答问题,还能:
\begin{itemize}
    \item \textbf{思考}:分析当前情况,决定下一步做什么
    \item \textbf{行动}:调用工具(如搜索、计算、查询数据库)
    \item \textbf{观察}:获取工具返回的结果
    \item \textbf{调整}:根据结果调整策略,继续执行任务
\end{itemize}

就像人类助手一样:你让他"查一下明天的天气",他会先思考"需要调用天气API",然后执行,最后告诉你结果。

\textbf{Agent 的核心组件}:

\begin{enumerate}
    \item \textbf{LLM 核心}:负责推理和决策
    \item \textbf{工具集(Tools)}:Agent 可以调用的外部功能
    \item \textbf{记忆(Memory)}:存储对话历史和中间结果
    \item \textbf{规划器(Planner)}:制定执行计划
    \item \textbf{执行器(Executor)}:执行工具调用
\end{enumerate}

\textbf{Agent 的类型}:

\begin{itemize}
    \item \textbf{ReAct Agent}:结合推理(Reasoning)和行动(Acting)
    \item \textbf{Plan-and-Execute Agent}:先制定计划,再执行
    \item \textbf{AutoGPT Agent}:自主规划多步骤任务
    \item \textbf{BabyAGI Agent}:基于目标的任务管理
\end{itemize}

\subsection{Function Calling(函数调用)}

\textbf{严谨解释}:

Function Calling 是大语言模型的一种能力,允许模型在生成文本时,识别需要调用外部函数的场景,并生成符合函数签名的结构化调用请求。

\textbf{数学表示}:

给定输入 $x$ 和函数集合 $\mathcal{F} = \{f_1, f_2, \ldots, f_n\}$,Function Calling 的过程为:

\begin{equation}
\text{FunctionCall}(x, \mathcal{F}) = \arg\max_{f \in \mathcal{F}} P(f|x) \cdot \text{GenerateArgs}(f, x)
\end{equation}

其中:
\begin{itemize}
    \item $P(f|x)$ 是模型判断是否需要调用函数 $f$ 的概率
    \item $\text{GenerateArgs}(f, x)$ 是生成函数参数的过程
\end{itemize}

\textbf{通俗解释}:

Function Calling 就像给AI一个"工具箱":
\begin{itemize}
    \item AI 看到一个问题,比如"今天北京天气怎么样?"
    \item AI 意识到需要调用"天气查询"函数
    \item AI 生成函数调用:\texttt{get\_weather(location="北京")}
    \item 系统执行函数,返回结果
    \item AI 基于结果生成最终回复
\end{itemize}

这就像你告诉助手"查天气",助手知道要去调用天气API,而不是自己"编造"天气信息。

\textbf{Function Calling 的工作流程}:

\begin{enumerate}
    \item \textbf{函数定义}:定义可用的函数及其参数
    \item \textbf{模型推理}:模型分析输入,决定是否需要调用函数
    \item \textbf{参数生成}:如果需要,生成函数调用的参数
    \item \textbf{函数执行}:系统执行函数调用
    \item \textbf{结果整合}:将函数结果整合到模型回复中
\end{enumerate}

\textbf{Function Calling 的优势}:

\begin{itemize}
    \item \textbf{实时信息}:可以获取最新数据(如天气、股价)
    \item \textbf{精确计算}:可以调用计算工具,避免模型计算错误
    \item \textbf{外部系统集成}:可以连接数据库、API、文件系统等
    \item \textbf{可扩展性}:可以轻松添加新功能
\end{itemize}

\subsection{Tools(工具)}

\textbf{严谨解释}:

Tools 是 Agent 可以调用的外部功能接口,通常定义为函数或 API,用于扩展模型的能力边界。

\textbf{工具定义}:

一个工具可以形式化为:
\begin{equation}
\text{Tool} = (\text{name}, \text{description}, \text{parameters}, \text{execute})
\end{equation}

其中:
\begin{itemize}
    \item $\text{name}$ 是工具名称
    \item $\text{description}$ 是工具描述(用于模型理解)
    \item $\text{parameters}$ 是参数模式(JSON Schema)
    \item $\text{execute}$ 是执行函数
\end{itemize}

\textbf{通俗解释}:

Tools 就像给AI的"超能力":
\begin{itemize}
    \item \textbf{搜索工具}:让AI可以搜索互联网
    \item \textbf{计算工具}:让AI可以做精确计算
    \item \textbf{文件工具}:让AI可以读写文件
    \item \textbf{数据库工具}:让AI可以查询数据库
    \item \textbf{API工具}:让AI可以调用各种API
\end{itemize}

就像给机器人装上各种"手臂"和"传感器",让它能做更多事情。

\textbf{常见工具类型}:

\begin{itemize}
    \item \textbf{信息检索}:搜索引擎、向量数据库
    \item \textbf{计算工具}:计算器、代码执行器
    \item \textbf{文件操作}:文件读写、目录遍历
    \item \textbf{网络工具}:HTTP请求、API调用
    \item \textbf{数据库工具}:SQL查询、NoSQL操作
    \item \textbf{系统工具}:系统命令、进程管理
\end{itemize}

\subsection{MCP(Model Context Protocol)}

\textbf{严谨解释}:

MCP(Model Context Protocol)是一个标准化的协议,用于定义模型如何与外部工具和上下文进行交互。它提供了一套统一的接口规范,使得不同的模型和工具可以无缝集成。

\textbf{协议结构}:

MCP 定义了三层协议:
\begin{enumerate}
    \item \textbf{Context Layer}:定义上下文如何传递和管理
    \item \textbf{Tool Layer}:定义工具如何注册和调用
    \item \textbf{Protocol Layer}:定义通信协议和消息格式
\end{enumerate}

\textbf{通俗解释}:

MCP 就像"通用插头标准":
\begin{itemize}
    \item 不同的AI模型(如GPT-4、Claude)就像不同的"电器"
    \item 不同的工具(如搜索、计算)就像不同的"插座"
    \item MCP 定义了"插头标准",让任何"电器"都能插到任何"插座"上
\end{itemize}

这就像USB标准:有了USB标准,任何USB设备都能连接到任何USB接口。

\textbf{MCP 的核心特性}:

\begin{itemize}
    \item \textbf{标准化}:统一的接口规范
    \item \textbf{可扩展}:易于添加新工具
    \item \textbf{可组合}:工具可以组合使用
    \item \textbf{类型安全}:强类型的参数定义
\end{itemize}

\textbf{MCP 的优势}:

\begin{itemize}
    \item \textbf{互操作性}:不同模型和工具可以无缝协作
    \item \textbf{可维护性}:统一的协议便于维护和更新
    \item \textbf{可扩展性}:易于添加新功能
    \item \textbf{标准化}:减少集成成本
\end{itemize}

\subsection{Agent、Function Calling、Tools、MCP 对比}

\begin{table}[h]
\centering
\begin{tabular}{lcccc}
\toprule
\textbf{特性} & \textbf{Agent} & \textbf{Function Calling} & \textbf{Tools} & \textbf{MCP} \\
\midrule
定位 & 智能系统 & 模型能力 & 功能接口 & 协议标准 \\
\midrule
范围 & 完整系统 & 调用机制 & 单个功能 & 通信协议 \\
\midrule
自主性 & 高(自主决策) & 中(模型决定) & 低(被动调用) & 无(协议层) \\
\midrule
复杂度 & 高 & 中 & 低 & 中 \\
\midrule
主要用途 & 复杂任务执行 & 外部功能调用 & 能力扩展 & 标准化集成 \\
\bottomrule
\end{tabular}
\caption{Agent、Function Calling、Tools、MCP 对比}
\end{table}

\textbf{关系说明}:

\begin{itemize}
    \item \textbf{Agent} 是完整的智能系统,可以使用 \textbf{Tools} 来扩展能力
    \item \textbf{Function Calling} 是 Agent 使用 Tools 的机制
    \item \textbf{MCP} 定义了 Agent、Tools 之间的通信协议
    \item \textbf{Tools} 是 Agent 可以调用的具体功能
\end{itemize}

\textbf{通俗类比}:

\begin{itemize}
    \item \textbf{Agent} = 智能机器人(完整的系统)
    \item \textbf{Tools} = 机器人的各种工具(锤子、螺丝刀等)
    \item \textbf{Function Calling} = 机器人使用工具的方式("拿起锤子"、"敲打")
    \item \textbf{MCP} = 工具接口标准(确保工具能装到机器人上)
\end{itemize}

\subsection{实际应用案例}

\subsubsection{案例1:天气查询 Agent}

\textbf{场景}:用户询问"明天北京的天气如何?"

\textbf{实现步骤}:

\begin{enumerate}
    \item \textbf{工具定义}(MCP 格式):
    \begin{lstlisting}[language=Python, caption=天气查询工具定义]
{
    "name": "get_weather",
    "description": "查询指定城市的天气信息",
    "parameters": {
        "type": "object",
        "properties": {
            "location": {
                "type": "string",
                "description": "城市名称"
            },
            "date": {
                "type": "string",
                "description": "日期,格式:YYYY-MM-DD"
            }
        },
        "required": ["location"]
    }
}
    \end{lstlisting}
    
    \item \textbf{Agent 推理}:
    \begin{lstlisting}[language=Python, caption=Agent 推理过程]
# 用户输入
user_input = "明天北京的天气如何?"

# Agent 分析
# 1. 识别意图:需要查询天气
# 2. 提取信息:location="北京", date="明天"
# 3. 决定调用:get_weather 函数
    \end{lstlisting}
    
    \item \textbf{Function Calling}:
    \begin{lstlisting}[language=Python, caption=Function Calling 执行]
# 模型生成函数调用
function_call = {
    "name": "get_weather",
    "arguments": {
        "location": "北京",
        "date": "2025-01-07"  # 明天的日期
    }
}

# 执行函数
result = execute_function(function_call)
# 返回:{"temperature": 5, "condition": "晴", "wind": "3级"}
    \end{lstlisting}
    
    \item \textbf{生成回复}:
    \begin{lstlisting}[language=Python, caption=Agent 生成最终回复]
# Agent 基于结果生成回复
response = f"""
根据查询结果,明天(2025-01-07)北京的天气情况:
- 温度:5°C
- 天气:晴天
- 风力:3级

建议穿着轻便外套,适合外出活动。
"""
    \end{lstlisting}
\end{enumerate}

\subsubsection{案例2:数据分析 Agent}

\textbf{场景}:用户要求"分析销售数据,找出销量最好的产品"

\textbf{实现步骤}:

\begin{enumerate}
    \item \textbf{工具定义}:
    \begin{lstlisting}[language=Python, caption=数据分析工具定义]
tools = [
    {
        "name": "read_csv",
        "description": "读取CSV文件",
        "parameters": {
            "type": "object",
            "properties": {
                "file_path": {"type": "string"}
            }
        }
    },
    {
        "name": "query_data",
        "description": "查询数据",
        "parameters": {
            "type": "object",
            "properties": {
                "sql": {"type": "string"}
            }
        }
    },
    {
        "name": "analyze_data",
        "description": "分析数据",
        "parameters": {
            "type": "object",
            "properties": {
                "data": {"type": "array"},
                "operation": {"type": "string"}
            }
        }
    }
]
    \end{lstlisting}
    
    \item \textbf{Agent 执行流程}:
    \begin{lstlisting}[language=Python, caption=数据分析 Agent 执行]
# 步骤1:读取数据
result1 = agent.call_tool("read_csv", {"file_path": "sales.csv"})

# 步骤2:查询销量数据
result2 = agent.call_tool("query_data", {
    "sql": "SELECT product, SUM(quantity) as total FROM sales GROUP BY product ORDER BY total DESC"
})

# 步骤3:分析结果
result3 = agent.call_tool("analyze_data", {
    "data": result2,
    "operation": "find_max",
    "field": "total"
})

# 步骤4:生成报告
report = agent.generate_response(
    f"根据数据分析,销量最好的产品是:{result3['product']},"
    f"总销量为:{result3['total']}件"
)
    \end{lstlisting}
\end{enumerate}

\subsubsection{案例3:代码生成与执行 Agent}

\textbf{场景}:用户要求"写一个函数计算斐波那契数列,并测试前10个数"

\textbf{实现步骤}:

\begin{lstlisting}[language=Python, caption=代码生成与执行 Agent]
# 工具定义
tools = [
    {
        "name": "generate_code",
        "description": "生成Python代码",
        "parameters": {
            "type": "object",
            "properties": {
                "task": {"type": "string"},
                "language": {"type": "string", "default": "python"}
            }
        }
    },
    {
        "name": "execute_code",
        "description": "执行Python代码",
        "parameters": {
            "type": "object",
            "properties": {
                "code": {"type": "string"}
            }
        }
    }
]

# Agent 执行
# 1. 生成代码
code = agent.call_tool("generate_code", {
    "task": "写一个函数计算斐波那契数列"
})
# 返回:
# def fibonacci(n):
#     if n <= 1:
#         return n
#     return fibonacci(n-1) + fibonacci(n-2)

# 2. 执行测试
test_code = f"""
{code}
for i in range(10):
    print(fibonacci(i))
"""
result = agent.call_tool("execute_code", {"code": test_code})

# 3. 生成回复
response = f"""
已生成斐波那契函数并测试前10个数:
{result}
"""
\end{lstlisting}

\subsubsection{案例4:多工具组合 Agent}

\textbf{场景}:用户要求"搜索最新的AI论文,总结要点,并保存到文件"

\textbf{实现步骤}:

\begin{lstlisting}[language=Python, caption=多工具组合 Agent]
# 工具链
tools = [
    {"name": "search_web", "description": "搜索网页"},
    {"name": "summarize_text", "description": "总结文本"},
    {"name": "write_file", "description": "写入文件"}
]

# Agent 执行流程
# 步骤1:搜索论文
papers = agent.call_tool("search_web", {
    "query": "latest AI papers 2025",
    "num_results": 5
})

# 步骤2:总结每篇论文
summaries = []
for paper in papers:
    summary = agent.call_tool("summarize_text", {
        "text": paper["content"],
        "max_length": 200
    })
    summaries.append({
        "title": paper["title"],
        "summary": summary
    })

# 步骤3:保存到文件
agent.call_tool("write_file", {
    "file_path": "ai_papers_summary.txt",
    "content": "\n\n".join([f"{s['title']}\n{s['summary']}" for s in summaries])
})

# 步骤4:生成回复
response = f"""
已完成任务:
1. 搜索了5篇最新的AI论文
2. 总结了每篇论文的要点
3. 保存到文件:ai_papers_summary.txt
"""
\end{lstlisting}

\subsection{使用 LangChain 实现 Agent}

\textbf{ReAct Agent 示例}:

\begin{lstlisting}[language=Python, caption=LangChain ReAct Agent]
from langchain.agents import initialize_agent, Tool
from langchain.llms import OpenAI
from langchain.chains import LLMChain

# 1. 定义工具
def search_tool(query: str) -> str:
    """搜索工具"""
    # 实际实现搜索逻辑
    return f"搜索结果:{query}"

def calculator_tool(expression: str) -> str:
    """计算器工具"""
    try:
        result = eval(expression)
        return str(result)
    except:
        return "计算错误"

tools = [
    Tool(
        name="Search",
        func=search_tool,
        description="用于搜索最新信息"
    ),
    Tool(
        name="Calculator",
        func=calculator_tool,
        description="用于数学计算"
    )
]

# 2. 初始化 Agent
llm = OpenAI(temperature=0)
agent = initialize_agent(
    tools,
    llm,
    agent="react-chat",
    verbose=True
)

# 3. 使用 Agent
response = agent.run("搜索'Python编程',然后计算 123 * 456")
print(response)
\end{lstlisting}

\textbf{自定义 Agent 示例}:

\begin{lstlisting}[language=Python, caption=自定义 Agent 实现]
from langchain.agents import AgentExecutor, create_react_agent
from langchain.prompts import PromptTemplate
from langchain.tools import Tool

# 1. 定义提示模板
prompt = PromptTemplate.from_template("""
你是一个有用的AI助手,可以使用以下工具:
{tools}

使用以下格式:
Question: 输入的问题
Thought: 你应该思考要做什么
Action: 要采取的行动,应该是[{tool_names}]中的一个
Action Input: 行动的输入
Observation: 行动的结果
... (这个思考/行动/行动输入/观察可以重复N次)
Thought: 我现在知道最终答案了
Final Answer: 原始输入问题的最终答案

Question: {input}
Thought: {agent_scratchpad}
""")

# 2. 创建 Agent
agent = create_react_agent(llm, tools, prompt)

# 3. 执行 Agent
agent_executor = AgentExecutor(agent=agent, tools=tools, verbose=True)
result = agent_executor.invoke({"input": "计算 100 的平方根"})
\end{lstlisting}

\subsection{最佳实践}

\textbf{1. 工具设计原则}:

\begin{itemize}
    \item \textbf{单一职责}:每个工具只做一件事
    \item \textbf{清晰描述}:工具描述要清晰,帮助模型理解何时使用
    \item \textbf{参数验证}:严格验证参数,避免错误调用
    \item \textbf{错误处理}:优雅处理错误,返回有用信息
\end{itemize}

\textbf{2. Agent 设计原则}:

\begin{itemize}
    \item \textbf{明确目标}:Agent 应该有明确的任务目标
    \item \textbf{有限工具}:不要给 Agent 太多工具,避免混乱
    \item \textbf{错误恢复}:设计错误恢复机制
    \item \textbf{可观测性}:记录 Agent 的决策过程
\end{itemize}

\textbf{3. Function Calling 最佳实践}:

\begin{itemize}
    \item \textbf{函数命名}:使用清晰、描述性的函数名
    \item \textbf{参数设计}:参数应该明确、类型安全
    \item \textbf{返回值}:返回结构化数据,便于模型理解
    \item \textbf{文档完善}:提供详细的函数文档
\end{itemize}

\textbf{4. MCP 使用建议}:

\begin{itemize}
    \item \textbf{遵循标准}:严格遵循 MCP 协议规范
    \item \textbf{类型安全}:使用强类型的参数定义
    \item \textbf{版本管理}:管理协议版本,保持兼容性
    \item \textbf{错误处理}:定义标准的错误码和错误信息
\end{itemize}

\section{更多大模型算法与技术}

\textbf{说明}:本节介绍其他重要的大模型算法和技术。关于强化学习相关内容(RLHF、PPO、DPO、GRPO),请参考前面的"大语言模型中的强化学习"章节。

\subsection{Orca 和 Orca 2}

\textbf{概念解释}:Orca 是微软提出的通过模仿学习提升小模型能力的框架。

\textbf{核心思想}:

\begin{itemize}
    \item 使用大模型(如 GPT-4)生成高质量数据
    \item 小模型学习大模型的推理过程
    \item 通过逐步推理(step-by-step reasoning)提升能力
\end{itemize}

\textbf{训练数据}:

\begin{equation}
\mathcal{D} = \{(x, y_{\text{teacher}}, \text{reasoning}_{\text{teacher}})\}
\end{equation}

其中 $\text{reasoning}_{\text{teacher}}$ 是大模型的推理过程。

\subsection{Chain-of-Thought (CoT)}

\textbf{概念解释}:Chain-of-Thought 通过引导模型进行逐步推理,提升复杂推理任务的表现。

\textbf{示例}:

\begin{lstlisting}[language=Python, caption=CoT 示例]
prompt = """
Q: 一个商店有15个苹果,卖出了8个,又进货了12个,现在有多少个?

A: 让我们一步步思考:
1. 初始有15个苹果
2. 卖出了8个,剩余:15 - 8 = 7个
3. 又进货了12个,现在有:7 + 12 = 19个
所以答案是19个。
"""
\end{lstlisting}

\textbf{效果}:

\begin{itemize}
    \item 在数学推理任务上提升 20-30\%
    \item 在逻辑推理任务上提升 15-25\%
\end{itemize}

\subsection{Tree of Thoughts (ToT)}

\textbf{概念解释}:ToT 扩展了 CoT,通过树状结构探索多个推理路径。

\textbf{算法流程}:

\begin{enumerate}
    \item \textbf{生成}:为当前状态生成多个可能的推理步骤
    \item \textbf{评估}:评估每个步骤的质量
    \item \textbf{搜索}:使用广度优先或深度优先搜索最佳路径
\end{enumerate}

\textbf{优势}:

\begin{itemize}
    \item 探索多个解决方案
    \item 回溯能力,可以修正错误
    \item 在复杂推理任务上表现更好
\end{itemize}

\subsection{ReAct(Reasoning + Acting)}

\textbf{概念解释}:ReAct 结合推理和行动,使模型可以调用外部工具。

\textbf{框架}:

\begin{equation}
\text{Action} = \text{Reasoning} + \text{Acting} + \text{Observation}
\end{equation}

\textbf{示例}:

\begin{lstlisting}[language=Python, caption=ReAct 示例]
# 模型可以:
# 1. 思考:需要查询天气信息
# 2. 行动:调用天气API
# 3. 观察:获取天气数据
# 4. 思考:基于天气数据给出建议
# 5. 回答:根据天气建议穿衣
\end{lstlisting}

\subsection{Mixture of Experts (MoE)}

\textbf{概念解释}:MoE 通过稀疏激活多个专家模型,在保持参数量的同时提升模型容量。

\textbf{架构}:

\begin{equation}
y = \sum_{i=1}^{n} g_i(x) \cdot E_i(x)
\end{equation}

其中:
\begin{itemize}
    \item $E_i$ 是第 $i$ 个专家模型
    \item $g_i(x)$ 是门控函数,决定激活哪些专家
    \item 通常只激活 $k$ 个专家(如 $k=2$)
\end{itemize}

\textbf{优势}:

\begin{itemize}
    \item 参数总量大,但激活参数少
    \item 训练和推理效率高
    \item 模型容量大
\end{itemize}

\textbf{代表模型}:

\begin{itemize}
    \item \textbf{Switch Transformer}:Google 的 MoE 模型
    \item \textbf{GLaM}:Google 的通用语言模型
    \item \textbf{Mixtral}:Mistral AI 的 MoE 模型
\end{itemize}

\subsection{Retrieval-Augmented Generation (RAG) 进阶}

\textbf{Self-RAG}:

Self-RAG 让模型自主决定何时检索、如何检索:

\begin{enumerate}
    \item \textbf{检索决策}:判断是否需要检索
    \item \textbf{检索执行}:如果需要,执行检索
    \item \textbf{生成}:基于检索结果生成回复
    \item \textbf{自我评估}:评估生成质量
\end{enumerate}

\textbf{Corrective RAG}:

Corrective RAG 通过错误检测和纠正提升 RAG 质量:

\begin{enumerate}
    \item 生成初始回复
    \item 检测错误或不确定部分
    \item 针对错误部分重新检索
    \item 生成纠正后的回复
\end{enumerate}

\subsection{多模态大模型}

\textbf{CLIP}:

\begin{itemize}
    \item \textbf{架构}:图像编码器 + 文本编码器
    \item \textbf{训练}:对比学习,拉近匹配的图像-文本对
    \item \textbf{应用}:图像检索、图像生成(DALL-E)
\end{itemize}

\textbf{GPT-4V}:

\begin{itemize}
    \item \textbf{能力}:理解图像和文本
    \item \textbf{应用}:图像问答、图像描述、视觉推理
\end{itemize}

\textbf{LLaVA}:

\begin{itemize}
    \item \textbf{架构}:视觉编码器(CLIP) + 语言模型(LLaMA)
    \item \textbf{训练}:视觉指令微调
    \item \textbf{应用}:视觉问答、图像理解
\end{itemize}

\subsection{代码生成模型}

\textbf{Codex / GitHub Copilot}:

\begin{itemize}
    \item \textbf{基础模型}:GPT-3
    \item \textbf{训练数据}:GitHub 代码
    \item \textbf{能力}:代码生成、代码补全、代码解释
\end{itemize}

\textbf{CodeLlama}:

\begin{itemize}
    \item \textbf{基础模型}:LLaMA 2
    \item \textbf{特点}:专门针对代码训练
    \item \textbf{能力}:代码生成、代码补全、代码调试
\end{itemize}

\textbf{StarCoder}:

\begin{itemize}
    \item \textbf{参数量}:15B
    \item \textbf{训练数据}:800+ 编程语言
    \item \textbf{特点}:开源、可商用
\end{itemize}

\subsection{评估基准补充}

\textbf{HellaSwag}:

\begin{itemize}
    \item \textbf{任务}:常识推理
    \item \textbf{格式}:选择题,选择最合理的句子续写
    \item \textbf{评估}:准确率
\end{itemize}

\textbf{TruthfulQA}:

\begin{itemize}
    \item \textbf{任务}:评估模型的真实性
    \item \textbf{关注点}:避免生成错误信息
    \item \textbf{评估}:真实率(Truthfulness)
\end{itemize}

\textbf{GSM8K}:

\begin{itemize}
    \item \textbf{任务}:小学数学问题
    \item \textbf{评估}:准确率
    \item \textbf{特点}:需要多步推理
\end{itemize}

\textbf{HumanEval}:

\begin{itemize}
    \item \textbf{任务}:Python 代码生成
    \item \textbf{评估}:通过率(Pass@k)
    \item \textbf{特点}:164 个编程问题
\end{itemize}

\section{总结}

第一部分介绍了大语言模型的先进技术:

\textbf{参数高效微调}:
\begin{itemize}
    \item LoRA:低秩适应,大幅减少参数量
    \item QLoRA:量化 + LoRA,进一步降低内存
    \item PEFT:统一框架,支持多种方法
\end{itemize}

\textbf{监督微调}:
\begin{itemize}
    \item SFT:有监督微调
    \item 指令微调:提升指令遵循能力
    \item 对话微调:优化多轮对话
\end{itemize}

\textbf{推理加速}:
\begin{itemize}
    \item vLLM:PagedAttention 优化
    \item Flash Attention:内存高效注意力
    \item 量化推理:INT8/INT4 量化
\end{itemize}

\textbf{部署与优化}:
\begin{itemize}
    \item 模型压缩:知识蒸馏、剪枝
    \item 服务化部署:API 服务
    \item 推理优化:KV Cache、连续批处理
\end{itemize}

\section{评估指标与方法}

评估指标是衡量大语言模型性能的重要工具。不同的任务需要不同的评估指标,本节详细介绍各种评估指标的定义、计算方法和实现。

\subsection{困惑度(Perplexity)}

\textbf{概念解释}:困惑度(Perplexity, PPL)是语言模型评估中最常用的指标,衡量模型对测试数据的预测不确定性。困惑度越低,模型性能越好。

\textbf{数学定义}:

对于测试序列 $\mathbf{w} = w_1, w_2, \ldots, w_N$,困惑度定义为:
\begin{equation}
\text{PPL}(\mathbf{w}) = P(w_1, w_2, \ldots, w_N)^{-\frac{1}{N}} = \sqrt[N]{\frac{1}{P(w_1, w_2, \ldots, w_N)}}
\end{equation}

使用链式法则展开:
\begin{equation}
\text{PPL}(\mathbf{w}) = \sqrt[N]{\prod_{i=1}^N \frac{1}{P(w_i | w_1, \ldots, w_{i-1})}}
\end{equation}

使用对数形式(数值稳定):
\begin{equation}
\text{PPL}(\mathbf{w}) = \exp\left(-\frac{1}{N}\sum_{i=1}^N \log P(w_i | w_1, \ldots, w_{i-1})\right)
\end{equation}

\textbf{符号说明}:
\begin{itemize}
    \item $N$:序列长度(词数)
    \item $w_i$:第 $i$ 个词
    \item $P(w_i | w_1, \ldots, w_{i-1})$:给定前文的条件概率
    \item $\log$:自然对数
\end{itemize}

\textbf{从零实现}:

\begin{lstlisting}[caption=困惑度从零实现]
import torch
import torch.nn.functional as F
import math

def calculate_perplexity(model, tokenizer, text, device="cuda"):
    """
    计算文本的困惑度
    
    参数:
        model: 语言模型
        tokenizer: 分词器
        text: 输入文本
        device: 设备
    """
    # 分词
    tokens = tokenizer.encode(text, return_tensors="pt").to(device)
    input_ids = tokens[:, :-1]  # 输入(除了最后一个token)
    target_ids = tokens[:, 1:]   # 目标(除了第一个token)
    
    model.eval()
    with torch.no_grad():
        # 前向传播
        outputs = model(input_ids)
        logits = outputs.logits  # (batch_size, seq_len, vocab_size)
        
        # 计算每个位置的对数概率
        log_probs = F.log_softmax(logits, dim=-1)
        
        # 获取目标token的对数概率
        # log_probs: (batch_size, seq_len, vocab_size)
        # target_ids: (batch_size, seq_len)
        # 需要选择每个位置对应target的对数概率
        selected_log_probs = log_probs.gather(
            dim=2,
            index=target_ids.unsqueeze(2)
        ).squeeze(2)  # (batch_size, seq_len)
        
        # 计算平均负对数似然
        nll = -selected_log_probs.mean().item()
        
        # 困惑度
        perplexity = math.exp(nll)
    
    return perplexity, nll

# 使用示例
from transformers import AutoModelForCausalLM, AutoTokenizer

model_name = "gpt2"  # 示例模型
model = AutoModelForCausalLM.from_pretrained(model_name)
tokenizer = AutoTokenizer.from_pretrained(model_name)

text = "The quick brown fox jumps over the lazy dog."
ppl, nll = calculate_perplexity(model, tokenizer, text)
print(f"文本: {text}")
print(f"负对数似然: {nll:.4f}")
print(f"困惑度: {ppl:.4f}")
\end{lstlisting}

\textbf{使用库实现}:

\begin{lstlisting}[caption=使用 evaluate 库计算困惑度]
from evaluate import load

perplexity = load("perplexity", module_type="metric")

# 计算困惑度
results = perplexity.compute(
    model_id="gpt2",
    add_start_token=False,
    predictions=["The quick brown fox jumps over the lazy dog."]
)

print(f"困惑度: {results['mean_perplexity']:.4f}")
\end{lstlisting}

\textbf{计算案例}:

\textbf{案例1}:给定序列 "the cat sat",假设模型预测的概率为:
\begin{itemize}
    \item $P(\text{cat} | \text{the}) = 0.3$
    \item $P(\text{sat} | \text{the cat}) = 0.5$
\end{itemize}

\textbf{逐步计算}:
\begin{align}
\text{PPL} &= \exp\left(-\frac{1}{2}[\log(0.3) + \log(0.5)]\right) \\
&= \exp\left(-\frac{1}{2}[-1.204 + (-0.693)]\right) \\
&= \exp\left(-\frac{1}{2} \times (-1.897)\right) \\
&= \exp(0.9485) \\
&= 2.582
\end{align}

\textbf{代码验证}:

\begin{lstlisting}[caption=案例1代码验证]
import math

# 给定概率
probs = [0.3, 0.5]

# 计算困惑度
log_probs = [math.log(p) for p in probs]
nll = -sum(log_probs) / len(probs)
ppl = math.exp(nll)

print(f"概率: {probs}")
print(f"对数概率: {log_probs}")
print(f"平均负对数似然: {nll:.4f}")
print(f"困惑度: {ppl:.4f}")  # 2.582
\end{lstlisting}

\textbf{案例2}:更长的序列

假设序列长度为 10,平均对数概率为 -2.0:

\begin{align}
\text{PPL} &= \exp\left(-\frac{1}{10} \times (-2.0) \times 10\right) \\
&= \exp(2.0) \\
&= 7.389
\end{align}

\subsection{BLEU 分数}

\textbf{概念解释}:BLEU(Bilingual Evaluation Understudy)是机器翻译和文本生成任务中最常用的评估指标,通过比较 n-gram 匹配来衡量生成文本与参考文本的相似度。

\textbf{数学定义}:

\textbf{n-gram 精确度}:
\begin{equation}
P_n = \frac{\sum_{\text{n-gram} \in \text{candidate}} \text{Count}_{\text{clip}}(\text{n-gram})}{\sum_{\text{n-gram} \in \text{candidate}} \text{Count}(\text{n-gram})}
\end{equation}

其中 $\text{Count}_{\text{clip}}$ 是截断计数,不超过参考文本中该 n-gram 的最大出现次数。

\textbf{BLEU 分数}:
\begin{equation}
\text{BLEU} = \text{BP} \times \exp\left(\sum_{n=1}^N w_n \log P_n\right)
\end{equation}

其中:
\begin{itemize}
    \item $\text{BP}$:简短惩罚(Brevity Penalty)
    \item $w_n$:n-gram 权重,通常 $w_n = 1/N$
    \item $N$:最大 n-gram 阶数,通常 $N=4$
\end{itemize}

\textbf{简短惩罚}:
\begin{equation}
\text{BP} = \begin{cases}
1 & \text{if } c > r \\
e^{1-r/c} & \text{if } c \leq r
\end{cases}
\end{equation}

其中 $c$ 是候选文本长度,$r$ 是参考文本长度。

\textbf{从零实现}:

\begin{lstlisting}[caption=BLEU 分数从零实现]
from collections import Counter
import math

def get_ngrams(tokens, n):
    """获取 n-gram"""
    return [tuple(tokens[i:i+n]) for i in range(len(tokens)-n+1)]

def calculate_bleu(reference, candidate, max_n=4):
    """
    计算 BLEU 分数
    
    参数:
        reference: 参考文本(列表,每个元素是一个参考)
        candidate: 候选文本(token列表)
        max_n: 最大 n-gram 阶数
    """
    # 如果 reference 是字符串,转换为列表
    if isinstance(reference[0], str):
        reference = [ref.split() for ref in reference]
    if isinstance(candidate, str):
        candidate = candidate.split()
    
    # 计算简短惩罚
    c = len(candidate)
    r = min(len(ref) for ref in reference)  # 最接近的参考长度
    bp = 1.0 if c > r else math.exp(1 - r / c)
    
    # 计算各阶 n-gram 精确度
    precisions = []
    
    for n in range(1, max_n + 1):
        # 候选文本的 n-gram
        candidate_ngrams = get_ngrams(candidate, n)
        candidate_counts = Counter(candidate_ngrams)
        
        # 参考文本的 n-gram(所有参考)
        reference_counts_list = []
        for ref in reference:
            ref_ngrams = get_ngrams(ref, n)
            reference_counts_list.append(Counter(ref_ngrams))
        
        # 计算截断计数
        clipped_count = 0
        total_count = sum(candidate_counts.values())
        
        for ngram, count in candidate_counts.items():
            # 在所有参考中找到该 n-gram 的最大计数
            max_ref_count = max(
                ref_counts.get(ngram, 0) for ref_counts in reference_counts_list
            )
            clipped_count += min(count, max_ref_count)
        
        # n-gram 精确度
        precision = clipped_count / total_count if total_count > 0 else 0
        precisions.append(precision)
    
    # 计算几何平均
    if any(p == 0 for p in precisions):
        return 0.0
    
    log_precision_sum = sum(math.log(p) for p in precisions)
    bleu = bp * math.exp(log_precision_sum / len(precisions))
    
    return bleu, precisions, bp

# 测试案例
reference = ["the cat is on the mat"]
candidate = "the cat the cat on the mat"

bleu_score, precisions, bp = calculate_bleu(reference, candidate)
print(f"参考: {reference}")
print(f"候选: {candidate}")
print(f"1-gram 精确度: {precisions[0]:.4f}")
print(f"2-gram 精确度: {precisions[1]:.4f}")
print(f"3-gram 精确度: {precisions[2]:.4f}")
print(f"4-gram 精确度: {precisions[3]:.4f}")
print(f"简短惩罚: {bp:.4f}")
print(f"BLEU 分数: {bleu_score:.4f}")
\end{lstlisting}

\textbf{使用库实现}:

\begin{lstlisting}[caption=使用 sacrebleu 库]
from sacrebleu import BLEU

bleu = BLEU()

# 单个参考
reference = ["the cat is on the mat"]
candidate = "the cat the cat on the mat"
score = bleu.sentence_score(candidate, reference)
print(f"BLEU 分数: {score.score:.4f}")

# 多个参考
references = [
    ["the cat is on the mat"],
    ["there is a cat on the mat"]
]
score = bleu.sentence_score(candidate, references)
print(f"BLEU 分数(多参考): {score.score:.4f}")
\end{lstlisting}

\textbf{计算案例}:

\textbf{案例1}:
\begin{itemize}
    \item 参考:\texttt{"the cat is on the mat"}
    \item 候选:\texttt{"the cat the cat on the mat"}
\end{itemize}

\textbf{逐步计算}:

\textbf{1-gram}:
\begin{itemize}
    \item 候选:\texttt{the(2), cat(2), the(2), cat(2), on(1), the(2), mat(1)}
    \item 参考:\texttt{the(2), cat(1), is(1), on(1), the(2), mat(1)}
    \item 截断计数:\texttt{the(2), cat(1), on(1), mat(1)} = 5
    \item 总数:7
    \item $P_1 = 5/7 = 0.7143$
\end{itemize}

\textbf{2-gram}:
\begin{itemize}
    \item 候选:\texttt{(the cat)(2), (cat the)(1), (the cat)(2), (cat on)(1), (on the)(1), (the mat)(1)}
    \item 参考:\texttt{(the cat)(1), (cat is)(1), (is on)(1), (on the)(1), (the mat)(1)}
    \item 截断计数:\texttt{(the cat)(1), (cat on)(0), (on the)(1), (the mat)(1)} = 3
    \item 总数:6
    \item $P_2 = 3/6 = 0.5000$
\end{itemize}

\textbf{简短惩罚}:
\begin{itemize}
    \item $c = 7$, $r = 6$
    \item $\text{BP} = e^{1-6/7} = e^{0.1429} = 1.1537$
\end{itemize}

\textbf{BLEU 分数}:
\begin{align}
\text{BLEU} &= \text{BP} \times \exp\left(\frac{1}{4}[\log P_1 + \log P_2 + \log P_3 + \log P_4]\right) \\
&= 1.1537 \times \exp\left(\frac{1}{4}[\log(0.7143) + \log(0.5) + \log(0) + \log(0)]\right) \\
&= 0 \quad \text{(因为 } P_3 = P_4 = 0\text{)}
\end{align}

\textbf{代码验证}:

\begin{lstlisting}[caption=案例1代码验证]
reference = ["the cat is on the mat"]
candidate = "the cat the cat on the mat"

bleu_score, precisions, bp = calculate_bleu(reference, candidate)
print(f"1-gram 精确度: {precisions[0]:.4f}")  # 0.7143
print(f"2-gram 精确度: {precisions[1]:.4f}")  # 0.5000
print(f"简短惩罚: {bp:.4f}")  # 1.1537
print(f"BLEU 分数: {bleu_score:.4f}")  # 0.0000(因为3-gram和4-gram为0)
\end{lstlisting}

\textbf{案例2}:更好的匹配

参考:\texttt{"the cat sat on the mat"} \\
候选:\texttt{"the cat sat on the mat"}

这是完美匹配,BLEU 分数应该接近 1.0。

\textbf{逐步计算}:

\textbf{1-gram}:
\begin{itemize}
    \item 候选:\texttt{the(2), cat(1), sat(1), on(1), the(2), mat(1)}
    \item 参考:\texttt{the(2), cat(1), sat(1), on(1), the(2), mat(1)}
    \item 截断计数:所有匹配 = 6
    \item 总数:6
    \item $P_1 = 6/6 = 1.0$
\end{itemize}

\textbf{2-gram}:
\begin{itemize}
    \item 候选:\texttt{(the cat)(1), (cat sat)(1), (sat on)(1), (on the)(1), (the mat)(1)}
    \item 参考:\texttt{(the cat)(1), (cat sat)(1), (sat on)(1), (on the)(1), (the mat)(1)}
    \item 截断计数:所有匹配 = 5
    \item 总数:5
    \item $P_2 = 5/5 = 1.0$
\end{itemize}

类似地,$P_3 = 1.0$,$P_4 = 1.0$。

\textbf{简短惩罚}:
\begin{itemize}
    \item $c = 6$, $r = 6$
    \item $\text{BP} = 1.0$(因为 $c = r$)
\end{itemize}

\textbf{BLEU 分数}:
\begin{align}
\text{BLEU} &= 1.0 \times \exp\left(\frac{1}{4}[\log(1.0) + \log(1.0) + \log(1.0) + \log(1.0)]\right) \\
&= 1.0 \times \exp(0) \\
&= 1.0
\end{align}

\textbf{代码验证}:

\begin{lstlisting}[caption=案例2代码验证]
reference = ["the cat sat on the mat"]
candidate = "the cat sat on the mat"

bleu_score, precisions, bp = calculate_bleu(reference, candidate)
print(f"所有 n-gram 精确度: {precisions}")  # [1.0, 1.0, 1.0, 1.0]
print(f"简短惩罚: {bp:.4f}")  # 1.0000
print(f"BLEU 分数: {bleu_score:.4f}")  # 1.0000
\end{lstlisting}

\textbf{案例3}:部分匹配

参考:\texttt{"the cat is sitting on the mat"} \\
候选:\texttt{"a cat sits on mat"}

\textbf{逐步计算}:

\textbf{1-gram}:
\begin{itemize}
    \item 候选:\texttt{a(1), cat(1), sits(1), on(1), mat(1)}
    \item 参考:\texttt{the(2), cat(1), is(1), sitting(1), on(1), the(2), mat(1)}
    \item 匹配:\texttt{cat(1), on(1), mat(1)} = 3
    \item 总数:5
    \item $P_1 = 3/5 = 0.6$
\end{itemize}

\textbf{简短惩罚}:
\begin{itemize}
    \item $c = 5$, $r = 7$
    \item $\text{BP} = e^{1-7/5} = e^{-0.4} = 0.6703$
\end{itemize}

由于 2-gram、3-gram、4-gram 匹配较少,最终 BLEU 分数较低。

\subsection{ROUGE 分数}

\textbf{概念解释}:ROUGE(Recall-Oriented Understudy for Gisting Evaluation)主要用于文本摘要评估,关注召回率。

\textbf{ROUGE-N}:

\begin{equation}
\text{ROUGE-N} = \frac{\sum_{S \in \text{References}} \sum_{\text{n-gram} \in S} \text{Count}_{\text{match}}(\text{n-gram})}{\sum_{S \in \text{References}} \sum_{\text{n-gram} \in S} \text{Count}(\text{n-gram})}
\end{equation}

\textbf{ROUGE-L}(最长公共子序列):

\begin{equation}
\text{ROUGE-L} = \frac{(1+\beta^2) R_{lcs} P_{lcs}}{R_{lcs} + \beta^2 P_{lcs}}
\end{equation}

其中:
\begin{align}
R_{lcs} &= \frac{LCS(X, Y)}{m} \quad \text{(召回率)} \\
P_{lcs} &= \frac{LCS(X, Y)}{n} \quad \text{(精确率)}
\end{align}

$LCS(X, Y)$ 是最长公共子序列长度,$m$ 是参考长度,$n$ 是候选长度。

\textbf{从零实现}:

\begin{lstlisting}[caption=ROUGE 分数从零实现]
from collections import Counter

def lcs_length(x, y):
    """计算最长公共子序列长度"""
    m, n = len(x), len(y)
    dp = [[0] * (n + 1) for _ in range(m + 1)]
    
    for i in range(1, m + 1):
        for j in range(1, n + 1):
            if x[i-1] == y[j-1]:
                dp[i][j] = dp[i-1][j-1] + 1
            else:
                dp[i][j] = max(dp[i-1][j], dp[i][j-1])
    
    return dp[m][n]

def rouge_n(reference, candidate, n=1):
    """计算 ROUGE-N"""
    ref_ngrams = Counter(get_ngrams(reference.split(), n))
    cand_ngrams = Counter(get_ngrams(candidate.split(), n))
    
    matches = sum(min(ref_ngrams[ngram], cand_ngrams[ngram]) 
                  for ngram in ref_ngrams)
    total = sum(ref_ngrams.values())
    
    return matches / total if total > 0 else 0.0

def rouge_l(reference, candidate, beta=1.2):
    """计算 ROUGE-L"""
    ref_tokens = reference.split()
    cand_tokens = candidate.split()
    
    lcs_len = lcs_length(ref_tokens, cand_tokens)
    
    if len(ref_tokens) == 0 or len(cand_tokens) == 0:
        return 0.0
    
    recall = lcs_len / len(ref_tokens)
    precision = lcs_len / len(cand_tokens)
    
    if recall + precision == 0:
        return 0.0
    
    f_score = (1 + beta**2) * recall * precision / (recall + beta**2 * precision)
    return f_score, recall, precision

# 测试
reference = "the cat is on the mat"
candidate = "the cat sat on the mat"

rouge_1 = rouge_n(reference, candidate, n=1)
rouge_2 = rouge_n(reference, candidate, n=2)
rouge_l_score, recall, precision = rouge_l(reference, candidate)

print(f"ROUGE-1: {rouge_1:.4f}")
print(f"ROUGE-2: {rouge_2:.4f}")
print(f"ROUGE-L: {rouge_l_score:.4f} (Recall: {recall:.4f}, Precision: {precision:.4f})")
\end{lstlisting}

\textbf{计算案例}:

\textbf{案例1}:
\begin{itemize}
    \item 参考:\texttt{"the cat is on the mat"}
    \item 候选:\texttt{"the cat sat on the mat"}
\end{itemize}

\textbf{ROUGE-1 计算}:
\begin{itemize}
    \item 参考 1-gram:\texttt{the(2), cat(1), is(1), on(1), mat(1)},共 6 个
    \item 候选 1-gram:\texttt{the(2), cat(1), sat(1), on(1), mat(1)},共 6 个
    \item 匹配:\texttt{the(2), cat(1), on(1), mat(1)} = 5
    \item ROUGE-1 = $5/6 = 0.8333$
\end{itemize}

\textbf{ROUGE-L 计算}:
\begin{itemize}
    \item 参考序列:\texttt{[the, cat, is, on, the, mat]}
    \item 候选序列:\texttt{[the, cat, sat, on, the, mat]}
    \item LCS:\texttt{[the, cat, on, the, mat]},长度为 5
    \item Recall = $5/6 = 0.8333$
    \item Precision = $5/6 = 0.8333$
    \item ROUGE-L = $\frac{2 \times 0.8333 \times 0.8333}{0.8333 + 0.8333} = 0.8333$
\end{itemize}

\textbf{代码验证}:

\begin{lstlisting}[caption=ROUGE 案例1代码验证]
reference = "the cat is on the mat"
candidate = "the cat sat on the mat"

rouge_1 = rouge_n(reference, candidate, n=1)
rouge_2 = rouge_n(reference, candidate, n=2)
rouge_l_score, recall, precision = rouge_l(reference, candidate)

print(f"ROUGE-1: {rouge_1:.4f}")  # 0.8333
print(f"ROUGE-2: {rouge_2:.4f}")
print(f"ROUGE-L: {rouge_l_score:.4f}")  # 0.8333
print(f"Recall: {recall:.4f}, Precision: {precision:.4f}")
\end{lstlisting}

\textbf{案例2}:
\begin{itemize}
    \item 参考:\texttt{"the cat is on the mat"}
    \item 候选:\texttt{"cat mat"}
\end{itemize}

\textbf{ROUGE-1 计算}:
\begin{itemize}
    \item 参考 1-gram:6 个
    \item 候选 1-gram:\texttt{cat(1), mat(1)},共 2 个
    \item 匹配:\texttt{cat(1), mat(1)} = 2
    \item ROUGE-1 = $2/6 = 0.3333$
\end{itemize}

\textbf{ROUGE-L 计算}:
\begin{itemize}
    \item LCS:\texttt{[cat, mat]},长度为 2
    \item Recall = $2/6 = 0.3333$
    \item Precision = $2/2 = 1.0$
    \item ROUGE-L = $\frac{2 \times 0.3333 \times 1.0}{0.3333 + 1.0} = 0.5000$
\end{itemize}

\subsection{METEOR 分数}

\textbf{概念解释}:METEOR 考虑同义词匹配,比 BLEU 更灵活。

\textbf{数学公式}:

\begin{equation}
\text{METEOR} = (1 - \text{Penalty}) \times \text{F}_{\text{mean}}
\end{equation}

其中:
\begin{align}
\text{F}_{\text{mean}} &= \frac{P \times R}{\alpha P + (1-\alpha) R} \\
\text{Penalty} &= 0.5 \times \left(\frac{\text{chunks}}{\text{unigrams\_matched}}\right)^3
\end{align}

\subsection{BERTScore}

\textbf{概念解释}:BERTScore 使用 BERT 嵌入计算语义相似度。

\textbf{数学公式}:

\begin{align}
\text{Precision} &= \frac{1}{|\hat{\mathbf{x}}|} \sum_{\hat{\mathbf{x}}_i \in \hat{\mathbf{x}}} \max_{\mathbf{x}_j \in \mathbf{x}} \hat{\mathbf{x}}_i^T \mathbf{x}_j \\
\text{Recall} &= \frac{1}{|\mathbf{x}|} \sum_{\mathbf{x}_j \in \mathbf{x}} \max_{\hat{\mathbf{x}}_i \in \hat{\mathbf{x}}} \hat{\mathbf{x}}_i^T \mathbf{x}_j \\
\text{F1} &= 2 \times \frac{\text{Precision} \times \text{Recall}}{\text{Precision} + \text{Recall}}
\end{align}

\section{评测基准与数据集}

\subsection{GLUE 和 SuperGLUE}

\textbf{GLUE}:General Language Understanding Evaluation,包含9个自然语言理解任务。

\textbf{SuperGLUE}:GLUE 的升级版,包含更具挑战性的任务。

\subsection{MMLU}

\textbf{概念解释}:Massive Multitask Language Understanding,包含57个任务,涵盖数学、物理、历史等多个领域。

\subsection{HumanEval 和 MBPP}

\textbf{HumanEval}:164个Python编程问题,评估代码生成能力。

\textbf{MBPP}:974个Python编程问题。

\subsection{MT-Bench 和 AlpacaEval}

\textbf{MT-Bench}:多轮对话评估基准。

\textbf{AlpacaEval}:指令遵循能力评估。

\section{QA Pair(问答对)}

\subsection{LoRA 相关问答}

\textbf{Q1: 什么是 LoRA?它的核心思想是什么?}

\textbf{A:} LoRA(Low-Rank Adaptation)是一种参数高效微调方法。核心思想是:对于预训练权重矩阵 $\mathbf{W}$,不直接更新它,而是学习一个低秩分解的增量 $\Delta\mathbf{W} = \mathbf{B}\mathbf{A}$,其中 $\mathbf{A} \in \mathbb{R}^{r \times k}$,$\mathbf{B} \in \mathbb{R}^{d \times r}$,$r \ll \min(d,k)$。这样只需要训练 $r(d+k)$ 个参数,而不是 $dk$ 个参数。

\textbf{Q2: LoRA 和全参数微调的区别是什么?}

\textbf{A:} 
\begin{itemize}
    \item \textbf{参数量}:LoRA 只更新 0.1-1\% 的参数,全参数微调更新 100\% 的参数
    \item \textbf{内存占用}:LoRA 内存占用大幅降低
    \item \textbf{训练速度}:LoRA 训练更快
    \item \textbf{效果}:LoRA 效果通常接近全参数微调(95-99\%)
    \item \textbf{灵活性}:LoRA 可以保存多个适配器,快速切换任务
\end{itemize}

\textbf{Q3: 如何选择 LoRA 的 rank?}

\textbf{A:} rank 的选择需要权衡:
\begin{itemize}
    \item \textbf{较小的 rank (4-8)}:参数量少,训练快,但可能表达能力不足
    \item \textbf{中等 rank (16-32)}:平衡性能和效率,适用于大多数任务
    \item \textbf{较大的 rank (64-128)}:表达能力更强,但参数量增加
    \item 建议从 $r=8$ 开始,根据效果调整
\end{itemize}

\subsection{评估指标相关问答}

\textbf{Q4: 如何计算 BLEU 分数?请给出详细步骤。}

\textbf{A:} BLEU 分数计算步骤:
\begin{enumerate}
    \item 计算各阶 n-gram(1-4)的精确度 $P_n$
    \item 计算简短惩罚 $\text{BP}$
    \item 计算几何平均:$\exp(\frac{1}{4}\sum_{n=1}^4 \log P_n)$
    \item 最终 BLEU = $\text{BP} \times \text{几何平均}$
\end{enumerate}

\textbf{Q5: 困惑度和 BLEU 的区别是什么?}

\textbf{A:}
\begin{itemize}
    \item \textbf{困惑度}:评估语言模型的预测不确定性,不需要参考文本,值越小越好
    \item \textbf{BLEU}:评估生成文本与参考文本的相似度,需要参考文本,值越大越好(0-1)
    \item \textbf{应用场景}:困惑度用于语言模型评估,BLEU 用于翻译和文本生成任务
\end{itemize}

\section{综合练习}

\subsection{概念理解题}

\begin{enumerate}
    \item 解释 LoRA 的数学原理,说明为什么低秩分解能够有效。
    \item 比较 QLoRA 和 LoRA 的区别,说明量化的作用。
    \item 解释 Flash Attention 如何减少内存占用。
    \item 说明 BLEU 分数的简短惩罚的作用。
    \item 解释 ROUGE-L 和 ROUGE-N 的区别。
\end{enumerate}

\subsection{计算题}

\begin{enumerate}
    \item \textbf{困惑度计算}:
    \begin{itemize}
        \item 给定序列长度为 100,平均对数概率为 -2.5,计算困惑度
        \item 手算并编写代码验证
    \end{itemize}
    
    \item \textbf{BLEU 计算}:
    \begin{itemize}
        \item 参考:\texttt{"the cat sat on the mat"}
        \item 候选:\texttt{"a cat sat on mat"}
        \item 计算 1-gram 到 4-gram 的精确度、简短惩罚和 BLEU 分数
        \item 手算并编写代码验证
    \end{itemize}
    
    \item \textbf{ROUGE 计算}:
    \begin{itemize}
        \item 参考:\texttt{"the cat is on the mat"}
        \item 候选:\texttt{"cat mat"}
        \item 计算 ROUGE-1、ROUGE-2 和 ROUGE-L
        \item 手算并编写代码验证
    \end{itemize}
\end{enumerate}

\subsection{代码实现题}

\begin{enumerate}
    \item 实现一个完整的 LoRA 训练脚本,包括数据加载、模型配置、训练循环和评估。
    \item 实现 BLEU 分数计算函数,支持多个参考文本。
    \item 实现 ROUGE 分数计算函数,包括 ROUGE-N 和 ROUGE-L。
    \item 实现困惑度计算函数,支持批量计算。
\end{enumerate}

\subsection{案例分析题}

\begin{enumerate}
    \item 给定一个文本生成任务,设计完整的评估方案,包括选择合适的评估指标、实现评估代码、分析结果。
    \item 分析 LoRA 在不同任务上的效果,比较不同 rank 设置的影响。
    \item 设计一个模型部署方案,包括模型压缩、服务化部署和性能优化。
\end{enumerate}

通过完成以上练习,读者可以深入理解大语言模型的先进技术和评估方法,掌握从模型微调到评估部署的完整流程。

