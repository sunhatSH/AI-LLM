\documentclass[12pt,a4paper]{article}
\usepackage[UTF8]{ctex}
\usepackage{amsmath}
\usepackage{amssymb}
\usepackage{tikz}
\usetikzlibrary{arrows,positioning,calc}
\usepackage{geometry}
\geometry{left=0cm,right=0cm,top=0cm,bottom=0cm}

\begin{document}
\thispagestyle{empty}
\pagestyle{empty}

\begin{center}
\vspace*{2cm}

{\Huge\bfseries AI/LLM 基础教程}\\[1cm]
{\Large 从数学基础到前沿应用}\\[2cm]

% 神经网络图
\begin{tikzpicture}[scale=1.2]
% 定义神经元间距
\def\spacing{1.5}

% 输入层 - 4个神经元,关于横轴对称
\node[circle, draw, minimum size=0.8cm, fill=blue!20, line width=1.5pt] (x1) at (0, -2.25) {$x_1$};
\node[circle, draw, minimum size=0.8cm, fill=blue!20, line width=1.5pt] (x2) at (0, -0.75) {$x_2$};
\node[circle, draw, minimum size=0.8cm, fill=blue!20, line width=1.5pt] (x3) at (0, 0.75) {$x_3$};
\node[circle, draw, minimum size=0.8cm, fill=blue!20, line width=1.5pt] (x4) at (0, 2.25) {$x_4$};

% 隐藏层1 - 8个神经元,关于横轴对称
\node[circle, draw, minimum size=0.7cm, fill=green!20, line width=1.5pt] (h11) at (4, -5.25) {$h_1^{(1)}$};
\node[circle, draw, minimum size=0.7cm, fill=green!20, line width=1.5pt] (h12) at (4, -3.75) {$h_1^{(2)}$};
\node[circle, draw, minimum size=0.7cm, fill=green!20, line width=1.5pt] (h13) at (4, -2.25) {$h_1^{(3)}$};
\node[circle, draw, minimum size=0.7cm, fill=green!20, line width=1.5pt] (h14) at (4, -0.75) {$h_1^{(4)}$};
\node[circle, draw, minimum size=0.7cm, fill=green!20, line width=1.5pt] (h15) at (4, 0.75) {$h_1^{(5)}$};
\node[circle, draw, minimum size=0.7cm, fill=green!20, line width=1.5pt] (h16) at (4, 2.25) {$h_1^{(6)}$};
\node[circle, draw, minimum size=0.7cm, fill=green!20, line width=1.5pt] (h17) at (4, 3.75) {$h_1^{(7)}$};
\node[circle, draw, minimum size=0.7cm, fill=green!20, line width=1.5pt] (h18) at (4, 5.25) {$h_1^{(8)}$};

% 隐藏层2 - 4个神经元,关于横轴对称
\node[circle, draw, minimum size=0.7cm, fill=orange!20, line width=1.5pt] (h21) at (8, -2.25) {$h_2^{(1)}$};
\node[circle, draw, minimum size=0.7cm, fill=orange!20, line width=1.5pt] (h22) at (8, -0.75) {$h_2^{(2)}$};
\node[circle, draw, minimum size=0.7cm, fill=orange!20, line width=1.5pt] (h23) at (8, 0.75) {$h_2^{(3)}$};
\node[circle, draw, minimum size=0.7cm, fill=orange!20, line width=1.5pt] (h24) at (8, 2.25) {$h_2^{(4)}$};

% 输出层 - 1个神经元,在横轴上
\node[circle, draw, minimum size=0.8cm, fill=red!20, line width=1.5pt] (y) at (12, 0) {$\hat{y}$};

% 输入层到隐藏层1的连接
\foreach \i in {1,...,4}
    \foreach \j in {1,...,8}
        \draw[->, gray!60, line width=0.5pt] (x\i) -- (h1\j);

% 隐藏层1到隐藏层2的连接
\foreach \i in {1,...,8}
    \foreach \j in {1,...,4}
        \draw[->, gray!60, line width=0.5pt] (h1\i) -- (h2\j);

% 隐藏层2到输出层的连接
\foreach \i in {1,...,4}
    \draw[->, gray!60, line width=0.5pt] (h2\i) -- (y);

% 标签
\node[above] at (0, 6.5) {\small\textbf{输入层}};
\node[above] at (4, 6.5) {\small\textbf{隐藏层1}};
\node[above] at (8, 6.5) {\small\textbf{隐藏层2}};
\node[above] at (12, 6.5) {\small\textbf{输出层}};
\end{tikzpicture}

\vspace{2cm}

{\large
\begin{tabular}{ll}
数学基础 & · 线性代数 · 概率论 · 优化理论 \\
Python编程 & · NumPy · Pandas · 数据处理 \\
机器学习 & · 监督学习 · 无监督学习 · 强化学习 \\
深度学习 & · 神经网络 · CNN · RNN · Transformer \\
大语言模型 & · Transformer · 注意力机制 · 预训练微调 \\
先进技术 & · LoRA · 推理加速 · 模型部署
\end{tabular}
}

\vfill

\end{center}

\end{document}

