\section{引言}

机器学习(Machine Learning, ML)是人工智能的核心分支,通过从数据中自动学习模式和规律,使计算机系统能够完成特定任务而无需显式编程。机器学习在图像识别、自然语言处理、推荐系统、医疗诊断、金融风控等领域取得了显著成功。

\textbf{机器学习的独特价值}:
\begin{itemize}
    \item \textbf{处理高维特征}:能够处理包含数千甚至数万个特征的数据,自动发现重要特征
    \item \textbf{捕捉非线性关系}:通过非线性模型(如神经网络、核方法)捕捉数据中的复杂模式
    \item \textbf{自适应学习}:能够从新数据中持续学习,适应数据分布的变化
    \item \textbf{端到端学习}:直接从原始数据学习到最终输出,减少人工特征工程的需求
\end{itemize}

本文档系统性地介绍机器学习的核心理论、经典算法和实际应用,涵盖监督学习、无监督学习、强化学习等主要范式,以及特征工程、模型评估、集成学习等重要技术。

\section{机器学习基础}

机器学习根据学习方式可以分为三大类:监督学习、无监督学习和强化学习。每种范式适用于不同类型的问题和场景。

\subsection{监督学习}

\begin{definition}[监督学习]
监督学习(Supervised Learning)是从带标签的训练数据中学习一个映射函数 $f: \mathcal{X} \to \mathcal{Y}$,使得对于新的输入 $\mathbf{x} \in \mathcal{X}$,能够预测其标签 $y \in \mathcal{Y}$。

给定训练数据集 $\mathcal{D} = \{(\mathbf{x}_1, y_1), (\mathbf{x}_2, y_2), \ldots, (\mathbf{x}_n, y_n)\}$,其中 $\mathbf{x}_i \in \mathbb{R}^d$ 是特征向量,$y_i$ 是对应的标签,目标是学习函数 $f$ 使得 $f(\mathbf{x}_i) \approx y_i$。
\end{definition}

\textbf{通俗解释}:监督学习就像有老师指导的学习。老师提供大量"题目-答案"对(训练数据),学生(模型)通过学习这些例子,掌握解题方法,然后能够解答新的题目。

\subsubsection{分类问题}

分类问题的目标是预测离散的类别标签。根据类别数量,可以分为:
\begin{itemize}
    \item \textbf{二分类}:只有两个类别,如垃圾邮件分类(垃圾/正常)、疾病诊断(患病/健康)
    \item \textbf{多分类}:有多个类别,如图像分类(猫/狗/鸟等)、手写数字识别(0-9)
\end{itemize}

\textbf{典型应用}:
\begin{example}[垃圾邮件分类]
\textbf{问题}:自动识别邮件是否为垃圾邮件。

\textbf{数据}:
\begin{itemize}
    \item 特征:邮件内容中的词频、发件人信息、邮件标题等
    \item 标签:垃圾邮件(1)或正常邮件(0)
\end{itemize}

\textbf{方法}:使用朴素贝叶斯、逻辑回归或支持向量机等分类算法。

\textbf{实际应用}:Gmail 使用机器学习模型过滤垃圾邮件,准确率超过 99\%。
\end{example}

\subsubsection{回归问题}

回归问题的目标是预测连续的数值。

\textbf{典型应用}:
\begin{example}[房价预测]
\textbf{问题}:根据房屋特征预测房价。

\textbf{数据}:
\begin{itemize}
    \item 特征:面积、卧室数、位置、建造年份等
    \item 标签:房价(连续值)
\end{itemize}

\textbf{方法}:使用线性回归、决策树回归或神经网络等算法。

\textbf{实际应用}:Zillow 的 Zestimate 使用机器学习模型预测房价,帮助用户了解房产价值。
\end{example}

\subsubsection{监督学习的优势与局限性}

\textbf{优势}:
\begin{itemize}
    \item 有明确的优化目标(最小化预测误差)
    \item 可以使用大量标注数据进行训练
    \item 模型性能可以通过测试集准确评估
    \item 理论基础相对完善
\end{itemize}

\textbf{局限性}:
\begin{itemize}
    \item 需要大量标注数据,标注成本高
    \item 对标注质量要求高,错误标注会影响模型性能
    \item 难以处理标注数据稀缺的场景
    \item 模型可能过拟合训练数据
\end{itemize}

\subsection{无监督学习}

\begin{definition}[无监督学习]
无监督学习(Unsupervised Learning)是从无标签的数据中学习数据的内在结构和模式,不需要人工标注。

给定数据集 $\mathcal{D} = \{\mathbf{x}_1, \mathbf{x}_2, \ldots, \mathbf{x}_n\}$,其中只有特征向量,没有标签,目标是发现数据中的隐藏模式、结构或分布。
\end{definition}

\textbf{通俗解释}:无监督学习就像没有老师指导的自学。学生只能看到大量数据,需要自己发现其中的规律和模式,比如哪些数据相似、数据如何分组等。

\subsubsection{聚类}

聚类是将相似的数据点分组的过程。

\textbf{典型应用}:
\begin{example}[客户细分]
\textbf{问题}:根据客户的购买行为将其分为不同的群体。

\textbf{数据}:客户的购买历史、浏览记录、消费金额等特征(无标签)。

\textbf{方法}:使用 K-means、层次聚类等算法将客户分为若干群体。

\textbf{实际应用}:电商平台使用聚类分析进行客户细分,为不同群体提供个性化推荐和营销策略。
\end{example}

\subsubsection{降维}

降维是将高维数据映射到低维空间,保留主要信息。

\textbf{典型应用}:
\begin{example}[数据可视化]
\textbf{问题}:将高维数据(如 1000 维)可视化到 2D 或 3D 空间。

\textbf{方法}:使用 PCA(主成分分析)、t-SNE、UMAP 等降维算法。

\textbf{实际应用}:在生物信息学中,使用 t-SNE 将基因表达数据降维到 2D,可视化不同细胞类型的分布。
\end{example}

\subsubsection{无监督学习的优势与局限性}

\textbf{优势}:
\begin{itemize}
    \item 不需要标注数据,数据获取成本低
    \item 可以发现人类未预见的模式和结构
    \item 适用于探索性数据分析
    \item 可以作为监督学习的预处理步骤
\end{itemize}

\textbf{局限性}:
\begin{itemize}
    \item 没有明确的优化目标,评估困难
    \item 结果可能难以解释和验证
    \item 对参数选择敏感(如聚类数量)
    \item 可能发现无意义的模式
\end{itemize}

\subsection{强化学习}

\begin{definition}[强化学习]
强化学习(Reinforcement Learning, RL)是智能体(Agent)通过与环境的交互来学习最优策略,通过试错和奖励信号来指导学习过程。

强化学习问题可以建模为马尔可夫决策过程(MDP):$(\mathcal{S}, \mathcal{A}, \mathcal{P}, \mathcal{R}, \gamma)$,其中:
\begin{itemize}
    \item $\mathcal{S}$:状态空间
    \item $\mathcal{A}$:动作空间
    \item $\mathcal{P}$:状态转移概率
    \item $\mathcal{R}$:奖励函数
    \item $\gamma$:折扣因子
\end{itemize}

智能体的目标是学习策略 $\pi: \mathcal{S} \to \mathcal{A}$,最大化累积奖励:
$$R = \sum_{t=0}^{\infty} \gamma^t r_t$$
\end{definition}

\textbf{通俗解释}:强化学习就像训练宠物。宠物(智能体)做出动作(如坐下),主人(环境)给出奖励(如给食物)或惩罚(如不给食物)。通过反复尝试,宠物学会在什么情况下做什么动作能获得最多奖励。

\subsubsection{强化学习的核心概念}

\begin{itemize}
    \item \textbf{状态(State)}:环境在某个时刻的完整描述
    \item \textbf{动作(Action)}:智能体可以执行的操作
    \item \textbf{奖励(Reward)}:环境对智能体动作的反馈信号
    \item \textbf{策略(Policy)}:从状态到动作的映射
    \item \textbf{价值函数(Value Function)}:评估状态或动作的长期价值
\end{itemize}

\subsubsection{典型应用}

\begin{example}[游戏 AI]
\textbf{问题}:训练 AI 在游戏中达到人类或超人类水平。

\textbf{方法}:
\begin{itemize}
    \item Deep Q-Network (DQN):使用深度神经网络近似 Q 函数
    \item Policy Gradient:直接优化策略函数
    \item Actor-Critic:结合价值函数和策略梯度
\end{itemize}

\textbf{实际应用}:
\begin{itemize}
    \item \textbf{AlphaGo}:DeepMind 开发的围棋 AI,2016 年击败世界冠军李世石
    \item \textbf{AlphaStar}:在《星际争霸 II》中达到大师级水平
    \item \textbf{OpenAI Five}:在 Dota 2 中击败职业战队
\end{itemize}
\end{example}

\begin{example}[机器人控制]
\textbf{问题}:训练机器人完成复杂任务,如抓取物体、行走等。

\textbf{方法}:使用强化学习让机器人在仿真或真实环境中通过试错学习。

\textbf{实际应用}:Boston Dynamics 的机器人使用强化学习进行运动控制,实现复杂的平衡和移动能力。
\end{example}

\subsubsection{强化学习的优势与局限性}

\textbf{优势}:
\begin{itemize}
    \item 适用于序列决策问题
    \item 可以通过试错学习,不需要大量标注数据
    \item 能够学习长期策略
    \item 适用于动态环境
\end{itemize}

\textbf{局限性}:
\begin{itemize}
    \item 训练过程可能很慢,需要大量交互
    \item 奖励函数设计困难
    \item 探索与利用的平衡问题
    \item 安全性问题(在关键应用中)
\end{itemize}

\section{经典算法}

本节介绍机器学习中的经典算法,包括线性回归、决策树、随机森林和支持向量机。这些算法虽然相对简单,但在许多实际问题中仍然非常有效。

\subsection{线性回归}

线性回归是监督学习中最基础的算法之一,用于解决回归问题。

\begin{definition}[线性回归]
线性回归假设目标变量 $y$ 与特征向量 $\mathbf{x}$ 之间存在线性关系:
$$y = \mathbf{w}^T\mathbf{x} + b + \epsilon$$
其中 $\mathbf{w} \in \mathbb{R}^d$ 是权重向量,$b \in \mathbb{R}$ 是偏置项,$\epsilon$ 是误差项(通常假设 $\epsilon \sim \mathcal{N}(0, \sigma^2)$)。

给定训练数据 $\{(\mathbf{x}_i, y_i)\}_{i=1}^{n}$,线性回归的目标是找到参数 $\mathbf{w}$ 和 $b$,使得预测误差最小。
\end{definition}

\textbf{通俗解释}:线性回归就像用一条直线去拟合数据点。例如,用一条直线拟合"房屋面积-房价"的关系,直线的斜率和截距就是学到的参数。

\subsubsection{最小二乘法}

最常用的方法是最小二乘法(Least Squares),最小化平方误差:
$$L(\mathbf{w}, b) = \sum_{i=1}^{n} (y_i - (\mathbf{w}^T\mathbf{x}_i + b))^2$$

通过求导并令导数为零,可以得到闭式解:
$$\mathbf{w}^* = (\mathbf{X}^T\mathbf{X})^{-1}\mathbf{X}^T\mathbf{y}$$

其中 $\mathbf{X} \in \mathbb{R}^{n \times d}$ 是特征矩阵,$\mathbf{y} \in \mathbb{R}^n$ 是标签向量。

\subsubsection{正则化}

为了防止过拟合,可以加入正则化项:

\begin{itemize}
    \item \textbf{Ridge 回归($L_2$ 正则化)}:
    $$L_{\text{Ridge}} = \sum_{i=1}^{n} (y_i - \mathbf{w}^T\mathbf{x}_i)^2 + \lambda \|\mathbf{w}\|_2^2$$
    其中 $\lambda > 0$ 是正则化系数。Ridge 回归倾向于产生较小的权重。
    
    \item \textbf{Lasso 回归($L_1$ 正则化)}:
    $$L_{\text{Lasso}} = \sum_{i=1}^{n} (y_i - \mathbf{w}^T\mathbf{x}_i)^2 + \lambda \|\mathbf{w}\|_1$$
    Lasso 回归可以产生稀疏解(许多权重为 0),实现特征选择。
\end{itemize}

\subsubsection{应用场景}

\begin{example}[房价预测]
使用线性回归预测房价,特征包括面积、卧室数、位置等。Ridge 回归可以防止过拟合,Lasso 回归可以自动选择重要特征。
\end{example}

\begin{example}[股票价格预测]
使用历史价格、交易量等特征预测未来股价。虽然股价受多种因素影响,线性回归可以作为基准模型。
\end{example}

\subsubsection{优势与局限性}

\textbf{优势}:
\begin{itemize}
    \item 简单易懂,计算效率高
    \item 有闭式解,训练快速
    \item 可解释性强(权重表示特征重要性)
    \item 适合作为基准模型
\end{itemize}

\textbf{局限性}:
\begin{itemize}
    \item 只能捕捉线性关系
    \item 对异常值敏感
    \item 假设特征之间相互独立
    \item 需要特征工程(如多项式特征)来捕捉非线性
\end{itemize}

\subsection{决策树}

决策树是一种基于树结构的分类和回归算法,通过一系列规则进行决策。

\begin{definition}[决策树]
决策树(Decision Tree)是一个树形结构,其中:
\begin{itemize}
    \item 内部节点:表示特征测试
    \item 分支:表示测试结果
    \item 叶节点:表示类别标签或数值
\end{itemize}

从根节点到叶节点的路径对应一条决策规则。
\end{definition}

\textbf{通俗解释}:决策树就像医生诊断疾病的流程。先问"是否发烧?",如果"是"再问"是否咳嗽?",根据一系列问题的答案,最终得出诊断结果。

\subsubsection{构建决策树}

决策树的构建是一个递归过程:

\begin{algorithm}
\caption{构建决策树}
\begin{algorithmic}
\REQUIRE 训练数据集 $\mathcal{D}$,特征集 $\mathcal{F}$
\ENSURE 决策树 $T$
\IF{$\mathcal{D}$ 中所有样本属于同一类别}
    \STATE 返回叶节点,标记为该类别
\ELSIF{$\mathcal{F}$ 为空或 $\mathcal{D}$ 为空}
    \STATE 返回叶节点,标记为 $\mathcal{D}$ 中多数类
\ELSE
    \STATE 选择最优特征 $f^* = \arg\max_{f \in \mathcal{F}} \text{IG}(\mathcal{D}, f)$
    \STATE 为 $f^*$ 的每个可能值创建分支
    \FOR{每个分支}
        \STATE $D_v$ = $\mathcal{D}$ 中 $f^* = v$ 的样本子集
        \IF{$D_v$ 为空}
            \STATE 添加叶节点,标记为 $\mathcal{D}$ 中多数类
        \ELSE
            \STATE 递归构建子树:$\text{BuildTree}(D_v, \mathcal{F} \setminus \{f^*\})$
        \ENDIF
    \ENDFOR
\ENDIF
\end{algorithmic}
\end{algorithm}

\subsubsection{特征选择准则}

常用的特征选择准则包括:

\begin{itemize}
    \item \textbf{信息增益(Information Gain)}:
    $$\text{IG}(\mathcal{D}, f) = H(\mathcal{D}) - \sum_{v} \frac{|\mathcal{D}_v|}{|\mathcal{D}|} H(\mathcal{D}_v)$$
    其中 $H(\mathcal{D})$ 是数据集的熵。
    
    \item \textbf{基尼不纯度(Gini Impurity)}:
    $$G(\mathcal{D}) = 1 - \sum_{k} p_k^2$$
    其中 $p_k$ 是类别 $k$ 在数据集中的比例。
    
    \item \textbf{基尼增益}:
    $$\text{GiniGain}(\mathcal{D}, f) = G(\mathcal{D}) - \sum_{v} \frac{|\mathcal{D}_v|}{|\mathcal{D}|} G(\mathcal{D}_v)$$
\end{itemize}

\subsubsection{剪枝}

为了防止过拟合,需要对决策树进行剪枝:

\begin{itemize}
    \item \textbf{预剪枝}:在构建过程中提前停止(如限制树深度、最小样本数)
    \item \textbf{后剪枝}:构建完整树后,自底向上删除节点,用叶节点替代
\end{itemize}

\subsubsection{应用场景}

\begin{example}[医疗诊断]
使用决策树根据症状(发烧、咳嗽、头痛等)诊断疾病。决策树的可解释性使得医生可以理解诊断依据。
\end{example}

\begin{example}[信用评估]
银行使用决策树评估贷款申请人的信用风险,根据收入、工作年限、信用历史等特征做出决策。
\end{example}

\subsubsection{优势与局限性}

\textbf{优势}:
\begin{itemize}
    \item 可解释性强,决策过程清晰
    \item 可以处理数值和类别特征
    \item 不需要特征缩放
    \item 可以捕捉非线性关系
\end{itemize}

\textbf{局限性}:
\begin{itemize}
    \item 容易过拟合
    \item 对数据的小变化敏感(不稳定)
    \item 倾向于选择具有更多取值的特征
    \item 难以处理特征之间的交互
\end{itemize}

\subsection{随机森林}

随机森林(Random Forest)是决策树的集成方法,通过组合多个决策树来提高性能。

\begin{definition}[随机森林]
随机森林由 $B$ 棵决策树组成,每棵树在训练时:
\begin{enumerate}
    \item 使用自助采样(Bootstrap Sampling)从训练集中采样
    \item 在每个节点分裂时,随机选择 $k$ 个特征(通常 $k = \sqrt{d}$)进行考虑
\end{enumerate}

预测时,对于分类问题使用投票,对于回归问题使用平均。
\end{definition}

\textbf{通俗解释}:随机森林就像一群专家投票做决策。每个专家(决策树)根据自己的经验(不同的训练数据)给出意见,最终综合所有专家的意见做出决策。这样比单个专家更可靠。

\subsubsection{算法流程}

\begin{algorithm}
\caption{随机森林训练}
\begin{algorithmic}
\REQUIRE 训练数据集 $\mathcal{D}$,树的数量 $B$,特征采样数 $k$
\ENSURE 随机森林 $\{T_1, T_2, \ldots, T_B\}$
\FOR{$b = 1$ to $B$}
    \STATE 使用自助采样从 $\mathcal{D}$ 中采样得到 $\mathcal{D}_b$
    \STATE 使用 $\mathcal{D}_b$ 训练决策树 $T_b$,在每个节点随机选择 $k$ 个特征
\ENDFOR
\end{algorithmic}
\end{algorithm}

\subsubsection{随机性的作用}

随机森林通过两种随机性提高性能:
\begin{itemize}
    \item \textbf{样本随机性}:每棵树使用不同的训练样本(自助采样)
    \item \textbf{特征随机性}:每棵树在每个节点考虑不同的特征子集
\end{itemize}

这种随机性使得各棵树之间具有多样性,减少过拟合,提高泛化能力。

\subsubsection{应用场景}

\begin{example}[图像分类]
在 CIFAR-10 数据集上,随机森林可以作为基准模型,虽然不如深度学习模型,但训练速度快,可解释性强。
\end{example}

\begin{example}[特征重要性分析]
随机森林可以计算特征重要性,帮助理解哪些特征对预测最重要。这在特征工程和模型解释中很有价值。
\end{example}

\subsubsection{优势与局限性}

\textbf{优势}:
\begin{itemize}
    \item 性能通常优于单棵决策树
    \item 对过拟合有较强的抵抗力
    \item 可以处理高维特征
    \item 可以计算特征重要性
    \item 训练可以并行化
\end{itemize}

\textbf{局限性}:
\begin{itemize}
    \item 模型可解释性不如单棵决策树
    \item 需要更多内存和计算资源
    \item 对于某些问题,可能不如梯度提升方法(如 XGBoost)
\end{itemize}

\subsection{支持向量机}

支持向量机(Support Vector Machine, SVM)是一种强大的分类算法,基于最大间隔原理。

\begin{definition}[支持向量机]
对于线性可分的二分类问题,SVM 寻找一个超平面 $\mathbf{w}^T\mathbf{x} + b = 0$,使得两类样本之间的间隔(margin)最大。

间隔定义为两类样本到超平面的最小距离:
$$\text{margin} = \frac{2}{\|\mathbf{w}\|}$$

SVM 的优化目标是:
$$\min_{\mathbf{w}, b} \frac{1}{2}\|\mathbf{w}\|^2 \quad \text{s.t.} \quad y_i(\mathbf{w}^T\mathbf{x}_i + b) \geq 1, \forall i$$
\end{definition}

\textbf{通俗解释}:SVM 就像在两个群体之间画一条"最宽的路"。这条路要尽可能宽,同时要确保两边的群体都在路的正确一侧。支持向量就是那些"站在路边"的样本点。

\subsubsection{软间隔 SVM}

对于线性不可分的情况,引入松弛变量 $\xi_i$,允许一些样本分类错误:

$$\min_{\mathbf{w}, b, \boldsymbol{\xi}} \frac{1}{2}\|\mathbf{w}\|^2 + C\sum_{i=1}^{n}\xi_i$$

约束条件:
$$y_i(\mathbf{w}^T\mathbf{x}_i + b) \geq 1 - \xi_i, \quad \xi_i \geq 0$$

其中 $C > 0$ 是惩罚参数,控制对误分类的惩罚程度。

\subsubsection{核技巧}

对于非线性问题,使用核函数将数据映射到高维空间,在高维空间中线性可分:

$$K(\mathbf{x}_i, \mathbf{x}_j) = \phi(\mathbf{x}_i)^T\phi(\mathbf{x}_j)$$

常用的核函数包括:
\begin{itemize}
    \item \textbf{多项式核}:$K(\mathbf{x}_i, \mathbf{x}_j) = (\mathbf{x}_i^T\mathbf{x}_j + 1)^d$
    \item \textbf{径向基函数(RBF)核}:$K(\mathbf{x}_i, \mathbf{x}_j) = \exp(-\gamma\|\mathbf{x}_i - \mathbf{x}_j\|^2)$
    \item \textbf{Sigmoid 核}:$K(\mathbf{x}_i, \mathbf{x}_j) = \tanh(\alpha\mathbf{x}_i^T\mathbf{x}_j + c)$
\end{itemize}

\subsubsection{对偶形式}

SVM 的对偶形式为:
$$\max_{\boldsymbol{\alpha}} \sum_{i=1}^{n}\alpha_i - \frac{1}{2}\sum_{i,j}\alpha_i\alpha_j y_i y_j K(\mathbf{x}_i, \mathbf{x}_j)$$

约束条件:
$$\sum_{i=1}^{n}\alpha_i y_i = 0, \quad 0 \leq \alpha_i \leq C$$

对偶形式的优势:
\begin{itemize}
    \item 只需要计算核函数,不需要显式映射到高维空间
    \item 支持向量($\alpha_i > 0$)的数量通常远小于样本数
    \item 更容易扩展到大规模问题
\end{itemize}

\subsubsection{应用场景}

\begin{example}[文本分类]
SVM 在文本分类任务中表现优异,特别是在小样本情况下。使用 TF-IDF 特征和 RBF 核,可以在新闻分类、情感分析等任务中取得良好效果。
\end{example}

\begin{example}[图像分类]
在图像分类任务中,SVM 可以作为特征分类器。例如,使用 CNN 提取特征,然后用 SVM 进行分类,这在某些情况下比端到端的 CNN 更有效。
\end{example}

\subsubsection{优势与局限性}

\textbf{优势}:
\begin{itemize}
    \item 在中小规模数据集上表现优异
    \item 通过核函数可以处理非线性问题
    \item 理论基础完善(基于统计学习理论)
    \item 对过拟合有较好的控制
    \item 支持向量提供了模型的稀疏表示
\end{itemize}

\textbf{局限性}:
\begin{itemize}
    \item 对大规模数据集计算成本高
    \item 对特征缩放敏感
    \item 核函数和参数选择需要经验
    \item 可解释性不如决策树
    \item 概率输出需要额外处理(Platt scaling)
\end{itemize}

\section{特征工程}

特征工程是机器学习中至关重要的一环,好的特征可以显著提升模型性能。特征工程包括特征选择、特征变换和特征编码等。

\subsection{特征选择}

特征选择是从原始特征中选择最有用的特征子集,减少维度,提高模型性能和可解释性。

\subsubsection{过滤方法(Filter Methods)}

过滤方法基于特征的统计特性进行选择,独立于具体的学习算法:

\begin{itemize}
    \item \textbf{方差选择}:移除方差很小的特征(几乎不变的特征)
    \item \textbf{相关系数}:选择与目标变量相关性高的特征
    \item \textbf{互信息}:选择与目标变量互信息大的特征
    \item \textbf{卡方检验}:用于分类问题,检验特征与标签的独立性
\end{itemize}

\subsubsection{包装方法(Wrapper Methods)}

包装方法使用学习算法来评估特征子集:

\begin{itemize}
    \item \textbf{前向选择}:从空集开始,逐步添加最有用的特征
    \item \textbf{后向消除}:从完整特征集开始,逐步移除最无用的特征
    \item \textbf{递归特征消除(RFE)}:递归地移除最不重要的特征
\end{itemize}

\subsubsection{嵌入方法(Embedded Methods)}

嵌入方法在模型训练过程中进行特征选择:

\begin{itemize}
    \item \textbf{Lasso 回归}:$L_1$ 正则化自动产生稀疏解
    \item \textbf{决策树}:通过特征重要性进行选择
    \item \textbf{随机森林}:通过特征重要性排序
\end{itemize}

\subsection{特征变换}

特征变换是对特征进行数学变换,改变特征的分布或关系。

\subsubsection{标准化和归一化}

\begin{itemize}
    \item \textbf{标准化(Z-score)}:
    $$z = \frac{x - \mu}{\sigma}$$
    将特征转换为均值为 0、标准差为 1 的分布。
    
    \item \textbf{最小-最大归一化}:
    $$x' = \frac{x - x_{\min}}{x_{\max} - x_{\min}}$$
    将特征缩放到 $[0, 1]$ 区间。
    
    \item \textbf{Robust 缩放}:使用中位数和四分位距,对异常值更鲁棒。
\end{itemize}

\textbf{为什么需要特征缩放}:
\begin{itemize}
    \item 许多算法(如 SVM、K-means、神经网络)对特征尺度敏感
    \item 梯度下降算法在特征尺度不一致时收敛慢
    \item 距离-based 算法(如 KNN)受特征尺度影响大
\end{itemize}

\subsubsection{多项式特征}

通过创建特征的多项式组合来捕捉非线性关系:

$$(x_1, x_2) \to (x_1, x_2, x_1^2, x_1 x_2, x_2^2)$$

\textbf{应用场景}:线性回归无法捕捉非线性关系时,可以使用多项式特征。

\subsubsection{对数变换}

对偏态分布进行对数变换,使其更接近正态分布:

$$x' = \log(x + 1)$$

\textbf{应用场景}:处理价格、收入等右偏分布的数据。

\subsection{特征编码}

特征编码是将类别特征转换为数值特征的过程。

\subsubsection{独热编码(One-Hot Encoding)}

将类别特征转换为二进制向量:

\begin{example}
颜色特征:["红", "绿", "蓝"] $\to$ 
\begin{itemize}
    \item 红:$[1, 0, 0]$
    \item 绿:$[0, 1, 0]$
    \item 蓝:$[0, 0, 1]$
\end{itemize}
\end{example}

\textbf{优势}:不引入类别间的顺序关系。

\textbf{局限性}:类别数量多时会产生高维稀疏特征。

\subsubsection{标签编码(Label Encoding)}

将类别映射为整数:

\begin{example}
["低", "中", "高"] $\to$ [0, 1, 2]
\end{example}

\textbf{注意}:只适用于有序类别,否则会引入虚假的顺序关系。

\subsubsection{目标编码(Target Encoding)}

使用目标变量的统计量对类别进行编码:

$$x' = \frac{\sum_{i: x_i = x} y_i}{|\{i: x_i = x\}|}$$

即用该类别的平均目标值作为编码。

\textbf{优势}:可以捕捉类别与目标的关系。

\textbf{注意}:需要防止过拟合(如使用交叉验证)。

\section{模型评估与验证}

模型评估是机器学习流程中的关键步骤,用于衡量模型性能、选择最佳模型和防止过拟合。

\subsection{评估指标}

\subsubsection{分类问题指标}

\begin{itemize}
    \item \textbf{准确率(Accuracy)}:
    $$\text{Accuracy} = \frac{\text{正确预测数}}{\text{总样本数}}$$
    
    \item \textbf{精确率(Precision)}:
    $$\text{Precision} = \frac{\text{TP}}{\text{TP} + \text{FP}}$$
    预测为正例中真正为正例的比例。
    
    \item \textbf{召回率(Recall)}:
    $$\text{Recall} = \frac{\text{TP}}{\text{TP} + \text{FN}}$$
    真正例中被正确预测的比例。
    
    \item \textbf{F1 分数}:
    $$\text{F1} = \frac{2 \times \text{Precision} \times \text{Recall}}{\text{Precision} + \text{Recall}}$$
    精确率和召回率的调和平均。
    
    \item \textbf{ROC 曲线和 AUC}:ROC 曲线以假正例率为横轴,真正例率为纵轴。AUC(曲线下面积)衡量分类器的整体性能。
\end{itemize}

\subsubsection{回归问题指标}

\begin{itemize}
    \item \textbf{均方误差(MSE)}:
    $$\text{MSE} = \frac{1}{n}\sum_{i=1}^{n}(y_i - \hat{y}_i)^2$$
    
    \item \textbf{均方根误差(RMSE)}:
    $$\text{RMSE} = \sqrt{\text{MSE}}$$
    
    \item \textbf{平均绝对误差(MAE)}:
    $$\text{MAE} = \frac{1}{n}\sum_{i=1}^{n}|y_i - \hat{y}_i|$$
    
    \item \textbf{决定系数($R^2$)}:
    $$R^2 = 1 - \frac{\sum_{i=1}^{n}(y_i - \hat{y}_i)^2}{\sum_{i=1}^{n}(y_i - \bar{y})^2}$$
    衡量模型解释的方差比例。
\end{itemize}

\subsection{交叉验证}

交叉验证是评估模型泛化能力的重要方法。

\subsubsection{K 折交叉验证}

将数据集分为 $K$ 折,每次使用 $K-1$ 折训练,剩余 1 折测试,重复 $K$ 次:

\begin{algorithm}
\caption{K 折交叉验证}
\begin{algorithmic}
\REQUIRE 数据集 $\mathcal{D}$,折数 $K$,学习算法 $\mathcal{A}$
\ENSURE 平均性能指标
\STATE 将 $\mathcal{D}$ 随机分为 $K$ 折:$\mathcal{D}_1, \ldots, \mathcal{D}_K$
\FOR{$k = 1$ to $K$}
    \STATE 训练集:$\mathcal{D}_{\text{train}} = \mathcal{D} \setminus \mathcal{D}_k$
    \STATE 测试集:$\mathcal{D}_{\text{test}} = \mathcal{D}_k$
    \STATE 使用 $\mathcal{D}_{\text{train}}$ 训练模型 $M_k = \mathcal{A}(\mathcal{D}_{\text{train}})$
    \STATE 在 $\mathcal{D}_{\text{test}}$ 上评估性能 $s_k$
\ENDFOR
\RETURN $\bar{s} = \frac{1}{K}\sum_{k=1}^{K} s_k$
\end{algorithmic}
\end{algorithm}

\textbf{优势}:
\begin{itemize}
    \item 充分利用数据
    \item 提供性能估计的方差
    \item 减少对数据划分的依赖
\end{itemize}

\textbf{常见选择}:$K = 5$ 或 $K = 10$。

\subsubsection{留一法交叉验证(LOOCV)}

$K = n$ 的特殊情况,每次留一个样本作为测试集。计算成本高,但无偏估计。

\subsection{过拟合与欠拟合}

\begin{definition}[过拟合]
过拟合(Overfitting)是指模型在训练集上表现很好,但在测试集上表现较差的现象。模型过度学习了训练数据的噪声和细节,导致泛化能力差。
\end{definition}

\begin{definition}[欠拟合]
欠拟合(Underfitting)是指模型在训练集和测试集上都表现较差的现象。模型过于简单,无法捕捉数据中的基本模式。
\end{definition}

\textbf{通俗解释}:
\begin{itemize}
    \item \textbf{过拟合}:就像学生死记硬背了所有练习题,但遇到新题目就不会了
    \item \textbf{欠拟合}:就像学生只学了基础知识,连练习题都做不好
\end{itemize}

\subsubsection{识别过拟合和欠拟合}

\begin{itemize}
    \item \textbf{过拟合的迹象}:
    \begin{itemize}
        \item 训练误差很小,但验证误差很大
        \item 模型复杂度高(如深度很深的决策树)
        \item 训练集和验证集性能差距大
    \end{itemize}
    
    \item \textbf{欠拟合的迹象}:
    \begin{itemize}
        \item 训练误差和验证误差都很大
        \item 模型复杂度低(如线性模型处理非线性问题)
        \item 模型无法捕捉数据的基本模式
    \end{itemize}
\end{itemize}

\subsubsection{解决方法}

\textbf{解决过拟合}:
\begin{itemize}
    \item 增加训练数据
    \item 减少模型复杂度
    \item 正则化($L_1$、$L_2$)
    \item Dropout(神经网络)
    \item 早停(Early Stopping)
\end{itemize}

\textbf{解决欠拟合}:
\begin{itemize}
    \item 增加模型复杂度
    \item 增加特征
    \item 减少正则化
    \item 增加训练时间
\end{itemize}

\subsection{正则化}

正则化(Regularization)是防止过拟合的核心技术,通过在损失函数中添加惩罚项来约束模型参数,从而控制模型复杂度,提高泛化能力。

\begin{definition}[正则化]
正则化通过在原始损失函数中添加正则化项 $\Omega(\theta)$ 来约束模型参数:

$$L_{\text{reg}}(\theta) = L(\theta) + \lambda \Omega(\theta)$$

其中:
\begin{itemize}
    \item $L(\theta)$ 是原始损失函数(如均方误差、交叉熵)
    \item $\Omega(\theta)$ 是正则化项,衡量模型复杂度
    \item $\lambda > 0$ 是正则化系数,控制正则化强度
\end{itemize}
\end{definition}

\textbf{通俗解释}:正则化就像给模型"戴紧箍咒"。想象一个学生准备考试:
\begin{itemize}
    \item \textbf{没有正则化}:学生可能死记硬背所有题目,但遇到新题就不会了(过拟合)
    \item \textbf{有正则化}:限制学生只能记住重点知识,虽然训练题可能答得不够完美,但新题答得更好(泛化能力更强)
\end{itemize}

\textbf{正则化的核心思想}:
\begin{itemize}
    \item \textbf{奥卡姆剃刀原理}:在同样能解释数据的模型中,选择最简单的
    \item \textbf{约束参数空间}:限制模型参数的取值范围,防止参数过大
    \item \textbf{降低模型复杂度}:减少模型对训练数据的过度拟合
\end{itemize}

\subsubsection{$L_2$ 正则化(Ridge 回归)}

$L_2$ 正则化通过在损失函数中添加权重的平方和来约束参数。

\begin{definition}[$L_2$ 正则化]
对于线性模型 $f(\mathbf{x}) = \mathbf{w}^T\mathbf{x} + b$,$L_2$ 正则化的损失函数为:

$$L_{\text{Ridge}}(\mathbf{w}, b) = \sum_{i=1}^{n} (y_i - (\mathbf{w}^T\mathbf{x}_i + b))^2 + \lambda \|\mathbf{w}\|_2^2$$

其中 $\|\mathbf{w}\|_2^2 = \sum_{j=1}^{d} w_j^2$ 是权重的 $L_2$ 范数的平方。
\end{definition}

\textbf{数学推导}:

对权重求偏导并令其为零:

$$\frac{\partial L_{\text{Ridge}}}{\partial \mathbf{w}} = -2\mathbf{X}^T(\mathbf{y} - \mathbf{X}\mathbf{w}) + 2\lambda \mathbf{w} = 0$$

整理得到闭式解:

$$\mathbf{w}^* = (\mathbf{X}^T\mathbf{X} + \lambda \mathbf{I})^{-1}\mathbf{X}^T\mathbf{y}$$

其中 $\mathbf{I}$ 是单位矩阵。注意:$\lambda \mathbf{I}$ 使得矩阵 $(\mathbf{X}^T\mathbf{X} + \lambda \mathbf{I})$ 总是可逆的(即使 $\mathbf{X}^T\mathbf{X}$ 不可逆)。

\textbf{几何解释}:

$L_2$ 正则化等价于在参数空间中寻找一个点,使得:
\begin{itemize}
    \item 该点尽可能接近最优解(最小化原始损失)
    \item 该点的 $L_2$ 范数尽可能小(参数尽可能接近原点)
\end{itemize}

这相当于在参数空间中寻找一个"平衡点",既拟合数据又保持参数较小。

\textbf{特点}:
\begin{itemize}
    \item \textbf{平滑收缩}:所有参数按比例缩小,不会完全变为零
    \item \textbf{数值稳定}:即使特征高度相关,也能得到稳定解
    \item \textbf{可解释性}:参数保持非零,便于解释特征重要性
\end{itemize}

\textbf{应用场景}:
\begin{itemize}
    \item 特征数量大于样本数量时(高维数据)
    \item 特征之间存在多重共线性时
    \item 需要所有特征都参与预测时
\end{itemize}

\subsubsection{$L_1$ 正则化(Lasso 回归)}

$L_1$ 正则化通过在损失函数中添加权重的绝对值之和来约束参数,能够产生稀疏解。

\begin{definition}[$L_1$ 正则化]
对于线性模型 $f(\mathbf{x}) = \mathbf{w}^T\mathbf{x} + b$,$L_1$ 正则化的损失函数为:

$$L_{\text{Lasso}}(\mathbf{w}, b) = \sum_{i=1}^{n} (y_i - (\mathbf{w}^T\mathbf{x}_i + b))^2 + \lambda \|\mathbf{w}\|_1$$

其中 $\|\mathbf{w}\|_1 = \sum_{j=1}^{d} |w_j|$ 是权重的 $L_1$ 范数。
\end{definition}

\textbf{几何解释}:

$L_1$ 正则化的约束区域是一个菱形(在二维空间中)或超立方体(在高维空间中)。由于约束区域的"尖角"特性,最优解往往落在坐标轴上,导致某些参数恰好为零。

\textbf{稀疏性原理}:

\begin{theorem}[$L_1$ 正则化的稀疏性]
当正则化系数 $\lambda$ 足够大时,$L_1$ 正则化会将某些权重精确地压缩到零,实现特征选择。
\end{theorem}

\textbf{为什么 $L_1$ 能产生稀疏解?}

考虑二维情况,损失函数的等高线和 $L_1$ 约束区域(菱形)的交点:
\begin{itemize}
    \item $L_2$ 约束(圆形):等高线与圆的切点通常不在坐标轴上
    \item $L_1$ 约束(菱形):等高线与菱形的切点很可能在菱形的顶点(坐标轴上),使得某个参数为零
\end{itemize}

\textbf{特点}:
\begin{itemize}
    \item \textbf{特征选择}:自动将不重要的特征权重置为零
    \item \textbf{稀疏解}:产生稀疏的权重向量,模型更简洁
    \item \textbf{可解释性}:只保留重要特征,便于理解模型
    \item \textbf{计算挑战}:损失函数在零点不可导,需要特殊优化方法(如坐标下降、次梯度方法)
\end{itemize}

\textbf{应用场景}:
\begin{itemize}
    \item 特征数量远大于样本数量时
    \item 需要自动特征选择时
    \item 希望模型更简洁、可解释时
    \item 特征之间存在冗余时
\end{itemize}

\subsubsection{Elastic Net}

Elastic Net 结合了 $L_1$ 和 $L_2$ 正则化的优点,既能产生稀疏解,又能处理特征间的相关性。

\begin{definition}[Elastic Net]
Elastic Net 的正则化项是 $L_1$ 和 $L_2$ 正则化的线性组合:

$$L_{\text{Elastic Net}}(\mathbf{w}, b) = \sum_{i=1}^{n} (y_i - (\mathbf{w}^T\mathbf{x}_i + b))^2 + \lambda_1 \|\mathbf{w}\|_1 + \lambda_2 \|\mathbf{w}\|_2^2$$

通常写成:
$$L_{\text{Elastic Net}}(\mathbf{w}, b) = \sum_{i=1}^{n} (y_i - (\mathbf{w}^T\mathbf{x}_i + b))^2 + \lambda \left(\alpha \|\mathbf{w}\|_1 + (1-\alpha) \|\mathbf{w}\|_2^2\right)$$

其中 $\alpha \in [0, 1]$ 控制 $L_1$ 和 $L_2$ 的混合比例。
\end{definition}

\textbf{优势}:
\begin{itemize}
    \item \textbf{结合优点}:既有 $L_1$ 的稀疏性,又有 $L_2$ 的稳定性
    \item \textbf{处理相关性}:当特征高度相关时,$L_1$ 可能随机选择一个,而 Elastic Net 倾向于选择一组相关特征
    \item \textbf{灵活性}:通过调整 $\alpha$ 在稀疏性和稳定性之间权衡
\end{itemize}

\textbf{应用场景}:
\begin{itemize}
    \item 特征数量大于样本数量
    \item 特征之间存在高度相关性
    \item 需要特征选择但也要保持模型稳定性
\end{itemize}

\subsubsection{权重衰减(Weight Decay)}

在深度学习中,$L_2$ 正则化通常被称为权重衰减(Weight Decay)。

\begin{definition}[权重衰减]
在梯度下降更新中,权重衰减等价于在每次更新时对权重进行缩放:

$$\mathbf{w}_{t+1} = \mathbf{w}_t - \eta \nabla_\mathbf{w} L(\mathbf{w}_t) - \eta \lambda \mathbf{w}_t = (1 - \eta \lambda) \mathbf{w}_t - \eta \nabla_\mathbf{w} L(\mathbf{w}_t)$$

其中 $\eta$ 是学习率,$\lambda$ 是权重衰减系数。
\end{definition}

\textbf{通俗解释}:权重衰减就像"记忆衰减"。每次更新参数时,不仅根据梯度调整,还会让参数"忘记"一部分(乘以小于1的因子),防止参数变得过大。

\textbf{在优化器中的应用}:

大多数现代优化器(如 Adam、SGD)都支持权重衰减:

\begin{itemize}
    \item \textbf{SGD with Weight Decay}:
    $$\mathbf{w}_{t+1} = \mathbf{w}_t - \eta (\nabla_\mathbf{w} L(\mathbf{w}_t) + \lambda \mathbf{w}_t)$$
    
    \item \textbf{Adam with Weight Decay}:在 Adam 的更新公式中加入权重衰减项
\end{itemize}

\textbf{注意事项}:
\begin{itemize}
    \item 权重衰减通常不应用于偏置项 $b$
    \item 权重衰减系数需要与学习率一起调整
    \item 过大的权重衰减可能导致欠拟合
\end{itemize}

\subsubsection{早停(Early Stopping)}

早停是一种简单有效的正则化方法,通过在验证集性能不再提升时停止训练来防止过拟合。

\begin{definition}[早停]
早停算法:
\begin{enumerate}
    \item 将数据集分为训练集和验证集
    \item 在训练过程中,定期在验证集上评估模型性能
    \item 如果验证集性能在 $k$ 个epoch内没有提升,则停止训练
    \item 返回验证集性能最好的模型参数
\end{enumerate}
\end{definition}

\textbf{为什么早停有效?}

\begin{itemize}
    \item \textbf{防止过训练}:训练时间过长时,模型可能开始记忆训练数据的噪声
    \item \textbf{自动选择最优模型}:在验证集上性能最好的模型通常泛化能力最强
    \item \textbf{计算效率}:不需要训练到收敛,节省计算资源
\end{itemize}

\textbf{实现细节}:
\begin{itemize}
    \item \textbf{耐心(Patience)}:允许验证集性能不提升的epoch数
    \item \textbf{最小改善(Min Delta)}:验证集性能的最小改善阈值
    \item \textbf{恢复最佳权重}:训练结束后恢复验证集上表现最好的权重
\end{itemize}

\textbf{应用场景}:
\begin{itemize}
    \item 深度学习模型训练
    \item 训练时间有限时
    \item 需要自动选择最优模型时
\end{itemize}

\subsubsection{数据增强(Data Augmentation)}

数据增强通过人工增加训练数据来防止过拟合,是一种隐式的正则化方法。

\begin{definition}[数据增强]
数据增强通过对原始数据进行变换生成新的训练样本,常见变换包括:
\begin{itemize}
    \item \textbf{图像}:旋转、翻转、缩放、裁剪、颜色变换、噪声添加
    \item \textbf{文本}:同义词替换、回译、随机删除、随机插入
    \item \textbf{音频}:时间拉伸、音调变换、添加噪声
\end{itemize}
\end{definition}

\textbf{为什么数据增强有效?}

\begin{itemize}
    \item \textbf{增加数据多样性}:让模型看到更多数据变体,提高鲁棒性
    \item \textbf{隐式正则化}:通过增加数据量,间接减少过拟合风险
    \item \textbf{提高泛化能力}:模型学习到更通用的特征,而不是记忆特定样本
\end{itemize}

\textbf{应用场景}:
\begin{itemize}
    \item 训练数据有限时
    \item 需要提高模型鲁棒性时
    \item 计算机视觉、自然语言处理等领域
\end{itemize}

\subsubsection{正则化方法对比}

\begin{table}[h]
\centering
\begin{tabular}{p{3cm}p{4cm}p{4cm}p{4cm}}
\toprule
\textbf{方法} & \textbf{优点} & \textbf{缺点} & \textbf{适用场景} \\
\midrule
$L_2$ 正则化 & 数值稳定、平滑收缩 & 不产生稀疏解 & 特征相关、需要所有特征 \\
\midrule
$L_1$ 正则化 & 特征选择、稀疏解 & 数值不稳定、优化困难 & 高维数据、特征选择 \\
\midrule
Elastic Net & 结合两者优点 & 需要调两个超参数 & 特征相关且需要选择 \\
\midrule
权重衰减 & 简单有效、广泛支持 & 需要调超参数 & 深度学习模型 \\
\midrule
早停 & 简单、自动选择模型 & 需要验证集 & 训练时间可控 \\
\midrule
数据增强 & 提高鲁棒性 & 需要领域知识 & 数据有限时 \\
\bottomrule
\end{tabular}
\caption{正则化方法对比}
\end{table}

\subsubsection{正则化系数选择}

正则化系数 $\lambda$ 的选择至关重要,通常通过交叉验证来确定。

\textbf{选择方法}:
\begin{itemize}
    \item \textbf{网格搜索}:在预定义的范围内搜索最优 $\lambda$
    \item \textbf{随机搜索}:随机采样 $\lambda$ 值
    \item \textbf{贝叶斯优化}:使用更智能的搜索策略
    \item \textbf{学习曲线}:观察不同 $\lambda$ 下的训练/验证误差
\end{itemize}

\textbf{经验法则}:
\begin{itemize}
    \item $\lambda$ 太小:正则化效果弱,可能过拟合
    \item $\lambda$ 太大:正则化过强,可能欠拟合
    \item 通常从 $10^{-4}$ 到 $10^2$ 的范围开始搜索
\end{itemize}

\subsubsection{正则化在深度学习中的应用}

在深度学习中,正则化技术更加多样:

\begin{itemize}
    \item \textbf{Dropout}:随机丢弃神经元(详见深度学习部分)
    \item \textbf{Batch Normalization}:归一化层输入,具有隐式正则化效果
    \item \textbf{权重衰减}:$L_2$ 正则化在深度学习中的实现
    \item \textbf{标签平滑}:防止模型过度自信
    \item \textbf{噪声注入}:在输入或隐藏层添加噪声
\end{itemize}

这些技术将在深度学习部分详细讨论。

\subsection{偏差-方差权衡}

\begin{definition}[偏差和方差]
\begin{itemize}
    \item \textbf{偏差(Bias)}:模型的期望预测与真实值的差异,衡量模型的拟合能力
    \item \textbf{方差(Variance)}:模型在不同训练集上预测的差异,衡量模型的稳定性
\end{itemize}

总误差可以分解为:
$$\text{Error} = \text{Bias}^2 + \text{Variance} + \text{Irreducible Error}$$
\end{definition}

\textbf{通俗解释}:
\begin{itemize}
    \item \textbf{高偏差}:模型太简单,无法捕捉数据模式(欠拟合)
    \item \textbf{高方差}:模型太复杂,对训练数据的小变化敏感(过拟合)
\end{itemize}

\subsubsection{偏差-方差权衡}

\begin{itemize}
    \item \textbf{简单模型}(如线性回归):
    \begin{itemize}
        \item 高偏差,低方差
        \item 可能欠拟合
    \end{itemize}
    
    \item \textbf{复杂模型}(如深度神经网络):
    \begin{itemize}
        \item 低偏差,高方差
        \item 可能过拟合
    \end{itemize}
    
    \item \textbf{理想模型}:
    \begin{itemize}
        \item 低偏差,低方差
        \item 需要合适的模型复杂度和正则化
    \end{itemize}
\end{itemize}

\section{集成学习方法}

集成学习通过组合多个基学习器来提高预测性能,是机器学习中的重要技术。

\subsection{Bagging}

Bagging(Bootstrap Aggregating)通过训练多个模型并平均预测结果来减少方差。

\begin{definition}[Bagging]
Bagging 算法:
\begin{enumerate}
    \item 使用自助采样从训练集中生成 $B$ 个不同的训练集
    \item 在每个训练集上训练一个基学习器
    \item 对于分类问题使用投票,对于回归问题使用平均
\end{enumerate}
\end{definition}

\textbf{通俗解释}:Bagging 就像多个专家独立给出意见,然后综合所有意见做决策。每个专家看到的数据略有不同,但综合起来更可靠。

\subsubsection{随机森林}

随机森林是 Bagging 的特例,基学习器是决策树,并在特征选择时引入随机性。

\subsubsection{优势与局限性}

\textbf{优势}:
\begin{itemize}
    \item 减少方差,提高泛化能力
    \item 可以并行训练
    \item 对过拟合有抵抗力
\end{itemize}

\textbf{局限性}:
\begin{itemize}
    \item 不能减少偏差
    \item 需要足够的计算资源
\end{itemize}

\subsection{Boosting}

Boosting 通过顺序训练多个弱学习器,每个学习器关注前一个学习器的错误。

\begin{definition}[Boosting]
Boosting 算法:
\begin{enumerate}
    \item 初始化样本权重
    \item 对于 $t = 1, \ldots, T$:
    \begin{itemize}
        \item 使用当前权重训练弱学习器 $h_t$
        \item 计算 $h_t$ 的误差
        \item 更新样本权重(增加错误样本的权重)
        \item 计算 $h_t$ 的权重 $\alpha_t$
    \end{itemize}
    \item 最终预测:$H(\mathbf{x}) = \sum_{t=1}^{T} \alpha_t h_t(\mathbf{x})$
\end{enumerate}
\end{definition}

\textbf{通俗解释}:Boosting 就像学生做错题后,老师重点讲解错题,学生反复练习直到掌握。每个弱学习器专注于前一个学习器的薄弱环节。

\subsubsection{AdaBoost}

AdaBoost(Adaptive Boosting)是经典的 Boosting 算法:

\begin{algorithm}
\caption{AdaBoost}
\begin{algorithmic}
\REQUIRE 训练集 $\mathcal{D} = \{(\mathbf{x}_i, y_i)\}_{i=1}^{n}$,弱学习器 $\mathcal{A}$,迭代次数 $T$
\ENSURE 集成模型 $H$
\STATE 初始化样本权重:$w_i^{(1)} = 1/n, \forall i$
\FOR{$t = 1$ to $T$}
    \STATE 使用权重 $\mathbf{w}^{(t)}$ 训练弱学习器:$h_t = \mathcal{A}(\mathcal{D}, \mathbf{w}^{(t)})$
    \STATE 计算加权误差:$\epsilon_t = \sum_{i: h_t(\mathbf{x}_i) \neq y_i} w_i^{(t)}$
    \STATE 计算学习器权重:$\alpha_t = \frac{1}{2}\ln\left(\frac{1-\epsilon_t}{\epsilon_t}\right)$
    \STATE 更新样本权重:$w_i^{(t+1)} = \frac{w_i^{(t)}}{Z_t} \exp(-\alpha_t y_i h_t(\mathbf{x}_i))$
\ENDFOR
\RETURN $H(\mathbf{x}) = \text{sign}\left(\sum_{t=1}^{T} \alpha_t h_t(\mathbf{x})\right)$
\end{algorithmic}
\end{algorithm}

\subsubsection{梯度提升}

梯度提升(Gradient Boosting)将 Boosting 视为优化问题,使用梯度下降来最小化损失函数。

\textbf{核心思想}:每个新学习器拟合前一个模型的负梯度(残差)。

\subsubsection{XGBoost 和 LightGBM}

\begin{itemize}
    \item \textbf{XGBoost}:优化的梯度提升实现,支持并行计算、正则化、处理缺失值
    \item \textbf{LightGBM}:更快的梯度提升实现,使用基于直方图的算法和 Leaf-wise 树生长策略
\end{itemize}

\textbf{应用场景}:在 Kaggle 等数据科学竞赛中,XGBoost 和 LightGBM 经常是获胜方案的核心组件。

\subsubsection{优势与局限性}

\textbf{优势}:
\begin{itemize}
    \item 通常比单个模型性能更好
    \item 可以减少偏差和方差
    \item 可以处理各种类型的数据
\end{itemize}

\textbf{局限性}:
\begin{itemize}
    \item 训练时间较长(顺序训练)
    \item 对异常值敏感
    \item 可解释性较差
\end{itemize}

\subsection{Stacking}

Stacking(堆叠)使用元学习器来组合多个基学习器的预测。

\begin{definition}[Stacking]
Stacking 算法:
\begin{enumerate}
    \item 使用交叉验证训练多个基学习器
    \item 使用基学习器的预测作为特征,训练元学习器
    \item 最终预测使用元学习器
\end{enumerate}
\end{definition}

\textbf{通俗解释}:Stacking 就像有一个"超级裁判",它不直接看原始数据,而是看各个"专家"(基学习器)的意见,然后综合这些意见做出最终判断。

\subsubsection{算法流程}

\begin{algorithm}
\caption{Stacking}
\begin{algorithmic}
\REQUIRE 训练集 $\mathcal{D}$,基学习器 $\{\mathcal{A}_1, \ldots, \mathcal{A}_M\}$,元学习器 $\mathcal{A}_{\text{meta}}$
\ENSURE 集成模型
\STATE 使用 K 折交叉验证训练基学习器
\FOR{$m = 1$ to $M$}
    \FOR{每折 $k$}
        \STATE 在折 $k$ 的训练集上训练 $h_{m,k}$
        \STATE 在折 $k$ 的验证集上生成预测 $p_{m,k}$
    \ENDFOR
    \STATE 基学习器 $m$ 的完整预测:$\mathbf{p}_m = [p_{m,1}, \ldots, p_{m,K}]$
\ENDFOR
\STATE 使用 $\{(\mathbf{p}_1, \ldots, \mathbf{p}_M), \mathbf{y}\}$ 训练元学习器
\RETURN 集成模型(基学习器 + 元学习器)
\end{algorithmic}
\end{algorithm}

\subsubsection{优势与局限性}

\textbf{优势}:
\begin{itemize}
    \item 可以捕捉基学习器之间的互补性
    \item 通常性能优于单个模型
    \item 灵活性高,可以使用不同类型的基学习器
\end{itemize}

\textbf{局限性}:
\begin{itemize}
    \item 计算成本高
    \item 需要仔细设计,避免过拟合
    \item 可解释性差
\end{itemize}

\section{在线学习与增量学习}

在线学习和增量学习使模型能够从新数据中持续学习,适应数据分布的变化。

\subsection{在线学习}

\begin{definition}[在线学习]
在线学习(Online Learning)是一种学习范式,模型逐个处理样本,每处理一个样本就更新模型参数,不需要存储所有历史数据。

给定数据流 $\{(\mathbf{x}_1, y_1), (\mathbf{x}_2, y_2), \ldots\}$,在线学习算法在时刻 $t$:
\begin{enumerate}
    \item 接收样本 $(\mathbf{x}_t, y_t)$
    \item 使用当前模型 $f_t$ 进行预测
    \item 根据预测误差更新模型:$f_{t+1} = \text{Update}(f_t, (\mathbf{x}_t, y_t))$
\end{enumerate}
\end{definition}

\textbf{通俗解释}:在线学习就像实时学习,每来一个新例子就立即学习,不需要等所有数据都收集完。就像学生每做一道题就立即知道答案并学习,而不是等所有题目做完再统一学习。

\subsubsection{在线梯度下降}

在线梯度下降是随机梯度下降的在线版本:

$$\mathbf{w}_{t+1} = \mathbf{w}_t - \eta_t \nabla_{\mathbf{w}} \ell(f(\mathbf{x}_t; \mathbf{w}_t), y_t)$$

其中 $\eta_t$ 是学习率,通常随时间衰减。

\subsubsection{应用场景}

\begin{example}[推荐系统]
在线推荐系统需要实时响应用户行为,根据用户的实时反馈更新推荐模型。例如,用户点击了某个商品,系统立即更新该用户的兴趣模型。
\end{example}

\begin{example}[广告投放]
在线广告系统需要根据实时点击率调整广告投放策略,快速适应市场变化。
\end{example}

\subsubsection{优势与局限性}

\textbf{优势}:
\begin{itemize}
    \item 内存效率高(不需要存储所有数据)
    \item 可以快速适应数据分布变化
    \item 适合大规模数据流
    \item 可以实时更新模型
\end{itemize}

\textbf{局限性}:
\begin{itemize}
    \item 对异常值敏感
    \item 可能遗忘历史信息
    \item 需要仔细设计学习率
    \item 难以评估模型性能
\end{itemize}

\subsection{增量学习}

\begin{definition}[增量学习]
增量学习(Incremental Learning)是模型在已有知识的基础上,从新数据中学习新知识,同时保留或整合旧知识的过程。

与在线学习的区别:增量学习通常处理批量新数据,并且需要处理"灾难性遗忘"问题。
\end{definition}

\textbf{通俗解释}:增量学习就像人类学习新知识。学习新内容时,不会完全忘记旧知识,而是将新旧知识整合在一起。但机器学习模型容易"遗忘"旧知识,需要特殊技术来解决。

\subsubsection{灾难性遗忘}

灾难性遗忘(Catastrophic Forgetting)是指模型在学习新任务时,会大幅降低在旧任务上的性能。

\textbf{原因}:神经网络参数在训练新任务时被更新,可能破坏对旧任务的记忆。

\subsubsection{解决方法}

\begin{itemize}
    \item \textbf{弹性权重巩固(EWC)}:在更新参数时,对重要参数施加惩罚,防止大幅改变
    \item \textbf{渐进式神经网络}:为每个任务添加新的网络分支,保留旧网络不变
    \item \textbf{回放机制}:存储部分旧数据,与新数据一起训练
    \item \textbf{知识蒸馏}:使用旧模型指导新模型学习
\end{itemize}

\subsubsection{应用场景}

\begin{example}[持续学习系统]
智能助手需要不断学习新技能,但不能忘记已有技能。例如,学习新语言时不能忘记已掌握的语言。
\end{example}

\begin{example}[个性化推荐]
推荐系统需要适应用户兴趣的变化,同时保留对用户长期偏好的理解。
\end{example}

\section{总结}

本文档系统性地介绍了机器学习的核心理论与应用,包括:

\begin{itemize}
    \item \textbf{机器学习基础}:监督学习、无监督学习和强化学习三大范式,每种范式适用于不同类型的问题
    \item \textbf{经典算法}:线性回归、决策树、随机森林和支持向量机等基础算法,虽然简单但在许多场景中仍然有效
    \item \textbf{特征工程}:特征选择、变换和编码等技术,是提升模型性能的关键
    \item \textbf{模型评估}:交叉验证、过拟合/欠拟合识别、偏差-方差权衡等评估技术
    \item \textbf{集成学习}:Bagging、Boosting 和 Stacking 等方法,通过组合多个模型提高性能
    \item \textbf{在线与增量学习}:使模型能够从新数据中持续学习,适应动态环境
\end{itemize}

\textbf{机器学习的核心价值}:
\begin{itemize}
    \item 能够从数据中自动学习模式,减少人工规则设计
    \item 可以处理高维、复杂的现实世界数据
    \item 通过持续学习适应环境变化
    \item 在多个领域取得了突破性进展
\end{itemize}

\textbf{未来发展方向}:
\begin{itemize}
    \item 自动化机器学习(AutoML):减少人工调参和特征工程
    \item 可解释性:提高模型的可解释性和可信度
    \item 持续学习:更好地处理数据分布变化和任务演化
    \item 小样本学习:在数据稀缺场景下仍能有效学习
    \item 联邦学习:在保护隐私的前提下进行分布式学习
\end{itemize}

机器学习作为人工智能的核心技术,将继续在各个领域发挥重要作用,推动技术进步和社会发展。

